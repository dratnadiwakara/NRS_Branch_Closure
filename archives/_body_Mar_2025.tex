
\section*{Introduction}

This paper examines the drivers of bank branch opening and closure decisions during the industry restructuring period from 2001 to 2023. We show that measures of local financial sophistication strongly predict both outcomes, as depositors in these areas exhibit greater interest-rate sensitivity, and as a result, lower deposit franchise value. The findings highlight the different incentives facing incumbent banks versus potential entrants. Incumbents are more likely to close branches in areas with rate-sensitive customers, from whom they have limited ability to extract rents. In contrast, potential entrants are less likely to open branches in areas with rate-insensitive depositors, as such customers are difficult to attract away from incumbent institutions despite their profitability.

Bank branches grew in aggregate until 2010, with the rate of branch openings exceeding that of closings by a factor of two to three. This net branch growth occurred even as the banking system consolidated. Total branches and offices, however, peaked in 2010 and then began to decline. The rate of decline accelerated around 2015 and further intensified following the COVID-19 pandemic (Figure \ref{fig:no_of_banks_branches}). These recent patterns stand in sharp contrast to the earlier period, when openings consistently outpaced closings (Figure \ref{fig:no_of_banks_branches}, Panel B). Before the Global Financial Crisis (GFC), a forty-year period of banking-industry restructuring unfolded as large banks expanded into new markets by acquiring smaller incumbent banks and by opening new branches. This expansion was fostered by the deregulation of restrictions on branching and interstate banking (\citet{Schneider2025}). Figure \ref{fig:no_of_banks_branches} shows that this restructuring sharply reduced the number of banks (due to M\&A), but not the number of branches. The purpose of this first wave of M\&A was fundamentally to extend large banks’ physical reach (\citet{kroszner2014regulation}). By contrast, restructuring since 2010 has reduced both the number of banks and the number of branches.

We argue that the primary driver of restructuring today has been technology, which allows customers to access their deposits without close proximity via internet and mobile banking, and also increases their ability to move money in response to shocks (a la SVB) or changes in market interest rates.  These changes reduce the profitability to banks of operating a dense branch network.  While the inexorable expansion of technology reduces branching across the board (even though some branches continue to be opened), our empirical strategy exploits the more rapid adoption of technology by financial sophisticated customers to explain why some branches are closed before others.  The current restructuring stands in contrast to the earlier period, when deregulation was the main driver. Thus, our paper contributes to explaining the recent decline in bank branch networks and the broader restructuring of the banking industry.

 
For incumbent banks (i.e., those who may close branches), our empirical design exploits
variation in deposit and lending conditions across each bank’s pre-existing network. For each branch owned by a given bank, we first construct a measure of interest-rate sensitivity (the Deposit-$\beta$).  This measure, as shown in Drechsler, Savov, and Schnabl (2017), represents a sufficient statistic for local market power, as it captures “not only the impact of concentration but also of depositors’ financial sophistication, attentiveness, and willingness to switch banks.”  We extend their concept by first constructing a bank-level predicted rate sensitivity from average local demographic characteristics, and second predicting each branch’s $\beta$ based on demographics in its geographic location. This strategy provides variation in branch-level deposit valuation \textit{within} each bank. With the branch-based measure of market power, we can understand how a given bank decides which of its branches to close as a function of the price sensitivity of that branch's local depositor base. Put differently, we estimate all our branch closure and opening models with bank (and bank-time) fixed effects.


For potential entering banks, we construct a set of candidate zip codes for each bank consisting of all zip codes located within CBSA areas where the bank owns at least one branch in the prior period, as well as all zip codes located in CBSAs where the bank opens a new branch.  As in  the closure analysis, we examine opening decisions within bank-year. For each candidate zip code, we assign the same Deposit-$\beta$ measure described above, which in this context reflects price sensitivity a bank would expect  if it decided to enter the zip code by opening a new branch.

We use our indirect approach because direct information on deposit rate sensitivity at the branch level is not available systematically. Recent research has used deposit quotes from RateWatch, but these data are only meaningful for a subset of branches (“rate setting branches”). And, as argued by Begenau and Stafford (2023), many large banks engage in uniform pricing, whereby they extend the same rate over a large region. Crucially, the branch-based Deposit-$\beta$ captures market power at the branch level – meaning, the profits the bank can extract from each branch. Our strategy works regardless of whether or not a bank engages in uniform pricing (which itself is unobservable). Banks which price differentially across their branches would experience lower deposit spreads at their high-$\beta$ branches compared to low-$\beta$ ones.  Banks which instead use uniform pricing would experience larger losses of funds at their high-$\beta$ branches, especially during periods of rising rates when investment opportunities are strong, because local competitors would offer better rates to local customers or price sensitive customers would move funds into market substitutes. Some uniform-pricing banks might take steps to retain deposits in their high-$\beta$ branches by offering ancillary services or with additional advertising and marketing efforts, but such actions would be costly.\footnote{This kind of behavior was used widely during the late 1970s to alleviate disintermediation from binding interest rate caps on deposits. Some banks went so far as offering expensive gifts such as televisions to retain deposits.}
 

Our location-based Deposit-$\beta$ strongly predicts both branch closures and openings, but operates differently across the two margins. Incumbent banks are less likely to close branches in areas with price-insensitive customers (low $\beta$), as such customers are especially profitable and easy to retain. In contrast, potential entrants are less likely to open branches in these same areas, as low rate sensitivity makes depositors harder to attract from incumbents. These patterns help explain why the industry has been—and will likely continue to be—reshaped. The widespread adoption of phone and internet banking reduces frictions associated with geographic distance and weakens banks’ local pricing power over deposits \citep{haendler2022keeping,jiang2022bank,koont2023digital}. Consistent with this, we find that closures are more common in areas with highly educated residents and those more exposed to the stock market—individuals more likely to adopt digital banking technologies—resulting in high Deposit-$\beta$s. We further document that individuals in these areas visit branches less frequently and travel greater distances when they do—reducing the strategic value of maintaining a physical presence.


We also find that the marginal impact of the deposit rate sensitivity on closures increases sharply after the COVID-19 pandemic. The pandemic was a `teachable moment' for many people, who learned that technology can effectively substitute for physical proximity. Work at home became prevalent, and this change forced many to rearrange their lives in ways that emphasized online rather than in-person interactions. 

The relationship between Deposit-$\beta$ and branch restructuring—both closures and openings—is stronger for large banks than for small ones. This difference is especially evident in the years following the COVID-19 pandemic, which marked a sharp increase in branch closures, particularly among large banks. Large banks responded more when deciding which branches to close or where to open new ones, whereas small banks exhibit weaker and less consistent patterns. These differences likely reflect underlying variation in customer composition and strategic focus: large banks tend to attract more financially sophisticated customers—those with higher income, education, and digital adoption—who are more interest-rate sensitive. This sorting is consistent with evidence from \citet{narayanan2024depositor}, who use cell phone data to show that large-bank branches are disproportionately visited by customers from more affluent and educated areas. Similarly, \citet{d2023deposit} find that large banks serve clientele who demand a broader set of financial services beyond deposit-taking. \cite{kundu2024diverging}, also studying the largest banks, find larger declines in branches among those paying high deposit rates.  As a result, large banks are more active in adjusting their branch networks in response to changes in the profitability of their deposit base.

%Variation in the value of the DF also matters more for large banks’ branching decisions than it does for small ones. The starkest difference appears in the years following the Pandemic. The Pandemic year itself came with the largest overall decline in branches, with almost 8\% of large-bank branches being closed (Figure \ref{fig:pct_branches_closed}). For large banks, during the 2020-2023 regime a one-standard-deviation decrease in the DF raised the probability of a branch closure by about 1 percentage point (or about one-quarter of the mean closure rate during these years), with a much smaller impact on smaller banks. This is likely due to the difference that reflects the sorting of customers within localities across banks based on size, whereby the more sophisticated customers gravitate toward large banks. Consistent with this idea, \citet{narayanan2024depositor} use cell phone data to link customers' characteristics to the branches that they visit. This paper shows that large banks attract customers living in areas with higher levels of income, education, and other measures of financial sophistication. Similarly, \citet{d2023deposit} argue that large banks cater to different clientele who value more complex services that go beyond deposits, compared to smaller banks.

Compared to deposits, we find at best only weak evidence that lending variables can explain branch restructuring. This is surprising given that much of the prior banking literature has demonstrated the importance of physical distance between bankers and borrowers. (e.g., \citet{petersen2002does,berger2005does}). But technology has significantly reduced the importance of distance. Until recently, most business sales were conducted largely from cash, which necessitated close physical proximity to a bank branch, if nothing else as a means to safeguard cash receipts. Today, businesses accept an increasing fraction of their sales from electronic payments (as opposed to cash), reducing the need for local branches for security-related purposes. Beyond that, the information environment has also changed significantly, again because payment flows are now dominated by electronic means.\footnote{Penetration of phone-based payments technologies has been faster in many parts of the world than in the US. As a result, for example, the number of branches per capita has fallen twice as fast in the EU as in the US. In the Netherlands, to take an extreme case, the number of bank branches has fallen by 85\% (https://fred.stlouisfed.org/series/DDAI02NLA643NWDB).
}  As such, physical proximity no longer matters much for information production.\footnote{\citet{buchak2018fintech} focus on the increasing market share of fintech lenders in the mortgage space and \citet{gopal2022rise}, who report a high and growing share of lending to small businesses by non-banks. For a review of the growing role of Fintech lenders generally, see \citet{berg2022fintech}} For these reasons, we argue that the demand for lending does not help explain branch restructuring because bank location matters little for effective credit provision by banks (or other lenders, such as Fintechs).\footnote{Note that we are not claiming that bank relationships no longer matter; nor are we arguing that bank-borrower lending relationships are no longer sticky, as has been documented across many studies. We are instead arguing that the physical distance between banks (or bankers) and borrowers matters much less than in the past.
} 

In the last part of our analysis, we use cell phone mobility data to track the usage of bank branches, focusing on the decline in the number of visitors around the Pandemic, as well as the distance those visitors travel to visit a branch. These metrics help predict bank branch closures but are less helpful in explaining openings.  Adding them to our model attenuates the impact of the value of the deposit franchise itself only slightly, and it continues to exhibit strong explanatory power. This occurs, we show, because both branch usage and rate sensitivity are strongly correlated, as both are driven by local demographics. These findings close the loop on our core argument: areas with sophisticated residents substitute away from brick-and-mortar banking and into technology. This behavior reduces the value of their deposits to banks because customer facility with technology increases their interest-rate sensitivity, and because technology reduces the amenity value to customers of close proximity to a bank branch. People who rely mainly on mobile apps and the internet to access their bank do not value a nearby branch. As a result, demographic variables drive all three outcomes: branch usage, interest-rate sensitivity, and branch closures.

We contribute to a nascent literature studying drivers of branch closures. \citet{keil2024demise} study the de-branching regime as we do, but that paper emphasizes variation across banks in technology adoption. Similarly, both \citet{haendler2022keeping} and \citet{jiang2022bank} show that banks offering customers online or phone-based access are more likely to close branches. \citet{koont2023digital} argues that bank investment in digital technologies leads to branchless competition. Our approach takes banks’ investments in technology as given (by absorbing bank-time effects) and instead focuses on how branch-level variation in the customer base explains closures. As such, our work is complementary to theirs. The core difference is that our effects depend on variation in customer adoption of technologies – that is, customer demand for non-branch access to their funds - which lowers the value of bank branches. We shut down variation in the supply of technology by controlling for bank-year and county-year fixed effects. 

Evidence from other jurisdictions supports the view that tech-savvy customers lower the value of a physical branch location.  \citet{yuan2023your} show that younger customers (Gen Z) who switch to similar banking products offered by Fintech firms leads to a rise in both the number and share of branch closures in China. \citet{zimin2022profitability} show that higher levels of financial digital literacy are related to de-branching in Russia.  Consistent with our arguments, \cite{benmelech2023bank} and \cite{koont2024destabilizing} argue that digital banking reduces depositor stickiness, both with regard to changes in market interest rates and also around the SVB crisis.

Our paper complements \citet{kumarjames2025lowinterest}.  While our paper focuses on understanding which, among a given bank’s existing branch network, that bank chooses to close, theirs focus is on the role of the deposit franchise in explaining closure rates across banks.  Like \citet{sarto2023secular}, they emphasize that banks lose value in the low interest rate environment after the GFC because deposit spreads over market rates compress.  Our empirical design complements theirs by absorbing these macro-level effects and instead focusing on how variation in the customer base influences branch-level closure decisions.

In contrast to our results, where we find little evidence that local lending affects branch closures, \citet{cespedes2024branching}, who study an earlier period (2011-2017), find that shadow bank entry affects bank closures by lowering local residual loan demand, especially at banks with a high cost of violating the Community Reinvestment Act (CRA). Their evidence, along with ours, suggests that the impact of lending may have changed in recent years, as information technology has become increasingly important. 

To the best of our knowledge, ours is the first paper to model bank closure and opening decisions separately, rather than focus on the \textit{net} change in bank branches within specific localities. Our results underscore the value of this approach,  as incumbents and potential entrants face opposing incentives.  For incumbents, areas where customers have low interest-rate sensitivity are attractive because they can extract rents from ‘sticky’ customers by pricing deposits well below market rates.  Such areas, however, are not attractive to potential entrants, since sticky customers are hard to draw away from the incumbents.  Our results imply that the Deposit-$\beta$ is a much better measure of local market power than the traditional metric based on concentration (HHI).  Low $\beta$ represents two factors which reinforce market power: 1) incumbent banks are able to exploit rate-insensitive people (this is the channel emphasized by \cite{drechsler2017deposits}); and, 2) low $\beta$ also raises entry barriers because potential-entrant banks know they will have a hard time drawing deposits away from incumbents.

Our motivation has been the de-branching in the US since the GFC.  But this de-branching is likely not complete; in fact, the decline in branching has occurred almost twice as fast in the EU.  Some areas, such as the Scandinavian countries, have seen branching fall by 90\%. Information and communications technology has made bank branches less important, but our results show that banks recognize different rates of customer technology adoption and adapt by closing branches strategically, starting with the least profitable ones. The earlier literature argued that the ownership and location introduced important frictions in banking services. These frictions are being eroded by information technology, which makes distance and physical proximity between banks and their customers increasingly irrelevant.\footnote{ For example, areas with highly concentrated ownership of branches experienced less competition in both deposit and lending markets. Closure of branches reduced local small business lending (\citet{nguyen2019credit}). Moreover, the flow of capital across markets was affected by connections between local areas from branch ownership networks \citep{gilje2016exporting,cortes2017tracing}. The most powerful evidence of the importance of bank branching emerged from studies of deregulation of restrictions on the ownership of bank branches both within and across state lines \citep{jayaratne1996finance,rice2010does}. Following such deregulation, credit and deposit market competition improved, capital mobility increased, and the supply of deposit and credit services increased.} In the end, we believe branch networks will be much smaller and less important in determining deposit pricing and credit availability.


\section{A Framework to Understand Branch Restructuring}
Innovations in both payment systems (e.g., PayPal, Venmo, Zelle, etc.), along with easy access to deposits via the internet and smartphones, have increased depositor sensitivity to market rates (thereby lowering bank pricing power) and also reduced the value to them of close proximity to a bank branch. We take these technological changes in the basic financial infrastructure as given, but exploit the fact that customer adoption of these technologies exhibits substantial heterogeneity. Younger and better-educated households, for example, interact with their banks via technology at higher rates and earlier than older and lower-income households (\citet{FDIC2023}). Hence, differences in the local demographics faced by different branches will drive variation in the value of those branches. We can represent the relationships we have in mind schematically as follows:
\begin{itemize}
    \item Local Demographics $\rightarrow$ Changes in the branch-level Deposit Franchise Value
    \item Local Demographics $\rightarrow$ Changes in Usage of Branches
    \item  Low DF and Low Branch Usage $\rightarrow$  More Branch Closures
     \item  Low DF and High Potential Branch Usage $\rightarrow$  More Branch Openings
\end{itemize}

These relationships capture the central mechanism in our analysis. First, local demographics—such as age, education, income, and stock market participation—affect how depositors interact with banks. In areas with more financially sophisticated households, depositors are more likely to adopt digital technologies, resulting in higher interest-rate sensitivity and a lower deposit franchise value (DF). Second, this same demographic variation drives patterns of branch usage: people in sophisticated zip codes visit branches less often and travel farther when they do, consistent with their substitution toward online and mobile banking channels. Third, branches that serve such communities—those with both low DF and low usage—are prime candidates for closure, as they offer less pricing power and limited in-person engagement. Finally, areas with low DF but relatively higher potential usage (e.g., less sophisticated markets not yet fully penetrated) may still be attractive for entry, as new branches in those areas can target customers who remain dependent on physical access but are not yet served by the bank. 

This framework underlies our empirical design and helps explain the observed heterogeneity in restructuring decisions across banks, branches, and local markets.
Our empirical section below estimates these relationships in three steps: 1) we first estimate the relationship between local demographics and the DF at the bank level, and then use that model to compute DF at the branch level; 2) we then estimate how demographics affect branch usage based on cell phone data; 3) we estimate how DF and branch usage affect branch restructuring decisions. As argued by \citet{egan2022cross}, most of the value created by banks stems from payments-related services, which allow banks to raise and retain deposits at interest rates well below market interest rates (\citet{lu2024making}). \citet{drechsler2017deposits} show how this pricing power over deposits leads to a novel channel of monetary policy transmission driven by the cyclicality in bank deposit pricing. As such, we focus most of our attention on changes in the value to banks of paying below-market interest rates on deposits. In our last set of tests, we introduce branch usage metrics to the models, which are only available during the most recent portion of our sample.

\section{Data Sources}
We combine data from several sources to construct our analytic samples. This section describes the underlying datasets and how we use them in the analysis.

\subsection{Summary of Deposits}
We use the FDIC’s annual \textit{Summary of Deposits} (SOD), which allows us to measure the amount of deposits and location of each bank’s branch network in June of each year, and also to observe branch openings and closings. We estimate our restructuring models during the years from 2001 to 2023. A branch is `closed’ ('opened') in year \textit{t} if it appears (does not appear) in June of year \textit{t} in the SOD dataset but does not (does) appear in the June of year \textit{t+1}. This definition can be established with certainty not only because the SOD contains a branch-based ID variable, but also because it contains detailed data on each branch’s physical location (e.g., latitude and longitude, as well as state, city and street address). Figure \ref{fig:pct_branches_closed} reports the fraction of branches closed (Panel A) and opened (Panel B) in each of the years we study, split by bank size (with large defined as banks with more than \$100 billion in assets).  Branch opening rates exceed that of closings in every year prior to the GFC, and vice versa in every year after.\footnote{We have verified that branches identified as openings are indeed new, as opposed to a branch which may have been closed and subsequently purchased by a bank entrant.}  Branch closures spike during the year of and following the COVID-19 Pandemic, with large banks closing nearly 8\% of their branches in 2020 - where again we define 2020 as the period between June 2020 and June 2021.


\subsection{Bank Call Reports}

The Federal Financial Institutions Examination Council (FFIEC) requires US banks to file information on their financial health and performance at the end of each quarter and these are made publicly available. These ``Call Reports'' provide a breakdown of balance sheets and income statements. %For our purposes, we obtain bank yearly information on assets, deposits, and interest expense on deposits (since we can only measure branch closures on an annual basis).

\subsection{Demographic data}

To capture local demographic factors, we use the American Community Survey (ACS) 5-Year Data, which provides economic and socioeconomic information across various geographical levels in the United States. Because ACS data begin in 2009, we use 2009 values—based on 2005–2009 averages—to proxy for earlier years in our analysis. We use the census tract level information on income, education, and age, and we also capture a measure of stock market participation using zip code level data from the IRS Statistics of Income (SOI) on Individual Income Tax Returns, specifically the fractions of tax returns reporting dividend income and capital gains.

\subsection{Branch Usage data}

We use the Advan (formerly SafeGraph) Monthly Patterns dataset, which provides aggregated raw counts of visits to points of interest (POIs) in the US, gathered from a panel of mobile devices. This anonymized and aggregated dataset provides details on monthly visitor frequency, duration, the origin census block group, and the distance the median branch visitor traveled. The dataset initiates from January 2019 and ends in 2023; however, we do not have usage data during the height of the pandemic. We use these data to identify how many customers use each bank branch and how far the median branch visitor traveled. Because of its limited coverage of branches closed prior to 2022, and because we lag the usage measures, we report branch closure models controlling for usage only for 2022 and 2023.

\subsection{Other Data Sources}

In addition to the primary datasets outlined above, we incorporate several other data sources to construct control variables for our regressions. The Home Mortgage Disclosure Act (HMDA) data is used to calculate mortgage loan growth at the county level, while the Community Reinvestment Act (CRA) data is used to calculate small business loan growth at the county level. County Business Patterns (CBP) data is used to calculate establishment growth and payroll growth at the county level. Additionally, the Federal Housing Finance Agency (FHFA) Underserved Areas data is used to create a zip code-level dummy variable indicating whether a zip code is classified as a low-to-moderate income (LMI) area, defined as having a median income below 80\% of the area median income.  


\section{The Cross Section of The Deposit Franchise (DF)}

\subsection{Bank-Level Deposit Franchise and Its Determinants}

We follow \cite{dss2023}, who construct a simple measure of bank’s deposit franchise value, equal to the present value of gains associated with pricing the current stock of deposits at below-market rates of interest (so the deposit interest rate $r_i^d < r^f$, the market interest rate).\footnote{The DSS framework does not capture value created by expectations of future deposit growth. Since banks would clearly care about growth, we capture this effect (crudely) by including past growth of total deposits (as well as small business loans and mortgages) into our closure models.} We follow \citet{dss2023} in assuming that banks set their deposit rate equal to a fixed fraction of the market rate (so, $r_i^d = \beta_i \times r^f$ ), and that banks with different depositor clientele and banks operating in markets with different levels of competition will optimally choose different deposit betas. Also following \citet{dss2023}, we assume deposits run off equally in each year over the next 10 years. With these assumptions, the DF for bank \textit{i }is given by:


\begin{equation}
DF_i = (1-\beta_i - \frac{c_i}{r^p}) \times [1-\frac{1}{(1+r^p)^{10}}]
\end{equation}


where $\beta_i = \Delta r_i^d/ \Delta r^f$,  $r_i^d$ is the interest expense on deposits per dollar of deposits, $r^f$ is the Fed Funds rate, and  $r^p$ is the the long-term interest rate, which is assumed to be 2.5\%. 

In our application, we set \textit{c} (the annual operating cost per dollar of deposits) to zero, as we do not have good ways to capture the annual per-dollar operating costs associated with each branch. To the extent that these costs are similar across branches for each bank, they would have no effect on our closure model because we estimate all of our effects within bank (and year). In other words, the costs would be absorbed by the bank-year fixed effects. Consistent with this claim, \citet{narayanan2024depositor} show no relationship between customer demographics and bank non-interest expenses (a rough measure of \textit{c}).

We estimate $DF_i$ per dollar of deposits across the last three monetary tightening cycles, using bank-level realizations of $\beta_i$: first for the 2004–2006 cycle, second for the 2016–2019 cycle, and third for the 2022–2024 cycle, applying a consistent estimation framework in each period. We first estimate cross-sectional regressions to reveal how local factors affect the deposit $\beta$. This allows both the mean level of $\beta$ to evolve over time and also allows the cross-sectional effects of demographics and concentration to vary over time. Realized $\beta_i$ equals the change in bank annualized interest expenses per dollar of deposits over each cycle (from Bank \textit{Call Reports}), normalized by the change in Federal Funds rate over that cycle (= 4.25\% in the 2004-2006 cycle, 2.5\% in the 2016-2019 cycle, and 4\% during 2022-2023). We use just the increasing portion of each rate cycle. The end of both the 2004-2006 and 2016-2019 cycles coincide with crises (GFC and the COVID Pandemic), making the subsequent rate declines and bank reaction to those declines difficult to interpret, and not reflective of their normal response to market-rate changes.\footnote{Banks received massive deposit inflows during this time due to large transfer payments to households, workers and firms under the CARES Act. These exogenous shocks to deposits may disturb the normal pricing reaction of banks to a decline in interest rates which would not be a good representation of their pricing power.} The sample period in the second cycle ends at the first quarter of 2023 to ensure that changes following the Silicon Valley Bank (SVB) collapse do not impact our estimations. 

To build cross-bank local drivers of $\beta$, we average the demographic and market characteristics of residents living near each bank’s branch (based on zip code), weighted by the amount of deposits each bank holds in each of its branches.\footnote{We rely on the pooled models (i.e., large and small banks together) to identify the effects of local variables on the deposit franchise. Estimating these models for large banks alone would be problematic because they own branch networks distributed widely across the country, thus restricting the variation in their exposure to (averaged) local factors.} These regressions are structured, as follows:

\begin{equation}
    \beta_{i,t} = \sum \gamma_t^k D_{i,t}^k + \eta_t HHI_{i,t} +\text{Other controls}+\varepsilon_{i,t}
\end{equation}

where \textit{i} represents bank, \textit{t} represents one of the three rate cycles, \textit{k} represents four demographic variables: age (using three quartile-binned indicators), log of mean family income, the fraction of tax filers reporting stock-based income, and the fraction with a college degree. We average each of these demographics across each zip code in which bank \textit{i} owns its branches, weighted by deposits in each branch. In addition, we capture the deposit-weighted average level of concentration across each bank’s markets (\textit{HHI\textsubscript{i,t}}), where markets are defined at the county level. The other control variables include bank-level population density, calculated as the county-level population density weighted by deposits in each county, a measure of bank size (log of total assets), transactions deposits / assets, plus a constant. The dependent variable in Equation (2) equals the change in the interest expenses on deposits per dollar of deposits in each cycle scaled by the corresponding change in the Fed Funds rate.

Table \ref{tab:bank_desc_stats} reports summary statistics for the regression samples in the three cycles, separated for large ($>$\$100 billion) and small banks. Explanatory variables are measured at the beginning of each rate cycle. For small banks, the mean realized $\beta_i$ ranges from 0.17 to 0.23; for large banks, the $\beta_i$  ranges from 0.24 to 0.34.\footnote{In principle, one could build an analogous metric based on loan-market pricing power.  However, Call Report measures of loan pricing, such as average interest income on C\&I loans, would be driven mainly by large borrowers; such loans would not reflect local pricing power.  Moreover, most of the variation in observed lending rates reflects differences in risk rather than mark-ups from market power.  Hence, as we describe below we use quantity-based measures of local lending conditions from both the mortgage markets and the market for loans to small businesses.}  Large banks operate in areas with younger, more educated, wealthier populations that have higher rates of stock market participation than small banks. These demographic differences do not vary much over time across large and small banks. Small banks are more likely to have branches in rural areas.

Figure \ref{fig:df_density} reports histograms of the bank-level deposit franchise (DF) values (Panel A), split by the three monetary tightening cycles and by bank size. Large banks have lower DF on average in all three cycles (recall Table 2) because they have higher $\beta$s, although there is substantial overlap in the distributions.

Table \ref{tab:deposit_beta_reg} reports estimates of Equation (2) for the three rate cycles. Age, income and education are all correlated with bank pricing power as expected (columns 1-3). Banks with (potential) customers near their branches who are younger and more highly educated have \textit{lower} pricing power (higher $\beta$). Banks have higher pricing power in areas with higher income. The age effect is driven by the oldest quartile. Increasing the share of a bank’s clientele with a college degree by one sigma (from the large-bank sample) raises $\beta$ by 0.021 (=0.16 x 0.15) in the third cycle, for example. While market concentration (HHI) enters both models negatively (suggesting greater bank pricing power in more concentrated areas), its effects lose statistical significance in the third cycle. In contrast, local population density – which itself is strongly correlated with HHI - has a large impact which increases in importance over time. Large banks also exhibit much higher $\beta$s than small, and this effect also increases in the last cycle.\footnote{Note that we absorb size effects in our closure models with fixed effects.}

Columns 4-6 of Table \ref{tab:deposit_beta_reg} report a more parsimonious model which collapses two demographic characteristics into one: the fraction of residents in ‘sophisticated’ zip codes. We build the zip-code classification by flagging localities with above-median education and above-median stock market participation. We use this indicator to differentiate areas dominated by financially sophisticated people versus those without. We include age and income in these models as separate factors. As the results show, the deposit $\beta$ is consistently higher in sophisticated areas, and the impact of financial sophistication increases in the last two cycles relative to the first. For context, a bank raising all of its deposits in areas with financially sophisticated depositors would have 2\%-3\% lower present-value of profits generated from each dollar of deposits, compared to a bank raising all of its deposits in areas with less-sophisticated depositors.

\subsection{Predicted Deposit Franchise Values at the Branch Level}
We construct branch-level measures of the DF by applying the coefficients from Equation (2) to each bank’s branches using the demographic measures from each branch’s zip code (opposed to the average across all branches, as in Equation (2)). We use the coefficients from the first interest-rate cycle for the years 2001-2014, from the second cycle for 2015-2019, and the coefficients from the third cycle for the years 2020-2023.\footnote{Our aim is the allow the marginal effects of customer demographics to shift over time. We recognize that the coefficients we use are not strictly out-of-sample. Changing this mapping has little effect on our results because the coefficients are fairly stable, and because the demographic variables are very persistent.} Since banks own branches in zip codes with different demographic factors, this procedure generates within-bank variation in the value of the deposit franchise.

Panel B of Figure \ref{fig:df_density} shows the distribution of predicted deposit franchise values at the branch level. The densities reveal substantial within-bank heterogeneity, reflecting variation in local demographic characteristics across branch locations. This variation provides the identifying power for our branch-level analysis of closure and decisions, as it allows us to compare branches within the same bank that differ in the profitability of their deposit base.

The persistence of branch-level differences over time is illustrated in Figure \ref{fig:df_scatter}, which compares predicted DF values across interest rate cycles. Panels A and B plot branch-level predicted DF for the 2016–2019 vs. 2022–2024 and 2004–2006 vs. 2016–2019 cycles, respectively. Panels C and D show the same comparisons using actual bank-level DF. As expected, the cross-section of DF is highly correlated over time, with a tighter relationship at the branch level. This reflects the stability of both the explanatory variables (demographics change slowly) and the coefficient estimates, as documented in Table \ref{tab:deposit_beta_reg}. Put simply, branches with above-average DF early in the sample tend to remain high-DF in later years. The figure also shows that DF values are generally lower in the post-pandemic period, as the regression line falls below the 45-degree line. Between the second and third cycles, the fitted line rotates downward around the 45-degree line, consistent with a sharp increase in the effects of bank size and population density on bank-level $\beta$s. Larger banks and those in urban areas now face higher $\beta$s, and these effects are strengthening over time. We use these predicted DF values in our core branch closure models below.

\section{Branch Usage, Technology Adoption, and the Role of Demographics}

As discussed earlier, customer demographics shape branch restructuring through two main channels: by influencing banks’ pricing power and by affecting the value customers place on physical proximity to a branch, particularly through differences in technology adoption. The COVID-19 pandemic provided a shock to this valuation, as many individuals became more comfortable using technology to substitute for in-person interactions. To capture these changes, we begin by examining shifts in branch foot traffic between 2019 and 2021 using Advan cell phone data, where \textit{Drop in Visits} is defined as (Traffic in 2019 – Traffic in 2021) / Traffic in 2019. We interpret larger declines in foot traffic as indicative of greater reliance on digital banking relative to in-branch services. We also use the Advan data to measure the median distance traveled by visitors to reach a branch in 2019. These metrics, based on location and time-stamped mobile device data, serve as inputs into the following model:

\begin{equation}
    \text{Usage}_{b,j} = \sum \gamma^k D_{j}^k +\text{Other controls}+\varepsilon_{b,j}
\end{equation}

where \textit{b} indexes banks, \textit{j} indexes branches, \textit{k} indexes the demographic variables observed in each branch \textit{j}’s zip code.\footnote{As noted, we do not have cell phone data during the Pandemic year of 2020. Even if we did, these data would be highly unrepresentative of normal behavior due to the effect of lockdowns and general fear of COVID contagion.} The control variables include a bank fixed effect, state fixed effect, the county-level population density, and the log total deposits held in branch \textit{j}.

Table \ref{table3} reports the estimates of equation (3) with all four demographic factors (Panel B), and also the more parsimonious version in which we collapse education and stock-market participation into a single sophisticated indicator (Panel A), as in Table \ref{tab:deposit_beta_reg}. Regressions are split based on bank size.

These results show a strong effect of local demographics on branch usage. For both large and small banks, usage declines sharply around the pandemic: the mean decline in foot-traffic between 2021 and 2019 equals 31\% for large-bank branches and 15\% for small.  The regressions show that this decline is higher in areas with more sophisticated clientele.\footnote{\cite{sakong2025drives} find higher demand for branches among high-income populations, based on foot-traffic prior to the Pandemic.  Our results suggest, however, that demand for branches fell most sharply during the pandemic among financially sophisticated populations.} The number of visits, for example, drop 6 percentage points more at large-bank branches located in these areas (and 10 percentage points more for small banks). Moreover, visitors from sophisticated areas are on average traveling from further away when they do visit a branch. In other words, customers in financially sophisticated locations value physical branches less than customers in other areas. Age also correlates strongly with usage, with foot-traffic falling most for areas with many young people (the omitted group in the regression). As the model shows, the age effects on the drop in visits is monotonic, while the effect on distance is non-monotonic across the distribution.

Next, we turn to studying how the two bank branch usage measures evolve dynamically around the Pandemic. To examine these changes, we estimate a regression of the following form:

\begin{align}
    \text{Usage Metric}_{i,m} &= \sum_m \beta_m \times \text{Sophisticated zip} \times (month=m)  \nonumber \\
    &+\text{Other controls}+\text{Fixed effects}+\varepsilon_{i,m}
\end{align}
where \textit{i} is the branch and \textit{m }is the month. The regression includes bank, month, and zip code fixed effects. The coefficient  captures the differential effect of financial sophistication on branch usage across months, relative to the omitted base month (January 2019).

Figure \ref{fig:dynamic_did_plots} reports the  estimates with the corresponding 90\% confidence intervals, presented separately for large and small banks. Panel A uses \textit{Drop in Visits} as the dependent variable and Panel B uses the natural logarithm of the median travel distance to the branch. The clear pattern in Figure \ref{fig:dynamic_did_plots} is that the number of visitors to branches drops more in financially sophisticated areas (i.e., the betas plot below zero), and the effect of financial sophistication grows around the pandemic.  These effects are most pronounced for the small banks (consistent with the regressions in Table 3). 

In contrast to visits, the average travel distance to branches showed no significant change, indicating that while fewer sophisticated people visited branches in these areas, the geographical reach of branch visitors remained stable.  As such, we focus on the cross-branch variation in travel distance, rather than its change around the Pandemic, in our models of branch restructuring below.

These findings align with broader societal shifts induced by the pandemic. Lockdown rules forced many people to rearrange their work schedules, leading to lasting behavioral changes, including a significantly higher prevalence of working from home. As \citet{barrero2023long} document, only about 5\% of Americans worked from home before the pandemic, but this figure surged to 60\% during the lockdowns and has since stabilized at around 30\%. Consistent the results in Table 3, the increase in remote work was far greater for individuals with a college degree or higher, reflecting their greater ability to adapt to remote work arrangements. The large difference in usage patterns based on demographics follows because higher income and more educated people were more likely to be able to work at home, compared to other people. 

As noted, these cross-sectional and time series patterns suggest that more financially sophisticated people value proximity to a nearby bank branch less than other people.  When the Pandemic hit, they reduced branch visits more, and consistently they travel further when they do visit a branch. Such effects, we argue, occur because these customers are accessing banking services increasingly with technology – internet and mobile banking – and more so than less sophisticated customers. The results of Tables \ref{tab:deposit_beta_reg} and \ref{table3} together imply that a branch’s natural depositor clientele – the people living near the branch - drives both the pricing power of those branches (Table \ref{tab:deposit_beta_reg}) as well as the usage of those branches (Table \ref{table3}). In our last set of models, we test whether these two usage factors help explain branch opening and closing decisions.

 

\section{Branch Closings and Openings}
Our analysis so far has shown that local demographics drive variation in deposit franchise value (DF) and branch usage, with more financially sophisticated areas exhibiting lower DF and reduced reliance on physical branches. We now turn to examining how these factors influence banks’ decisions about which branches to close and where to open new ones.

We begin by documenting the raw relationship between DF and branch closures across monetary tightening cycles, as shown in Figure \ref{fig:pct_closed_df_bin}. The figure presents a bin-scatter plot of the annual percentage change in branches during each cycle, with bins defined by the branch-level predicted DF (Panel A) or the bank-level actual DF (Panel B). The pattern is clear—particularly during the second and third cycles: branches with lower franchise value are more likely to be closed. This effect appears in both the branch-based and bank-based panels, with closure rates ranging from about 3 to 5 percent per year for low-DF branches, compared to less than 2 percent for those with high DF.\footnote{A similar graphical analysis is not appropriate for branch openings, as banks face a wide range of potential locations for new branches. In contrast, branch closures are limited to zip codes where the bank already operates.}


\subsection{Empirical Design}

To examine these patterns more formally, we estimate three types of models to analyze the drivers of branch restructuring. The first focuses on the deposit franchise value.  Since the data structure differs between openings and closings, we describe each in turn:

\subsubsection{Closures}

We use the following linear probability model where the dependent variable (\textit{Closure}) indicates if a particular branch \textit{j} owned by bank \textit{b} was closed in year \textit{t}:

\begin{equation}
    \text{Closure}_{b,j,t} = \gamma DF_{b,j,t} + \text{Other controls} + \text{Fixed effects} + \varepsilon_{b,j,t} \tag{5a}
\end{equation}

In (5a), \textit{DF\textsubscript{b,j,t}} equals the predicted value of the deposit franchise for branch \textit{j} owned by bank \textit{b} at time \textit{t}, as described above. Our baseline estimates pool the branch-year data across the full 2001-2023 period. We report models with $state\times year$ and $bank \times year$  fixed effects, and we report models with  $county \times year$ fixed effects as well. By including $bank \times year $ fixed effects (as well as  $county \times year$ effects in some specifications), we fully absorb both the general trends in banking and technology, as well as heterogeneity in the supply of technology across banks. For instance, \citet{haendler2022keeping} shows that large banks adopted and updated mobile apps earlier and more frequently than smaller banks. This approach absorbs supply-side differences in the quality and quantity of online and mobile banking services, as these are common across all customers of a given bank, regardless of branch location. In addition, the $county \times year$ absorbs variation in local access to technology, such as differences in investment in the quality of the cell phone network. Thus, identification comes solely from variation in the impact of local demographics (i.e., demand-side factors) on branch closures.

At this stage, we exclude measures of branch usage, as these are only available for the final two years of our sample. We also estimate the model separately for large and small banks, reflecting differences in customer demographics (\citet{d2023deposit}) and the potential variation in marginal effects due to differences in the quantity and quality of services offered.

For control variables, we include the lagged log level of deposits, the three-year past growth rate of (i) deposits, (ii) mortgage applications and (iii) small business loan originations to capture local supply and demand conditions for deposits and loans. County-level growth in the number of establishments and payroll serves capture local economic growth. We also control for county-level population density. Two M\&A-related indicators are added: the first equals one if the bank has owned at least one branch in the zip code for the past three years, and the second equals one if the current branch was acquired by the bank in the past three years and bank has owned at least one other branch in the same zip code prior to the acquisition. Finally, we include an indicator for branches in low- and moderate-income (LMI) areas, as defined by the Community Reinvestment Act, where banks face regulatory pressure to lend locally.

\subsubsection{Openings}

To understand branch openings, we construct a $bank \times zipcode \times year$ dataset which captures candidate zip codes where each bank might choose to open a new branch.  For each bank-year, we include all zip codes in the CBSAs where the bank owned at least one branch in the prior year, and we add all zip codes in CBSAs in which the bank opens a new branch in the current year.  Note that the set of candidates zip-codes differs across banks and time.  We also drop all zip codes in which no bank ever owns a branch during our sample.  The dependent variable is set to one if the bank opens a new branch in the candidate zip code and zero otherwise.\footnote{The latter zip codes - those in CBSAs where a bank opens a branch for the first time - are potentially endogenous because they are conditional on the bank entering the area.  In Internet Appendix, however, we verify that our core results are similar if we exclude these observations.}  

With this sample, we estimate linear probability models parallel to 5(a), although we replace the lagged log level of deposits in an incumbent branch with the log of (1+deposits) based on all bank branches located in the given zip code during the prior year (i.e., branches owned by competing banks).  Since potential entrants have no deposits from the prior period, we interpret this variable as a measure of the potential (or maximum) level of deposits a new bank could raise.  As in 5(a), we estimate the base model with $bank \times year$ fixed effects, as well as $state \times year$ or $county \times year$.


\subsubsection{Reduced Forms}

In our second set of models, we extend the analysis by using a reduced-form version of (5a) and its analog for openings, replacing the predicted deposit franchise (DF) metric with the underlying demographic and market concentration variables that were used to construct it. The regression specification is as follows:

\begin{equation}
    \text{Closure}_{b,j,t} = \sum_k \gamma_t^k D_{b,j,t}^k + \eta HHI_{b,j,t} + \text{Other controls} + \text{Fixed effects} + \varepsilon_{b,j,t} \tag{5b}
\end{equation}

Here, $D_{b,j,t}^k$ represents the k-th demographic variable at branch \textit{j} owned by bank \textit{b} and time \textit{t}. Fixed effects again include $bank \times year$ and  $state \times year$ or $county \times year$.  We report similar regressions for the choice to open new branches. 

This approach allows us to test which local variables are most tightly linked to branch restructuring decisions. In these models, we report an additional specification combining education and stock-market participation into a single financial sophistication measure. 

\subsubsection{Models with Usage}

Our third set of models incorporates the two cell phone-based branch usage measures as right-hand side variables. These specifications allow us to assess the relative importance of pricing power (deposit franchise value) versus customer convenience or the amenity value of proximity.  

For the closure analysis, we can directly observe the two usage measures, as we do in Table \ref{table3}.  Branch-level usage patterns, however, are potentially endogenous and may respond to depositor expectations that a given branch will close.\footnote{In fact, by regulation banks are required to inform depositors of an impending closure by mail with at least 90 days notice.  See https://www.fdic.gov/consumer-resource-center/2024-07/your-bank-branch-relocating-or-closing.}  For example, if depositors are informed that their branch will close, they may increase their in-person visitations to the branch. Hence, we adopt a 'leave out' strategy to build the two usage measures, as follows: for each branch, we compute the average \textit{Drop in Visits} and the average \textit{Log(Distance Km)} for all other branches located in the same zip code.  As such, we drop all branches which are located in zip codes without competing branches.

For openings, there is no latent endogeneity problem because banks can only form expectations of usage based on patterns observed for existing branches of other banks.  So, we build the usage measures based on the zip-code level averages of \textit{Drop in Visits} and \textit{Log(Distance Km)} for all branches located in each bank's candidate zip codes.

As noted, although the Advan data start in 2019, we estimate these models only in 2022 and 2023. The industry (NAICS) codes for the closed branches were changed in 2021. As a result, branches closed in 2020 and 2021 have different NAICS in the Advan data, which doesn't provide a historical time series of these codes by location. When creating the dataset, we initially filtered locations with NAICS code 522 to indicate a banking office, so by necessity we filtered out these locations due to the change in the NAICS to a different code.\footnote{It is computationally prohibitive to standardize all the US addresses and then match only by the address without first filtering by the NAICS code.}

Across all three sets of models, we build standard errors by clustering at the bank level.

\subsection{Results}

We first present summary statistics of the branch-year panel sample used for the closure regressions, focusing on the years 2012 and 2019. Tables \ref{tab:branch_desc_stats} and \ref{tab:branch_opening_desc_stats} provide summary statistics of key characteristics for the branch closure sample and branch opening sample, respectively. For both the dependent variables as well as the DF and the level of deposits, we report the sample standard deviation (SD) as well as the "SD (within)," which removes variation explained by the bank and county fixed effects; we use that latter metric to assess economic significance of the regression results below.

The tables split the summary statistics by bank size to highlight differences between small and large banks. While many mean characteristics are similar across the two groups, there are notable distinctions. Large banks, for example, hold significantly more deposits per branch, with the typical branch holding more than twice as many deposits as those of small banks. Small banks exhibit faster lending growth to small businesses. Geographically, small banks are more prevalent in rural areas, as indicated by their branches being in regions with much lower population density. 

\subsubsection{Baseline Results}

Table \ref{tab:baseline_closure_regression} reports our estimates of Equation (5a). We report the pooled sample (columns 1 \& 2), and then the split-sample results by bank size (over versus under \$100 billion in assets) in columns 3-6. For each set of specifications, we report models with $state \times year$ fixed effects and separately with $county \times year$ fixed effects. For all banks, in columns (1) and (2), higher DF leads to lower probability of a branch being closed. The magnitude is substantial across all banks, but also larger for the large banks. A one-standard-deviation increase in the DF ($\approx$ 0.004 for large banks and 0.003 for small banks), for example, leads to a decline in annual branch closure probability of about 0.4 percentage points (column 4). This effect is large, equal to about 10\% of the unconditional mean closure rate (=4\% per year for the large banks) during our sample. Beyond the DF, which captures the per-dollar value of the current stock of deposits, higher levels of deposits in the branch from the preceding year also has strong power to predict branch closures for large banks. Hence the value of both ``deposits in place” seems to drive closure decisions.

In contrast, neither the growth in market-level deposits nor loans (mortgage and small business loans) has much ability to explain branch closures. We do find a marginally significant effect of small business lending growth on smaller banks’ branch closures, but none for larger ones. This contrast suggests that the core purpose of bank branches (especially for large banks) is to support the deposit franchise, where banks remain dominant. On the other hand, banks have become increasingly \textit{less} important as suppliers of local credit - mortgages and small business loans. Moreover, deposits constitute about 85\% of all bank financing, while local lending comprises a small percentage of total bank investments (again, especially for larger banks).\footnote{In the Internet Appendix, we report enhanced models with two additional measures of local lending: one based on the log of small-business loan originations in the bank-county-year, and the other based on the log of mortgage originations by bank-county-year.  As with local lending growth, these bank-specific measures have weak statistical relationships to bank closures (and openings), with inconsistent sign patterns across samples and specifications.} 

The models also suggest that banks are much more likely to close branches acquired recently if they already had branch presence in the zip code, and less likely to close `legacy’ branches, meaning those which have not been acquired over the past three years. We find no evidence that banks close branches in localities defined as LMIs under the Community Banking Act; if anything, large banks are \textit{less} apt to close branches in these areas.\footnote{Banks are required to give regulators and local customers notice before closing a branch under Section 42 of the Federal Deposit Insurance Act. Hence, large banks may be concerned that closing branches in LMI areas could lower their CRA rating, which in turn could impinge on future acquisitions.}

Table \ref{tab:baseline_open_regression} reports the baseline estimates for branch openings. The DF plays a central role in predicting entry. Higher DF—reflecting lower interest-rate sensitivity among local depositors—is associated with reduced entry. A one-standard-deviation increase in DF (approximately 0.004 for large banks and 0.0045 for small banks, after removing variation explained by fixed effects) lowers the probability of opening a branch by 0.12\% (=0.004 x -0.3, from column (4)), or about 30\% of the unconditional opening rate of 0.42\%.\footnote{For large banks, the unconditional branch opening rate declined from approximately 0.8\% prior to the GFC to about 0.2\% afterward. The corresponding rates for small banks fell from 0.3\% to 0.09\%.} Local deposit levels and economic growth also enter strongly, with the expected sign: more deposits and faster establishment growth increase the likelihood of entry, opposite to what we observe for closures.  The economic magnitude of the level of deposits is similar to that of DF.  Hence, banks enter new markets which are 'rich' in deposits, but only when price sensitivity is high (DF is low).\footnote{Consistent with our results on entry \citet{begenau2023uniform} show that deposits grow much faster at new branches compared to older ones.}  

Taken together, the opening and closure results show that low values of DF predict both a higher likelihood of exit by incumbent banks and a greater probability of entry by new banks.\footnote{We have also estimated closure models which control for an indicator equal to one for zip codes with new branches opened within the past three years.  These markets exhibit higher closure rates, but adding this variable has little impact on our core results.  Similarly, zip codes with recent closures have a higher probability of entry (openings), but again adding this has little effect on our core results.  See the Internet Appendix for these results.}

% Table \ref{tab:baseline_open_regression} reports the analogous estimates for openings.  The level of deposits at the zip-code level as well as local economic growth both enter strongly and with opposite sign compared to closures: more deposits and higher establishment growth predict greater entry.  For one standard deviation increase in within fixed-effects standard deviation ($\approx$ 0.004 for large banks and 0.0045 for small banks), the probability of opening a branch drops by 0.12\%, which is approximately 30\% of unconditional probability of opening a branch.  But the effect of the DF - which varies based on local-resident price sensitivity - explains both entry and exit similarly.  Low values of DF predicts that incumbent banks become more likely to exit, but also increases the probability of entry by new banks.  Said differently, areas where customers have high interest sensitivity (and thus low DF) are characterized by high levels of both entry and exit.  These areas are more dynamic; areas with low rate-sensitivity depositors, in contrast, experience low levels of both entry and exit.  Beyond deposits, lending growth has at best a weak relationship to entry with inconsistent sign patterns, similar to its weak effect on closures.

%The economic significance of DF is also sizable for openings, as it is for closures.  After removing the fixed effects, the DF in the openings data has residual standard deviation of 0.004. So, a one-sigma decrease in DF raises the probability of opening a branch in a given zip code by 0.0012 for large banks (=0.004 x -0.3, from column (4)).  The unconditional probability of opening in a given zip code equals about 0.01 before the GFC and about 0.005 after (conditional on a bank opening at least one branch), so the economic impact is on the order of 10 to 20 percent of the unconditional probability.

\subsubsection{Reduced Form results}

Tables \ref{tab:reduced_form_closure} and \ref{tab:reduced_form_opening} reports parallel models of branch closures and openings estimated as reduced forms (as in Equation (5b)). Panel A reports the full sample, and Panel B split by size. We find that branch openings and closings, for both large and small banks, are more likely in areas dominated by financially sophisticated people. Each of these effects is consistent with our interpretation that the value of the DF plays a key role in driving branch restructuring decisions. Comparing these results with those from Table 2, we see that high levels of financial sophistication lead to lower DF (higher deposit beta), and higher rates of both branch openings and closings.  Younger populations are also strongly associated with lower DF (higher beta) and more entry (Table 8).  For closures, however, the effects of age are less consistent across banks of different sizes.

\subsubsection{Estimations over time}

Tables \ref{tab:closures_by_regime} and \ref{tab:openings_by_regime} report estimates of Equation (5a) during four regimes: the period prior to the GFC (2001-2007); the years affected by the GFC (2008-2011); the post-GFC / pre-Pandemic years (2012-2019); and the post-Pandemic years (2020-2023). Panel A reports the model for large banks, and Panel B small ones (with $county \times year$ or $state \times year$ effects). These results suggest that branches with low deposit franchise value are consistently most likely to be closed, with the marginal effect of the DF increasing over time. The effects are largest after the Pandemic. For large banks, a standard-deviation decline in the deposit franchise (=0.006) comes with an increase in closure probability of more than one percentage point (column (8)), equal to about one-quarter of the unconditional closure rate. As in the baseline model, openings are also consistently higher in areas with low DF, both across the four regimes and also for large and small banks.\footnote{In an earlier version of this paper we report consistent effects over time using year-by-year regressions rather than pooled ones across the four regimes.}  

To summarize, the effects of both the level and pricing of deposits across all models, across all types of banks, and across time tell the same story.  First, incumbent banks \textit{close} branches where the per-dollar economic rents are low and where the total amount of deposits in their branch is also low.  Second, banks \textit{open} new branches in areas with high levels of deposits held by incumbents (because they are entering to raise deposits), but only where local depositors are price sensitive (because they can't effectively draw deposits away from incumbents when customare are rate insensitive).

\subsubsection{Adding Usage}

Tables \ref{tab:closure_with_usage_controls} and \ref{tab:openings_with_usage_controls} report our estimates for closures (Table 11) and openings (Table 12) after incorporating usage.  These models allow us to compare the relative importance of branch usage (from declines in foot traffic from 2019 to 2021) vs. banks’ deposit pricing power (DF). The regressions include only the last two years of our sample (2022 and 2023) due to data constraints in the Advan cell phone data, as noted above. Both the results for openings and closings continue to show that the DF remains the dominant driver of branch restructuring. Adding the usage metrics slightly attenuates the effect of the DF on closure, from –0.89 to –0.63 for large banks (Table 12, Panel B, columns (3) and (4)).  Consistent with expectations, areas which experience large drops in branch usage (\textit{Drop in visits}) around the pandemic experience more closures.  This effect represents the 'teachable moment' of the Pandemic, in which many people learned how to substitute on-line technology for in-person interactions.  As such, it represents a large shock to the value of close proximity to a bank branch.  In addition, we find consistent evidence that banks are more likely to close branches located in areas where customers travel from greater distances.

The effects of usage on openings are less clear, however.  Like closures, DF continues to correlate negatively with branch openings.  However, we find some evidence - mainly from small banks - that openings also are higher in areas with large declines in foot traffic.  This may reflect the fact the these areas are ones where technology adoption was greatest, which lowers both the DF (by raising price sensitivity) and also lowers the amenity value of close geographic proximity.  We find essentially no explanatory power of travel distance in any of the openings specifications.


\section{Conclusion}

Banks opened new branches at a higher rate than they closed them until the GFC, when this pattern reversed sharply. We show that variation in economic profits generated from bank deposits helps explain branch restructuring patterns, as banks are most likely to close branches in areas with  low franchise value due to high interest sensitivity of local residents. Technologies which make physical proximity less important and which lower the cost of moving funds to substitute investments, we argue, drove the regime shift in branching starting around 2010. In contrast to other research, our empirical design exploits different rates of technology adoption by bank customers, which varies depending on characteristics associated with financial sophistication. The results suggest that the 2020 Pandemic had a large effect on technology adoption, leading to a sharp decline in foot-traffic at branches and an overall rate of branch closure roughly double what had come before.

Our results point to the importance of analyzing industry structural change using gross measures of openings and closings.  Incumbent bank incentives differ sharply from those of potential de novo entrants.  As we show, entry is higher consistently across bank types and over time in areas where local residents have high price sensitivity – exactly the areas where incumbents are most likely to exit.  Conversely, price insensitivity of depositors creates an endogenous entry barrier for greenfield investment, and helps explain why most bank extensions into new markets happen via M\&A facilitated by deregulation, which allows an entering bank to buy the existing customer base.

Understanding the drivers of branch closures matters because branch-based frictions have traditionally mediated flows of capital across markets and have affected local-market competition in both deposit and credit markets. Such frictions reduce financial market efficiency and integration. Lowering these frictions through technology furthers a process which began in the 1980s with deregulation of restrictions on branching and interstate banking. As such, continued bank restructuring will likely improve the functioning of local financial markets further.



