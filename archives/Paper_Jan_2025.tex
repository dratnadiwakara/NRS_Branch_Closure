\documentclass[12pt]{article}
\usepackage{fullpage}
\usepackage{amsmath}
\usepackage{amsthm}
\usepackage{longtable}
\usepackage{graphicx}
\usepackage{graphics}
\usepackage{epsfig}
\usepackage[longnamesfirst]{natbib}
\usepackage{lscape}
\usepackage{amsfonts}
\usepackage{amssymb}
\usepackage{float}
\usepackage{tabls}
\usepackage{setspace}
\usepackage{verbatim}
\usepackage{url}
\usepackage[multiple]{footmisc}
\usepackage[margin=1in]{geometry}
\usepackage{color,soul}
\usepackage{booktabs}
\usepackage{multirow}
\usepackage{ragged2e}
\usepackage{geometry}
\usepackage{pdflscape}
\usepackage{dcolumn}
\usepackage{enumerate}
\usepackage{array}
\usepackage{enumitem}
\usepackage{color}
\usepackage{hyperref}
\usepackage{epstopdf}
\usepackage[title]{appendix}
\usepackage{titlesec}
\usepackage[normalem]{ulem}
\usepackage[font=bf]{caption}
\usepackage{mathpazo}
\usepackage{hyperref}
\usepackage{tikz} 
\usepackage{tabularx,booktabs} 
\usepackage{booktabs}
\usepackage{subcaption}
\usepackage{graphicx}
\setcounter{MaxMatrixCols}{10}
\usepackage{siunitx}
%TCIDATA{OutputFilter=LATEX.DLL}
%TCIDATA{Version=5.50.0.2960}
%TCIDATA{Codepage=65001}
%TCIDATA{<META NAME="SaveForMode" CONTENT="1">}
%TCIDATA{BibliographyScheme=BibTeX}
%TCIDATA{LastRevised=Thursday, July 23, 2020 13:46:23}
%TCIDATA{<META NAME="GraphicsSave" CONTENT="32">}
%TCIDATA{Language=American English}
\usepackage[table]{xcolor}
%\definecolor{maroon}{cmyk}{0,0.87,0.68,0.32}
%\definecolor{mygray}{gray}{0.6}
\hypersetup{
colorlinks=true,
citecolor=blue,
linkcolor=red,
}
\titlelabel{\thetitle.\quad}



\newcommand{\sigtexfigure}{Standard errors are clustered at the bank level.}


\newcommand{\adddescription}[1]{
    \begin{minipage}{0.9\textwidth}
    \vspace{0.2cm}
     #1
    \end{minipage}\\
    \vspace{0.5cm}
}
\doublespacing
%\floatstyle{ruled}
\floatstyle{plaintop}
\restylefloat{table}
\restylefloat{figure}  
\newcommand{\floatintro}[1]{
\vspace*{0.1in}
{\small
#1
}
\vspace*{0.1in}
}
%\newcommand{\sym}{}
\sloppy

\begin{document}

\title{The Decline of Branch Banking\footnote{The results and views are those of the authors and do not reflect those of the Federal Reserve Bank of Richmond, or the Federal Reserve System.}}



\author{Rajesh P. Narayanan\thanks{Louisiana State University (rnarayan@lsu.edu)} \and Dimuthu Ratnadiwakara\thanks{The Federal Reserve Bank of Richmond; email: dimuthu.ratnadiwakara@gmail.com)} \and            Philip E. Strahan\thanks{  Boston College \& NBER (strahan@bc.edu)}}

\date{\small{Preliminary Draft; January 2025}}

\maketitle

\begin{abstract}
\begin{singlespace}
This paper examines why banks closed branches during 2015 to 2023, a period of significant restructuring in the banking industry.  Closures are higher in areas with more financially sophisticated residents, as better-educated and younger populations reduce the value of the deposit franchise by being more interest-rate sensitive and less reliant on branch proximity. Using a branch-level measure of deposit franchise value based on local demographics, we show that declines in the deposit franchise drive branch closures, especially for large banks. Lending has little explanatory power. The Pandemic accelerated closures by driving widespread adoption of digital banking.

\end{singlespace}
\end{abstract}
Keywords: deposit franchise, branch closure, digital banking


\thispagestyle{empty}

\clearpage


\section*{Introduction}

This paper examines the drivers of bank branch closure decisions during the significant industry restructuring period of 2015-2023. We show that measures of local financial sophistication strongly predict branch closures, as customers in these areas are more interest-rate sensitive and place less value on close proximity to their banks, both of which follow from greater facility using technology to access financial services. 

Bank branches grew in aggregate until 2010. This growth occurred at the same time that the banking system as a whole consolidated. Total branches and offices, however, peaked in 2010 and then began to fall. The decline accelerated around 2015 and further accelerated after the COVID-19 pandemic (Figure \ref{fig:no_of_banks_branches}). The recent patterns contrast sharply with the period between 1980 and the Global Financial Crisis. The banking-industry restructuring then occurred as large banks expanded into new markets by buying mostly smaller incumbent banks. Such change was fostered by deregulation of restrictions on branching and interstate banking (\citet{Schneider2025}). Figure \ref{fig:no_of_banks_branches} shows that this restructuring sharply reduced the number of banks (due to the M\&A), but not the number of bank branches, which continued to rise. The purpose of this first wave of M\&A was fundamentally about extending (large) banks’ physical reach (\citet{kroszner2014regulation}). Since 2010, however, the number of banks and the number of bank branches have been falling together. Our results suggest that the motivation for bank M\&A has shifted, with the first wave coming from the effects of deregulation and the second from the effects of information technology and changes to the payments system, which reduce the value of branches.

We show that bank branch closure decisions are largely driven by declines in the value of the deposit franchise—the ability of banks to retain deposits priced at interest rates \textit{below} the market rate. Our regressions provide no evidence that lending or local loan demand explains branch closures. Unlike deposits, neither local growth in mortgage originations nor local growth in small business loan originations has any significant explanatory power for branch closures. We argue that the primary driver of restructuring today is technology, which reduces both deposit market power and the non-pecuniary benefits of proximity to a branch. This contrasts with earlier periods, where deregulation was the main driver of restructuring. Thus, our paper contributes to explaining the recent decline in bank branch networks and the broader restructuring of the banking industry.

 
We construct an empirical design that gets identification from variation in deposit and lending conditions across each bank’s branch network. For each branch owned by a given bank, we first construct a measure of the value of that branch’s deposits based on the sensitivity of the interest expenses on deposits to market interest rates, where we exploit the interest rates cycle from 2016 to 2019 in the first half of our sample and the cycle from 2022 to 2024 in the second. The measure is based on \citet{dss2023} and simply captures the present value of rents generated per dollar of bank deposits (the “Deposit Franchise Value” or DF). Our twist on their approach is to construct a bank-level predicted rate sensitivity measure based on average local demographic variables and then assign this to each branch based on its geographical address. This strategy provides variation in branch-level deposit valuation \textit{within each bank}. With the branch-based DF measure, we can understand how a given bank decides which of its branches to close, as a function of the marginal profit each branch contributes based on the characteristics of the local depositor base. Said differently, we estimate all our branch closure models with bank (and bank-time) fixed effects.

We use our indirect approach because direct information on deposit pricing at the branch level is not available systematically. Some recent research has used deposit quote data from RateWatch, but these data are only meaningful for a small subset of branches (“rate setting branches”). Using RateWatch even for the rate-setters, as argued by \citet{begenau2023uniform}, is problematic as a measure of local pricing power because many large banks extend the same (quoted) rate over a large region. Moreover, branches assigned as `rate-setters’ are almost never closed. Our strategy instead starts with a model that explains the \textit{overall} bank-level deposit-cost sensitivity (the change in interest expenses on deposits / total deposits during the two rate cycles) as a function of local demographic variables (e.g., age, income, education, and rates of stock market participation) assigned to each bank by averaging across the local demographics near the bank’s branch footprint. This preliminary model provides coefficients, which we then use to build a predicted \textit{branch-level} measure of deposit price sensitivity (“Deposit $\beta$ ”) using the demographic variables associated with people living in the zip code of each branch. The predicted DF (which depends on $\beta$ ) varies across branches within each bank; two branches located very close to each other (e.g., within the same zip code) would be assigned the same franchise value, but a bank owning branches in different zip codes would have different franchise values, with the degree of difference depending on the variation in the demographic factors across the two areas. As we show, this construct has very strong power to predict bank closure decisions within the bank (and within bank-year).

Understanding the drivers of bank branch closure decisions helps provide a deeper understanding of why the industry has and will likely continue to be reshaped. Phone and internet banking have become widespread \citep{haendler2022keeping,jiang2022bank,koont2023digital}, and these technologies both reduce frictions associated with physical distance between depositors and branches, and also reduce local pricing power over depositors. Our results show that banks close branches more in markets with young and well-educated nearby residents and residents more exposed to the stock market because the deposit franchise value is worth less in such localities. As we show, because such people are more likely to adopt information technologies earlier and more aggressively, they visit bank branches less and travel from further distance when they do visit (compared to less financially sophisticated people).

We also find that the marginal impact of the deposit franchise on closures increases sharply in 2020, during the COVID-19 pandemic. The pandemic was a `teachable moment’ for many people, who learned that technology can effectively substitute for physical proximity. Work at home became prevalent, and this change forced many to rearrange their lives in ways that emphasized online rather than in-person interactions. Richer, more educated, and younger people made these adjustments more than others, and this shows up in the impact of the demographic drivers of the DF. In fact, we show a decline in the average value of deposits in the later interest rate cycle (2022-2024) compared to the earlier one (2016-2019), with a larger decline at big banks and in areas with high population densities.

Variation in the value of the DF also matters more for large banks’ decisions to close branches than it does for small ones. The marginal effect is more than double for large banks. The starkest difference appears in 2020. The pandemic year came with the largest overall decline in branches, with almost 8\% of large-bank branches being closed (Figure \ref{fig:pct_branches_closed}). For large banks, in 2020 a one-standard-deviation decrease in the DF raised the probability of a branch closure by about 2.7 percentage points (or about one-third of the 2020 mean closure rate), compared to about 1.0 percentage points for smaller banks. We think this difference reflects the sorting of customers within localities across banks based on size, whereby the more sophisticated customers gravitate toward large banks. Consistent with this idea, \citet{narayanan2024depositor} use cell phone data to link customers' characteristics to the branches that they visit. This paper shows clearly that large banks attract customers living in areas with higher levels of income, education, and other measures of financial sophistication. Similarly, \citet{d2023deposit} argue that large banks cater to different clientele who value more complex services that go beyond deposits, compared to smaller banks.

In contrast to the DF, we find little to no evidence that lending variables can explain branch closures. This is surprising. Much of the prior banking literature had demonstrated the importance of physical distance between bankers and borrowers (e.g., \citet{petersen2002does,berger2005does}). But technology obliterates the importance of distance. Until recently, most business sales came largely from cash transactions, which necessitates a close physical proximity to a bank branch, if nothing else as a means to safeguard cash receipts. Today, businesses accept an increasing fraction of their sales from electronic payments (as opposed to cash), removing the need for local branches for mere security reasons. Beyond that, the information environment also has changed sharply, again because payment flows are now dominated by electronic means.\footnote{Penetration of phone-based payments technologies has been faster in many parts of the world than in the US. As a result, for example, the number of branches per capita has fallen twice as fast in the EU as in the US. In the Netherlands, to take an extreme case, the number of bank branches has fallen by 85\% (https://fred.stlouisfed.org/series/DDAI02NLA643NWDB).
}  As such, physical proximity no longer matters much for information production.\footnote{\citet{buchak2018fintech} focus on the increasing market share of fintech lenders in the mortgage space and \citet{gopal2022rise}, who report a high and growing share of lending to small businesses by non-banks. For a review of the growing role of Fintech lenders generally, see \citet{berg2022fintech}} For these reasons, we argue that the demand for lending does not help explain branch closures because brank location no longer matters for effective credit provision by banks (or other lenders, such as Fintechs).\footnote{Note that we are not claiming that bank relationships no longer matter; nor are we arguing that bank-borrower lending relationships are no longer sticky, as has been documented across many studies. We are instead arguing that physical space between banks (or bankers) and borrowers matters much less than in the past.
} 

In the last part of our analysis, we use cell phone data to track the usage of bank branches, focusing on the number of visitors and the distance those visitors travel to visit a branch. These metrics help predict bank branch closures, but adding them to our model attenuates the impact of the value of the deposit franchise itself on slightly (and it retains strong explanatory power). This occurs, we show, because both branch usage and the DF are strongly correlated, as both are driven by local demographics. These findings close the loop on our core argument: areas with sophisticated residents substitute away from brick-and-mortar banking and into technology. This behavior reduces the value of their deposits to banks because customer facility with technology increases their interest-rate sensitivity, and because technology reduces the amenity value to customers of close proximity to a bank branch. People who rely mainly on mobile apps and the internet to access their bank do not value a nearby branch. As a result, demographic variables drive all three outcomes: branch usage, the DF (which depends mainly on interest-rate sensitivity), and branch closures.

We contribute to a nascent literature studying drivers of branch closures. \citet{keil2024demise} study the de-branching regime as we do, but that paper emphasizes variation across banks in technology adoption. Similarly, both \citet{haendler2022keeping} and \citet{jiang2022bank} show that banks offering customers online or phone-based access are more likely to close branches. \citet{koont2023digital} argues that bank investment in digital technologies leads to branchless competition. Our approach takes banks’ investments in technology as given (by absorbing bank-time effects) and instead focuses on how branch-level variation in the customer base explains closures. As such, our work is complementary to theirs. The core difference is that our effects depend on variation in customer adoption of technologies – that is, customer demand for non-branch access to their funds - which lowers the value of bank branches. We shut down variation in the supply of technology by controlling for bank-year and county-year fixed effects. Evidence from other jurisdictions supports the view that tech-savvy customers lower the value of a physical branch location.  \citet{yuan2023your} show that younger customers (Gen Z) who switch to similar banking products offered by Fintech firms leads to a rise in both the number and share of branch closures in China. Similarly, \citet{zimin2022profitability} show that higher levels of financial digital literacy are related to de-branching in Russia.

Our paper complements \citet{kumarjames2025lowinterest}.  While our paper focuses on understanding which, among a given bank’s existing branch network, that bank chooses to close, theirs focus on the role of the deposit franchise in explaining closure rates across banks.  Like \citet{sarto2023secular}, they emphasize that banks lose value in the low interest rate environment after the GFC because deposit spreads over market rates compress.  Our empirical design thus complements theirs, as we fully absorb these effects and focus instead on differential effects of customer variation on branch closures.

In contrast to our results, where we find little evidence that local lending affects branch closures, \citet{cespedes2024branching}, who study an earlier period (2011-2017), find that shadow bank entry affects bank closures by lowering local residual loan demand, especially at banks with a high cost of violating the Community Reinvestment Act (CRA). Their evidence, along with ours, suggests that the impact of lending may have changed in recent years, as information technology has become increasingly important. 

Our paper also contributes to the large literature emphasizing how the location and scope of bank branching has mediated the supply and flow of capital within and across local markets. Until recently the literature has spoken with one voice: the ownership and location of branches introduces substantial frictions in banking services, both on the deposit and lending sides. Areas with highly concentrated ownership of branches experienced less competition in both deposit and lending markets. Closure of branches reduced local small business lending (\citet{nguyen2019credit}). Moreover, the flow of capital across markets was affected by connections between local areas from branch ownership networks \citep{gilje2016exporting,cortes2017tracing}. The most powerful evidence of the importance of bank branching emerged from studies of deregulation of restrictions on the ownership of bank branches both within and across state lines \citep{jayaratne1996finance,rice2010does}. Following such deregulation, credit and deposit market competition improved, capital mobility increased, and the supply of deposit and credit services increased. The decline in branching in the past decade, our results suggest, has occurred because these frictions are being eroded by information technology, which makes distance and physical proximity between banks and their customers increasingly irrelevant. As such, the competition-enhancing benefits which followed declines in branch-based frictions from deregulation ought to continue from the advent and penetration of information technology.


\section{A Framework to Understand Branch Closures}
Innovations in both payment systems (e.g., PayPal, Venmo, Zelle, etc.), along with easy access to deposits via the internet and smartphones have increased depositor sensitivity to market rates (thereby lowering bank pricing power) and also reduced the value to them of close proximity to a bank branch. We take these technological changes in the basic financial infrastructure as given, but exploit the fact that customer adoption of these technologies exhibits substantial heterogeneity. Younger and higher-income households, for example, interact with their banks via technology at higher rates and earlier than older and lower-income households (\citet{FDIC2023}). Hence, differences in the local demographics faced by different branches will drive variation in the value of those branches. We can represent the relationships we have in mind schematically as follows:
\begin{itemize}
    \item Local Demographics $\rightarrow$ Changes in the branch-level Deposit Franchise Value
    \item Local Demographics $\rightarrow$ Changes in Usage of Branches
    \item  Low DF and Low Branch Usage $\rightarrow$  Branch Closures
\end{itemize}

Our empirical section below estimates these relationships in three steps: 1) we estimate the relationship between local demographics and the DF at the bank level, and then use that model to compute DF at the branch level; 2) we estimate how demographics affect branch usage based on cell phone data; 3) we estimate how DF and branch usage affect branch closure decisions. As argued by \citet{egan2022cross}, most of the value created by banks stems from payments-related services, which allow banks to raise and retain deposits at interest rates well below market interest rates (\citet{lu2024making}). \citet{drechsler2017deposits} show how this pricing power over deposits leads to a novel channel of monetary policy transmission driven by the cyclicality in bank deposit pricing. As such, we focus most of our attention on changes in the value to banks of paying below-market interest rates on deposits. In our last set of tests, we introduce branch usage metrics to the closure model, which are only available during the most recent portion of our sample. 


\section{Data Sources}
\subsection{Summary of Deposits}
We use the FDIC’s annual \textit{Summary of Deposits} (SOD), which allows us to measure the amount of deposits and location of each bank’s branch network in June of each year, and also to observe branch closures. We estimate our closure model during the years from 2015 to 2023. We start the analysis in 2015, rather than earlier, partly because the decline in branches starts accelerating in that year. Moreover, as we describe below, we need to construct branch-level measures of the value of the deposit franchise, which can only be observed during periods when interest rates change. This makes the years of the zero lower bound useless for our purposes; without variation in market rates, we cannot measure how deposit prices vary with changes in rates.

We define a branch as `closed’ in year \textit{t} if it appears in June of year \textit{t} in the SOD dataset but not in the June of year \textit{t+1}. This definition can be established with certainty not only because the SOD contains a branch-based ID variable, but also because it contains detailed data on each branch’s physical location (e.g., latitude and longitude, as well as state, city and street address). Figure \ref{fig:pct_branches_closed} reports the number and fraction of branches closed in each of the years we study, which exceeds 2\% in every year during our sample. Large banks – defined as those with more than \$100 billion in assets – closed branches at consistently higher rates than smaller banks. Moreover, branch closures spike during the year of and following the COVID-19 Pandemic, with large banks closing nearly 8\% of their branches in 2020.\footnote{Remember that 2020 in our definition captures all branches closed between June 2020 and June 2021.}


\subsection{Bank Call Reports}

The Federal Financial Institutions Examination Council (FFIEC) requires US banks to file information on their financial health and performance at the end of each quarter and these are made publicly available. These ``Call Reports'' provide a breakdown of balance sheets and income statements. For our purposes, we obtain bank yearly information on assets, deposits, and interest expense on deposits (since we can only measure branch closures on an annual basis).

\subsection{Demographic data}

To capture local demographics factors, we use the American Community Survey (ACS) 5-Year Data, which has economic and socioeconomic information across various geographical levels in the United States. We use the census tract level information on income, education, and age, and we also capture a measure of stock market participation using zip code level data from the IRS Statistics of Income (SOI) on Individual Income Tax Returns, specifically the fractions of tax returns reporting dividend income and capital gains.

\subsection{Branch Usage data}

We use the Advan (formerly SafeGraph) Monthly Patterns dataset, which provides aggregated raw counts of visits to points of interest (POIs) in the US, gathered from a panel of mobile devices. This anonymized and aggregated dataset provides details on monthly visitor frequency, duration, the origin census block group, and the distance the median branch visitor traveled. The dataset initiates from January 2019 and ends in 2023; however, we do not have usage data during the height of the pandemic. We use these data to identify how many customers use each bank branch and how far the median branch visitor traveled. Because of its limited coverage of branches closed prior to 2022, and because we lag the usage measures, we report branch closure models controlling for usage only for 2022 and 2023.

\subsection{Other Data Sources}

In addition to the primary datasets outlined above, we incorporate several other data sources to construct control variables for our regressions. The Home Mortgage Disclosure Act (HMDA) data is used to calculate mortgage loan growth at the county level, while the Community Reinvestment Act (CRA) data is used to calculate small business loan growth at the county level. County Business Patterns (CBP) data is used to calculate establishment growth and payroll growth at the county level. Additionally, the Federal Housing Finance Agency (FHFA) Underserved Areas data is used to create a zip code-level dummy variable indicating whether a zip code is classified as a low-to-moderate income (LMI) area, defined as having a median income below 80\% of the area median income.  


\section{The Cross Section of The deposit franchise (DF)}

We follow \cite{dss2023}, who construct a simple measure of bank’s deposit franchise value, equal to the present value of gains associated with pricing the current stock of deposits at below-market rates of interest (so the deposit interest rate $r_i^d < r^f$, the market interest rate).\footnote{The DSS framework does not capture value created by expectations of future deposit growth. Since banks would clearly care about growth, we capture this effect (crudely) by including past growth of total deposits (as well as small business loans and mortgages) into our closure models.} We follow \citet{dss2023} in assuming that banks set their deposit rate equal to a fixed fraction of the market rate (so, $r_i^d = \beta_i \times r^f$ ), and that banks with different depositor clientele and banks operating in markets with different levels of competition will optimally choose different deposit betas. Also following \citet{dss2023}, we assume deposits run off equally in each year over the next 10 years. With these assumptions, the DF for bank \textit{i }is given by:


\begin{equation}
DF_i = (1-\beta_i - \frac{c_i}{r^p}) \times [1-\frac{1}{(1+r^p)^{10}}]
\end{equation}


where $\beta_i = \Delta r_i^d/ \Delta r^f$,  $r_i^d$ is the interest expense on deposits per dollar of deposits, $r^f$ is the Fed Funds rate, and  $r^p$ is the the long-term interest rate, which is assumed to be 2.5\%. 

In our application, we set \textit{c} (the annual operating cost per dollar of deposits) to zero, as we do not have good ways to capture the annual per-dollar operating costs associated with each branch. To the extent that these costs are similar across branches for each bank, they would have no effect on our closure model because we estimate all of our effects within bank (and year). In other words, the costs would be absorbed by the bank-year fixed effects. Consistent with this claim, \citet{narayanan2024depositor} show no relationship between customer demographics and bank non-interest expenses (a rough measure of \textit{c}).

We estimate $DF_i$ per dollar of deposits during two periods from bank-by-bank realizations of $\beta_i$, first estimated during the 2016-2019 rate cycle, and second using the same framework over the 2022-2024 rate cycle. We first estimate cross-sectional regressions to reveal how local factors affect the deposit franchise value. This allows both the mean level of DF to evolve over time and also allows the cross-sectional effects of demographics and concentration to vary over time. Realized $\beta_i$ equals the change in bank annualized interest expenses per dollar of deposits over each cycle (from Bank \textit{Call Reports}), normalized by the change in Federal Funds rate over that cycle (=2.5\% in the 2016-2019 cycle and 4\% during 2022-2023). We use just the increasing portion of the two rate cycles. The end of the 2016-2019 cycle coincides with the COVID Pandemic, making the subsequent rate decline and bank reaction to that decline difficult to interpret.\footnote{Banks received massive deposit inflows during this time due to large transfer payments to households, workers and firms under the CARES Act. These exogenous shocks to deposits may disturb the normal pricing reaction of banks to a decline in interest rates which would not be a good representation of their pricing power.} The sample period in the second cycle ends at the first quarter of 2023 to ensure that changes following the Silicon Valley Bank (SVB) collapse do not impact our estimations. 

To build cross-bank local drivers of $\beta$, we average the demographic and market characteristics of residents living near each bank’s branch (based on zip code), weighted by the amount of deposits each bank holds in each of its branches.\footnote{We rely on the pooled models (i.e., large and small banks together) to identify the effects of local variables on the deposit franchise. Estimating these models for large banks alone would be problematic because they own branch networks distributed widely across the country, thus restricting the variation in their exposure to (averaged) local factors.} These regressions are structured, as follows:

\begin{equation}
    \beta_{i,t} = \sum \gamma_t^k D_{i,t}^k + \eta_t HHI_{i,t} +\text{Other controls}+\varepsilon_{i,t}
\end{equation}

where \textit{i} represents bank, \textit{t} represents one of the two rate cycles (2016-19 or 2022-24), \textit{k} represents four demographic variables: age (using three quartile-binned indicators), log of mean family income, the fraction of tax filers reporting stock-based income, and the fraction with a college degree. We average each of these demographics across each zip code in which bank \textit{i} owns its branches, weighted by deposits in each branch. In addition, we capture the deposit-weighted average level of concentration across each bank’s markets (\textit{HHI\textsubscript{i,t}}), where markets are defined at the county level. The other control variables include bank-level population density, calculated as the county-level population density weighted by deposits in each county, a measure of bank size (log of total assets), plus a constant. The dependent variable in Equation (2) equals the change in the interest expenses on deposits per dollar of deposits in each cycle scaled by the corresponding change in the Fed Funds rate (0.025 in the first cycle and 0.04 in the second).

Table \ref{table1} reports summary statistics for the regression samples in the two cycles, separated for large ($>$\$100 billion) and small banks. Explanatory variables are measured at the beginning of each rate cycle. For small banks, the mean realized $\beta_i$ equals 0.17 in the first interest-rate cycle and rises to 0.18 in the second; for large banks, the change was more striking, with the mean $\beta_i$ rising from 0.25 to 0.36, which implies a decline in the DF of about 20\% (recall Equation 1).\footnote{In principle, one could build an analogous metric based on loan-market pricing power.  However, Call Report measures of loan pricing, such as average interest income on C&I loans, would be driven mainly by large borrowers; such loans would not reflect local pricing power.  Moreover, most of the variation in observed lending rates reflects differences in risk rather than mark-ups from market power.  Hence, as we describe below we use quantity-based measures of local lending conditions from both the mortgage markets and the market for loans to small businesses.}  Large banks operate in areas with younger, more educated, wealthier populations that have higher rates of stock market participation than small banks. These demographic differences do not vary much over time across large and small banks. Small banks are more likely to have branches in rural areas.

Table \ref{table2} reports estimates of Equation (2) for the two rate cycles. Age, income and education are both correlated with bank pricing power as expected (columns 1 and 2). Banks with (potential) customers near their branches who are younger and more highly educated have \textit{lower} pricing power (higher $\beta$). Banks have higher pricing power in areas with higher income. The age effect is driven by the oldest quartile. Increasing the share of a bank’s clientele with a college degree by one sigma (from the large-bank sample) raises $\beta$ by 0.021 (=0.16 x 0.13) in the first cycle and by 0.023 (= 0.17 x 0.14) in the second. While market concentration (HHI) enters both models negatively (suggesting greater bank pricing power in more concentrated areas), its effects are not statistically significant. In contrast, local population density – which itself is strongly correlated with HHI - has a large impact. For example, increasing population density by one sigma raises $\beta$ by 0.011 in the first interest-rate cycle and 0.017 in the second. Large banks also exhibit much higher $\beta$s than small, and this effect also increases by more than 50\% in the second cycle.\footnote{Note that we absorb size effects in our closure models with fixed effects.}

Columns 3 and 4 of Table \ref{table2} report a more parsimonious model which collapses three demographic characteristics into one, equal to the fraction of residents in ‘sophisticated’ zip codes. We build the zip-code classification by flagging localities with above-median education, above-median fraction with stock market income, and above-median income. We use this indicator to differentiate areas dominated by financially sophisticated people versus those without. This simple indicator equals 1 for about one-third of the zip codes in our data, which reflects the high correlation across the three underlying measures. (If the three measures were uncorrelated, only 12.5\% (=$0.5^3$) of areas would be flagged as financially sophisticated.) We include age in these models as a separate factor, as age is not strongly correlated with the other three measures. As the results show, the deposit $\beta$ is 0.0212 higher in sophisticated areas during the first interest-rate cycle, and 0.0197 higher in the second. 

We construct branch-level measures of the DF by applying the coefficients from Equation (2) to each bank’s branches using the demographic measures from each branch’s zip code (opposed to the average across all branches, as in Equation 1). We use the coefficients from the first interest-rate cycle for the years 2015-2019, and the coefficients from the second cycle for the years 2020-2023.\footnote{Our aim is the allow the marginal effects of customer demographics to shift over time. We recognize that the coefficients we use are not strictly out-of-sample. Changing this mapping has little effect on our results because the coefficients are fairly stable, and because the demographic variables are very persistent.} Since banks own branches in zip codes with different demographic factors, this procedure generates within-bank variation in the value of the deposit franchise.

Figure \ref{fig:pct_closed_df_bin} illustrates our core result: branch closures respond strongly to variation in deposit pricing power.  We report a bin-scatter plot of the annual percentage change in branches during the first and second rate cycles, where each bin is defined based on the branch-level predicted DF (Panel A), or the bank-level actual value (Panel B).  The Figure reveals a strong pattern, which is that low franchise value increases branch closures; the effect is evident in both the branch-based and bank-based panels. This trend is evident across the DF distribution in both interest rate cycles, with a somewhat stronger relationship observed during the second cycle. The difference in closure rates is large, with low DF branches closed at a rate of about 3 - 5 percent per year, compared to about less than 2 percent for branches with low DF.

Figure \ref{fig:df_density} reports the histograms for the predicted measures of the DF at branch level (Panel A), as well as the bank-level DF (Panel B), both split by year and by bank size. Large banks have lower DF on average because they have higher $\beta$s, although there is substantial overlap in the distributions. 

Figure \ref{fig:df_scatter} reports a scatter plot of the  DF between the two rate cycles, with Panel A using the branch-based predicted values and Panel B using the actual bank-level measure. As expected, these are highly correlated, with a tighter relationship in the branch-based approach, which reflects the stability of both the explanatory factors over time (demographics are very slow moving), as well as the coefficient stability documented in Table \ref{table2}. The figure also shows that the DF is lower in the later period, with a decline that is highest for banks with lower DF in the earlier period. (The dashed line represents a 45 degree line.) This occurs because the effects of both bank size and population density on bank-level $\beta$s \textit{increased} sharply in the second period; larger banks and banks in urban areas have higher $\beta$s, and their effects are getting stronger over time. We use these predicted values below in our core models of branch closure.


\section{Predicting Branch Usage}

As we have explained, customer demographics influence branch closures in two key ways: by impacting banks' pricing power and by altering the value of close physical proximity to a bank branch through differences in technology adoption. We cannot directly measure technology adoption at the branch level, so instead, we use customer demographics to explain branch foot traffic from the Advan data. This approach captures in-person branch usage, which depends on physical proximity, as opposed to usage through a website or mobile banking App. Lower branch usage, we argue, suggests greater reliance on technology. The Advan foot traffic data allow us to measure how many people enter each branch and how far the median visitor travels on average to reach the branch. More visits to a given branch suggest a higher value of the branch, while higher average distance suggests a lower value. These metrics are based on location and time-stamped data from the cell phone network. With them, we report models with the following structure:

\begin{equation}
    \text{Usage}_{j,t} = \sum \gamma_t^k D_{j,t}^k +\text{Other controls}+\varepsilon_{j,t}
\end{equation}

where \textit{j} indexes branches, \textit{k} indexes the demographic variables observed in each branch \textit{j}’s zip code, and \textit{t} represents year (2021and 2022).\footnote{As noted, we do not have cell phone data during the Pandemic year of 2020. Even if we did, these data would be highly unrepresentative of normal behavior due to the effect of lockdowns and general fear of COVID contagion.} The control variables include a $bank \times year$ fixed effect, the county-level population density, and the log of lagged total deposits held in branch \textit{j}.

Table \ref{table3} reports the estimates of equation (3) with all four demographic factors (Panel B), and also the more parsimonious version in which we collapse income, education and stock-market participation into a single sophisticated indicator (Panel A), as in Table \ref{table2}. Regressions are split based on bank size.

These results show a strong effect of local demographics on branch usage. For large and small banks usage is low in areas with a more sophisticated clientele. The number of visits, for example, is 14\% lower at large-bank branches located in these areas. Moreover, visitors from sophisticated areas are on average traveling from further away when they do visit a branch. In other words, customers in financially sophisticated locations value physical branches less than customers in other areas. Age also correlates strongly with usage, but this variable is harder to interpret, as older people in general are less active cell phone users, less financially sophisticated on average, and less mobile compared to younger people. As the model shows, the age effects on number of visits is declining (older people visit less often), while the effect on distance is non-monotonic across the distribution.

Next, we turn to studying how bank branch usage patterns evolved over time, particularly in response to the COVID-19 pandemic and the associated rise in remote work. To examine these changes, we estimate a regression of the following form:

\begin{equation}
    \text{Usage Metric}_{i,m} = \sum_m \beta_m \times \text{Sophisticated zip} \times (month=m) +\text{Other controls}+\text{Fixed effects}+\varepsilon_{i,m}
\end{equation}
where \textit{i} is the branch and \textit{m }is the month. The two usage measures are the natural logarithm number of unique visitors and the median distance a visitor travelled. The regression includes bank, month, and zip code fixed effects. The coefficient  captures the differential effect of financial sophistication on branch usage across months, relative to the omitted base month (January 2019).

Figure \ref{fig:dynamic_did_plots} reports the  estimates with the corresponding 90\% confidence intervals, presented separately for large and small banks. Panel A uses log(number of visitors) as the dependent variable and Panel B uses the natural logarithm of the median travel distance to the branch. The clear pattern in Figure \ref{fig:dynamic_did_plots} is that the number of visitors to branches significantly declined over time in sophisticated areas (relative to less sophisticated), particularly after the 2020 Pandemic, with the effect being more pronounced for smaller banks. However, the average travel distance to branches showed no significant change, indicating that while fewer sophisticated people visited branches in these areas, the geographical reach of branch visitors remained stable. This finding aligns with broader societal shifts induced by the pandemic. Lockdown rules forced many people to rearrange their work schedules, leading to lasting behavioral changes, including a significantly higher prevalence of working from home. As \citet{barrero2023long} document, only about 5\% of Americans worked from home before the pandemic, but this figure surged to 60\% during the lockdowns and has since stabilized at around 30\%. Notably, the increase in remote work was far greater for individuals with a college degree or higher, reflecting their greater ability to adapt to remote work arrangements. The large difference in usage patterns based on demographics follows because higher income and more educated people were more likely to be able to work at home, compared to other people. 

As noted, these cross-sectional and time series patterns suggest that more financially sophisticated people value proximity to a nearby bank branch less than other people: they visit branches at lower frequency and travel further when they do visit. Such effects, we argue, occur because these customers are accessing banking services with technology – internet and mobile banking – more aggressively than less sophisticated customers. The results of Tables \ref{table2} and \ref{table3} together imply that a branch’s natural depositor clientele – the people living near the branch - drives both the pricing power of those branches (Table \ref{table2}) as well as the usage of those branches (Table \ref{table3}). Next, we test whether these two factors help explain branch closure decisions.

 

\section{Branch Closures}

\subsection{Empirical Design}

To understand the drivers of branch closures, we estimate three types of models. The first focuses on the deposit franchise value and uses the following linear probability model where the dependent variable (\textit{Closure}) indicates if a particular branch \textit{j} was closed in year \textit{t}:

\begin{equation}
    \text{Closure}_{j,t} = \gamma DF_{j,t} + \text{Other controls} + \text{Fixed effects} + \varepsilon_{j,t} \tag{5a}
\end{equation}

In (5a), \textit{DF\textsubscript{j,t}} equals the predicted value of the deposit franchise for branch \textit{j} and time \textit{t}, as described above. Our baseline estimates pool the branch-year data across the full 2015-2023 period. We report models with $state\times year$ and $bank \times year$  fixed effects, and we report models with  $county \times year$ fixed effects as well. By including $bank \times year $ fixed effects (as well as  $county \times year$ effects in some specifications), we fully absorb both the general trends in banking and technology, as well as heterogeneity in the supply of technology across banks. For instance, \citet{haendler2022keeping} shows that large banks adopted and updated mobile apps earlier and more frequently than smaller banks. This approach absorbs supply-side differences in the quality and quantity of online and mobile banking services, as these are common across all customers of a given bank, regardless of branch location. In addition, the $county \times year$ absorbs variation in local access to technology, such as differences in investment in the quality of the cell phone network. Thus, identification comes solely from variation in the impact of local demographics (i.e., demand-side factors) on branch closures.

At this stage, we exclude measures of branch usage, as these are only available for the final two years of our sample. We also estimate the model separately for large and small banks, reflecting differences in customer demographics (\citet{d2023deposit}) and the potential variation in marginal effects due to differences in the quantity and quality of services offered.

For control variables, we include the lagged log level of deposits, the three-year past growth rate of (i) deposits, (ii) mortgage applications and (iii) small business loan originations to capture local supply and demand conditions for deposits and loans. County-level growth in the number of establishments and payroll serves capture local economic growth. We also control for county-level population density. Two M\&A-related indicators are added: one equals one if the bank has owned at least one branch in the zip code for the past three years, and the other equals one if the current branch was acquired by the bank in the past three years and bank has owned at least one other branch in the same zip code prior to the acquisition. Finally, we include an indicator for branches in low- and moderate-income (LMI) areas, as defined by the Community Reinvestment Act, where banks face regulatory pressure to lend locally.

In our second set of models, we extend the analysis by using a reduced-form version of (5a), replacing the predicted deposit franchise (DF) metric with the underlying demographic and market concentration variables that were used to construct it. The regression specification is as follows:

\begin{equation}
    \text{Closure}_{j,t} = \sum_k \gamma_t^k D_{j,t}^k + \eta HHI_{j,t} + \text{Other controls} + \text{Fixed effects} + \varepsilon_{j,t} \tag{5b}
\end{equation}

Here, $D_{j,t}^k$ represents the k-th demographic variable at branch \textit{j} and time \textit{t}. Fixed effects include $bank \times year$ and  $state \times year$ or $county \times year$ . This approach allows us to test which local variables are most tightly linked to branch closure decisions. In these models, we again report specifications combining the three of the demographics into a single financial sophistication measure (as above: based on education, income and stock-market participation), and we also report the unconstrained model. Standard errors are  clustering at the bank level.

Our third set of models incorporates cell phone-based branch usage measures, including the log of the number of visitors and the average distance traveled to the branch. This specification assesses the relative importance of pricing power (deposit franchise value) versus customer convenience or the amenity value of proximity:

\begin{equation}
    \text{Closure}_{j,t} = \gamma_1 DF_{j,t} + \gamma_2 \text{Usage}_{j,t} + \text{Other controls} + \text{Fixed effects} + \varepsilon_{j,t} \tag{5c}
\end{equation}

Although the Advan data start in 2019, we estimate Equation (5c) for closures only in 2022 and 2023. The industry (NAICS) codes for the closed branches were changed in 2021. As a result, branches closed in 2020 and 2021 have different NAICS in the Advan data, which doesn't provide a historical time series of these codes by location. When creating the dataset, we initially filtered locations with NAICS code 522 to indicate a banking office, so by necessity we filtered out these locations due to the change in the NAICS to a different code.\footnote{It is computationally prohibitive to standardize all the US addresses and then match only by the address without first filtering by the NAICS code.}

\subsection{Results}

We first present summary statistics of the branch-year panel sample used for the closure regressions, focusing on the year 2019. Table \ref{table4} provides an overview of key characteristics, split by bank size, to highlight differences between small and large banks. While many mean characteristics are similar across the two groups, there are notable distinctions. Large banks, for example, hold significantly more deposits per branch, with the typical branch holding more than twice as many deposits as those of small banks. Small banks exhibit faster lending growth to small businesses. Geographically, small banks are more prevalent in rural areas, as indicated by their branches being in regions with much lower population density. 

Table \ref{table5} reports our estimates of Equation (5a). We report the pooled sample (columns 1 \& 2), and then the split-sample results by bank size (over versus under \$100 billion in assets) in columns 3-6. For each set of specifications, we report models with $state \times year$ fixed effects and separately with $county \times year$ fixed effects. For all banks, in columns (1) and (2), higher DF leads to lower probability of a branch being closed. The magnitude is substantial across all banks, but also larger for the large banks. A one-standard-deviation increase in the DF (= 0.006), for example, leads to a decline in annual branch closure probability of about 1.0 percentage points (column 4).\footnote{To assess economic magnitudes, we remove variation in the DF which is explained by the fixed effects, which lowers the standard deviation from 0.01 to 0.006.} This effect is large, equal to about one-quarter of the unconditional mean closure rate (=4\% per year for the large banks) during our sample. Beyond the DF, which captures the value of the current stock of deposits, higher levels of deposit growth in the branch’s local market (county) from the preceding three years also has strong power to predict branch closures for large banks. Note that the market-level growth effect is only measurable when we leave the $county \times year$ fixed effects out. Hence the value of both ``deposits in place” and ``deposit growth opportunities” seem to drive closure decisions.

In contrast to deposits, neither the growth in mortgage applications nor the growth in small business lending has much power to explain branch closures. We do find a marginally significant effect of small business lending growth on smaller banks’ branch closures, but none for larger ones. This contrast suggests that the core purpose of bank branches (especially for large banks) is to support the deposit franchise, where banks remain dominant. On the other hand, banks have become increasingly \textit{less} important as suppliers of local credit - mortgages and small business loans. Moreover, deposits constitute about 85\% of all bank financing, while local lending comprises a small percentage of total bank investments (again, especially for larger banks). The model also suggests that banks are much more likely to close branches acquired recently if they already had branch presence in the zip code, and less likely to close `legacy’ branches, meaning those which have not been acquired over the past three years. We find no evidence that banks close branches in localities defined as LMIs under the Community Banking Act; if anything, large banks are \textit{less} apt to close branches in these areas.\footnote{Banks are required to give regulators, and local customers notice before closing a branch under Section 42 of the Federal Deposit Insurance Act. Hence, large banks may be concerned that closing branches in LMI areas could lower their CRA rating, which in turn could impinge on future acquisitions.}

Table \ref{table6} reports parallel models of branch closure estimated as reduced forms (Equation (5b)). Panel A reports the full sample, and Panel B split by size. We find that branch closures are more common when local clientele are more educated, have lower income, and invest more in the stock market. Each of these effects is consistent with our interpretation that the value of the DF plays a key role in driving branch closure decisions. Comparing the results from Table 2 and Table \ref{table6}, we learn that each factor which lowers the DF (i.e., raises beta) in turn leads to more branch closures. Panel B shows similar relationships between the small and large banks, with one exception, which is age. For large banks, the age of residents near the branch enters non-monotonically, with the probability of closure increasing from the youngest quartile, then decreasing (slightly) with age thereafter. For small banks, closure rates fall with age across the whole distribution. The other effects enter with the same sign patterns but different magnitudes. 

Table \ref{table7} reports year by year estimates of Equation (5a); Panels A and B report the model for large banks and Panel C and D for small ones (with $county \times year$ or $state \times year$ effects). These results suggest that branches with low deposit franchise value are consistently most likely to be closed, with the marginal effect of the DF increasing over time. The effect is largest in 2020, the Pandemic year. For large banks, a standard-deviation decline in the deposit franchise (=0.006) comes with an increase in closure probability of almost 2.7 percentage points (about 1.0 percentage points for small banks). The large-bank results suggest that the marginal effect of the DF is higher in all years after the Pandemic than before.

Table \ref{table8} reports our estimates of Equation (5c), which allows us to compare the relative importance of branch usage (from foot traffic) vs. banks’ deposit pricing power (DF). These regressions include only the last two years of our sample due to data constraints, as noted above. The results continue to show that DF remains the dominant driver of branch closures. Adding the usage metrics slightly attenuates the effect of the value of deposits, from –1.36 to –1.25 for large banks. We find no strong relationship between DF and closure for the small banks, however, once we absorb the $county \times year$ effects. Table \ref{table8} also shows that while the number of visitors to a branch has little effect on closures, banks are more likely to close branches with customers traveling from greater distances.


\section{Conclusion}

Banks have closed branches at a high rate since 2015, reversing their long-term but gradual growth. We show that declines in economic profits generated from bank deposits helps explain this trend, as banks are most likely to close branches with low franchise value. Technologies which make physical proximity less important and which lowers the cost of moving funds to substitute investments drove these changes. In contrast to other research, our empirical design exploits different rates of technology adoption by bank customers, which varies depending on characteristics associated with financial sophistication. The results suggest that the 2020 Pandemic had a large effect on adoption, as the overall rate of branch closure doubled at the same time that the marginal effect of our measure of the deposit franchise also rose sharply.

Understanding the drivers of branch closures matters because branch-based frictions have traditionally mediated flows of capital across markets and have affected local-market competition in both deposit and credit markets. Such frictions reduce financial market efficiency and integration. Lowering these frictions through technology furthers a process which began in the 1980s with deregulation of restrictions on branching and interstate banking. As such, continued bank restructuring will likely improve the functioning of local financial markets further.




\onehalfspacing
\clearpage
\bibliographystyle{chicago}
\bibliography{resources/branchclosure}

\newcommand{\sigtextable}{Standard errors (reported in parentheses) are clustered at the bank level. We use *,**, and *** to denote statistical significance at the 10\%, 5\%, and 1\% levels, respectively.}

\clearpage
\begin{figure}
    \centering
    \adddescription{
    \small
This figure shows the total number of banks (blue line, left axis) and the total number of bank offices (orange line, right axis) in the United States from 1934 to 2023. 
    }
    \includegraphics[width=\textwidth]{resources/figures/old/bank_branch_line_plot.jpg}
    \caption{Total Number of Banks and Bank Offices, 1934-2023}
    \label{fig:no_of_banks_branches}
\end{figure}



\clearpage
\begin{figure}
    \centering
    \adddescription{
    \small
This figure displays the annual percentage of branches closed by small banks ($<$100bn in assets, orange bars) and large banks ($>=$100bn in assets, blue bars) from 2015 to 2023. The x-axis represents the years, while the y-axis shows the percentage of branches closed.
    }
    \includegraphics[width=\textwidth]{resources/figures/old/pct_branches_closed.jpg}
    \caption{Percent of Branches Closed}
    \label{fig:pct_branches_closed}
\end{figure}


\clearpage
\begin{table}[]
    \centering
    \adddescription{
    \small
This table presents summary statistics for the bank-level samples during two interest rate cycles, separated for large banks ($>$100 billion in assets) and small banks. Panels A and B correspond to the late interest rate cycle, while Panels C and D correspond to the early cycle. For each group, statistics include the mean, standard deviation (SD), and selected percentiles (P10, P25, P50, P75, P90) for variables such as deposit beta, age, college-educated fraction, stock market participation fraction, family income, fraction of deposits in sophisticated zip codes, Herfindahl-Hirschman Index (HHI), and deposit-weighted population density. Observations are noted for each variable within the respective samples.
    }
    \vspace{0.5cm}
    {\small Panel A: Large Banks (Late Cycle)}
    \resizebox{0.8\textwidth}{!}{
        \begin{tabular}{lrrrrrrrr}
          \hline
        Variable & Obs & Mean & SD & P10 & P25 & P50 & P75 & P90 \\ 
          \hline
Deposit beta &  22 & 0.36 & 0.15 & 0.22 & 0.26 & 0.31 & 0.48 & 0.60 \\ 
  Age &  22 & 37.55 & 2.26 & 34.45 & 36.44 & 37.99 & 39.02 & 39.43 \\ 
  College educated fraction &  22 & 0.52 & 0.17 & 0.38 & 0.43 & 0.48 & 0.58 & 0.75 \\ 
  Stock market participation frac &  22 & 0.28 & 0.10 & 0.16 & 0.23 & 0.27 & 0.33 & 0.39 \\ 
  Family income (000) &  22 & 80.77 & 34.28 & 56.60 & 63.50 & 71.00 & 85.25 & 102.20 \\ 
  Frac. deposits in sophisticated zipcodes &  22 & 0.61 & 0.30 & 0.04 & 0.49 & 0.69 & 0.84 & 0.91 \\ 
  HHI &  22 & 0.28 & 0.18 & 0.14 & 0.19 & 0.22 & 0.29 & 0.40 \\ 
  Deposit-weighted Pop. density &  22 & 0.29 & 0.07 & 0.20 & 0.27 & 0.31 & 0.35 & 0.36 \\ 
           \hline
        \end{tabular}
        }
    \\
    \vspace{0.5cm}
{\small Panel B: Small Banks (Late Cycle)}

    \resizebox{0.8\textwidth}{!}{
    \begin{tabular}{lrrrrrrrr}
      \hline
    Variable & Obs & Mean & SD & P10 & P25 & P50 & P75 & P90 \\ 
      \hline
Deposit beta & 4285 & 0.18 & 0.12 & 0.04 & 0.09 & 0.16 & 0.25 & 0.35 \\ 
  Age & 4285 & 40.78 & 4.41 & 35.59 & 38.09 & 40.71 & 43.26 & 45.86 \\ 
  College educated fraction & 4285 & 0.29 & 0.14 & 0.15 & 0.19 & 0.25 & 0.34 & 0.48 \\ 
  Stock market participation frac & 4285 & 0.20 & 0.08 & 0.10 & 0.14 & 0.19 & 0.24 & 0.29 \\ 
  Family income (000) & 4285 & 57.74 & 18.27 & 39.00 & 46.00 & 54.00 & 65.00 & 80.00 \\ 
  Frac. deposits in sophisticated zipcodes & 4285 & 0.34 & 0.39 & 0.00 & 0.00 & 0.15 & 0.71 & 1.00 \\ 
  HHI & 4285 & 0.24 & 0.13 & 0.11 & 0.15 & 0.21 & 0.29 & 0.40 \\ 
  Deposit-weighted Pop. density & 4285 & 0.10 & 0.12 & 0.01 & 0.01 & 0.04 & 0.14 & 0.36 \\ 
       \hline
    \end{tabular}
    }

\vspace{0.5cm}
{\small Panel C: Large Banks (Early Cycle)}

    \resizebox{0.8\textwidth}{!}{
    \begin{tabular}{lrrrrrrrr}
      \hline
    Variable & Obs & Mean & SD & P10 & P25 & P50 & P75 & P90 \\ 
      \hline
Deposit beta &  26 & 0.25 & 0.10 & 0.15 & 0.20 & 0.22 & 0.29 & 0.40 \\ 
  Age &  26 & 37.86 & 2.83 & 34.32 & 36.44 & 37.99 & 39.14 & 41.18 \\ 
  College educated fraction &  26 & 0.54 & 0.16 & 0.39 & 0.44 & 0.53 & 0.62 & 0.76 \\ 
  Stock market participation frac &  26 & 0.30 & 0.12 & 0.17 & 0.23 & 0.28 & 0.35 & 0.43 \\ 
  Family income (000) &  26 & 84.42 & 33.64 & 59.00 & 65.25 & 73.00 & 94.50 & 115.00 \\ 
  Frac. deposits in sophisticated zipcodes &  26 & 0.64 & 0.30 & 0.18 & 0.49 & 0.71 & 0.85 & 0.96 \\ 
  HHI &  26 & 0.27 & 0.20 & 0.13 & 0.18 & 0.21 & 0.25 & 0.49 \\ 
  Deposit-weighted Pop. density &  26 & 0.30 & 0.07 & 0.21 & 0.29 & 0.32 & 0.36 & 0.36 \\ 
       \hline
    \end{tabular}
}

\vspace{0.5cm}
{\small Panel D: Small Banks (Early Cycle)}
    \resizebox{0.8\textwidth}{!}{
    \begin{tabular}{lrrrrrrrr}
      \hline
        Variable & Obs & Mean & SD & P10 & P25 & P50 & P75 & P90 \\ 
      \hline
Deposit beta & 4880 & 0.17 & 0.12 & 0.02 & 0.08 & 0.15 & 0.25 & 0.34 \\ 
  Age & 4880 & 40.76 & 4.47 & 35.56 & 38.10 & 40.66 & 43.23 & 45.86 \\ 
  College educated fraction & 4880 & 0.29 & 0.14 & 0.16 & 0.19 & 0.25 & 0.36 & 0.50 \\ 
  Stock market participation frac & 4880 & 0.20 & 0.09 & 0.10 & 0.14 & 0.19 & 0.24 & 0.30 \\ 
  Family income (000) & 4880 & 58.60 & 18.94 & 39.00 & 46.00 & 55.00 & 66.00 & 83.00 \\ 
  Frac. deposits in sophisticated zipcodes & 4880 & 0.36 & 0.40 & 0.00 & 0.00 & 0.17 & 0.75 & 1.00 \\ 
  HHI & 4880 & 0.23 & 0.13 & 0.11 & 0.14 & 0.20 & 0.28 & 0.38 \\ 
  Deposit-weighted Pop. density & 4880 & 0.11 & 0.13 & 0.01 & 0.01 & 0.04 & 0.16 & 0.36 \\ 
       \hline
    \end{tabular}
    }

    
    \caption{Descriptive Statistics - Bank Level}
    \label{table1}
\end{table}




\begin{table}[]
    \centering
    \adddescription{
    \small
This table presents estimates of deposit beta from Equation (2) for two interest rate cycles: 2016–2019 (columns 1 and 3) and 2022–2024 (columns 2 and 4). Columns 1 and 2 report detailed specifications with individual demographic variables, including age (quartiles), income, stock market participation, college education, Herfindahl-Hirschman Index (HHI), and deposit-weighted population density. Columns 3 and 4 present a parsimonious specification, collapsing demographic measures into the fraction of deposits in sophisticated zip codes. The dependent variable in all columns is deposit beta. \sigtextable
    }

    \resizebox{\textwidth}{!}{
    \begin{tabular}{@{\extracolsep{5pt}}lcccc} 
    \\[-1.8ex]\hline 
    \hline \\[-1.8ex] 
     & \multicolumn{4}{c}{Deposit Beta} \\ 
    \cline{2-5} 
     & 2016-2019 & 2022-2024 & 2016-2019 & 2022-2024 \\ 
    \\[-1.8ex] & (1) & (2) & (3) & (4)\\ 
    \hline \\[-1.8ex] 
 Frac. deposits in sophisticated zipcodes &  &  & 0.0214$^{***}$ & 0.0194$^{***}$ \\ 
  &  &  & (0.0049) & (0.0049) \\ 
  Age Q1-Q2 & 0.0020 & $-$0.0046 & $-$0.0001 & $-$0.0066 \\ 
  & (0.0045) & (0.0045) & (0.0044) & (0.0045) \\ 
  Age Q2-Q3 & 0.0037 & $-$0.0043 & 0.0026 & $-$0.0051 \\ 
  & (0.0051) & (0.0052) & (0.0049) & (0.0049) \\ 
  Age $>$ Q3 & $-$0.0139$^{**}$ & $-$0.0251$^{***}$ & $-$0.0108 & $-$0.0220$^{***}$ \\ 
  & (0.0071) & (0.0072) & (0.0067) & (0.0068) \\ 
  log(Income) & $-$0.0290$^{***}$ & $-$0.0236$^{**}$ &  &  \\ 
  & (0.0061) & (0.0099) &  &  \\ 
  Stock market participation frac & 0.0626$^{*}$ & 0.0474 &  &  \\ 
  & (0.0337) & (0.0367) &  &  \\ 
  College educated fraction & 0.1387$^{***}$ & 0.1365$^{***}$ &  &  \\ 
  & (0.0251) & (0.0265) &  &  \\ 
  HHI & $-$0.0238$^{*}$ & $-$0.0136 & $-$0.0204 & $-$0.0122 \\ 
  & (0.0138) & (0.0138) & (0.0139) & (0.0137) \\ 
  log(Assets) & 0.0052$^{***}$ & 0.0154$^{***}$ & 0.0068$^{***}$ & 0.0170$^{***}$ \\ 
  & (0.0014) & (0.0014) & (0.0014) & (0.0013) \\ 
  Deposit-weighted Pop. Density & 0.0721$^{***}$ & 0.1143$^{***}$ & 0.1278$^{***}$ & 0.1725$^{***}$ \\ 
  & (0.0210) & (0.0215) & (0.0174) & (0.0178) \\ 
  Transaction deposits/Assets & $-$0.1173$^{***}$ & $-$0.1114$^{***}$ & $-$0.1201$^{***}$ & $-$0.1134$^{***}$ \\ 
  & (0.0209) & (0.0200) & (0.0211) & (0.0201) \\ 
  Constant & 0.3818$^{***}$ & 0.2099$^{**}$ & 0.0848$^{***}$ & $-$0.0301$^{*}$ \\ 
  & (0.0665) & (0.1062) & (0.0179) & (0.0178) \\ 
 \hline \\[-1.8ex] 
Observations & 4,906 & 4,306 & 4,906 & 4,306 \\ 
R$^{2}$ & 0.0893 & 0.1744 & 0.0775 & 0.1657 \\ 
    \hline 
    \hline \\[-1.8ex] 
    \textit{Note:}  & \multicolumn{4}{r}{$^{*}$p$<$0.1; $^{**}$p$<$0.05; $^{***}$p$<$0.01} \\ 
    \end{tabular}
    }
    \caption{Bank-Level Deposit Beta}
    \label{table2}
\end{table}



\clearpage
\begin{figure}
    \centering
    \adddescription{
    \small
This figure illustrates the relationship between the percentage of branches closed (y-axis) and the decile of predicted branch-level deposit franchise (DF) values in Panel A, as well as bank-level deposit franchise (DF) values (x-axis) in Panel B. The data is presented for two distinct interest rate hike cycles: 2016-2019 (represented by orange circles) and 2022-2024 (represented by blue triangles). Each decile corresponds to a grouping of branches in Panel A and banks in Panel B based on their DF values, with lower deciles indicating branches and banks with lower DF values. 
    }
    Panel A: Branch-Level\\
    \includegraphics[width=0.8\textwidth]{resources/figures/old/df_bin_pct_closed.jpg}
    \vspace{0.5cm}\\
    Panel B: Bank-Level\\
        \includegraphics[width=0.8\textwidth]{resources/figures/old/df_bin_pct_closed_bank.jpg}
    \caption{Deposit Franchise and Branch Closure}
    \label{fig:pct_closed_df_bin}
\end{figure}




\clearpage
\begin{figure}
    \centering
    \adddescription{
    \small
Panel A of this figure illustrates density plots of the predicted deposit beta (DF) at the branch level, segmented by year (2015--2023) and bank size. Panel B presents analogous density plots using bank-level data. In both panels, small banks (with assets under \$100 billion) are represented by orange distributions, while large banks (with assets over \$100 billion) are depicted by blue distributions. 
    }
     Panel A: Branch-Level\\
    \includegraphics[width=\textwidth]{resources/figures/old/df_density.jpg}
        \vspace{0.5cm}\\
    Panel B: Bank-Level\\
    \includegraphics[width=0.6\textwidth]{resources/figures/old/df_density_bank.jpg}
    \caption{Deposit Franchise Value per Dollar of Deposits}
    \label{fig:df_density}
\end{figure}




\clearpage
\begin{figure}
    \centering
    \adddescription{
    \small
This scatter plot compares deposit franchise (DF) per dollar between the early cycle (2016–2019) and the late cycle (2022–2024) for small banks ($<$100bn in assets, orange points) and large banks ($>$100bn in assets, blue points). Panel A uses branch-level observations and Panel B uses bank-level observations. The x-axis represents DF per dollar in the early cycle, while the y-axis represents DF per dollar in the late cycle. The solid line represents the fitted relationship, and the dashed line indicates the 45-degree line, serving as a reference for equal DF values between the two periods. Each point corresponds to a bank-branch observation.
    }
    Panel A: Branch-Level\\
    \includegraphics[width=0.6\textwidth]{resources/figures/old/df_scatter.jpg}
            \vspace{0.5cm}\\
    Panel B: Bank-Level\\
    \includegraphics[width=0.6\textwidth]{resources/figures/old/df_scatter_bank.jpg}
    \caption{Deposit Franchise per Dollar: Early vs. Late Cycle}
    \label{fig:df_scatter}
\end{figure}


\clearpage
\begin{table}[]
    \centering
    \adddescription{
\small
This table presents the estimates from Equation (3), analyzing branch usage and the distance customers travel to branches, split by bank size (large or small). Panel A reports results using a parsimonious model with a single sophisticated zip code indicator, while Panel B uses separate demographic variables, including age quartiles, income, education, and stock market participation. Columns 1 and 2 in both panels analyze branch usage, measured as the log of visitors per million deposits, while columns 3 and 4 examine the log of distance (in kilometers) customers travel. Results are reported separately for large and small banks, with controls for bank-year and state-year fixed effects. \sigtextable

    }
\vspace{0.5cm}
{\small Panel A}\\
    \resizebox{0.6\textwidth}{!}{
\begin{tabular}{@{\extracolsep{5pt}}lcccc} 
\\[-1.8ex]\hline 
\hline \\[-1.8ex] 
%  & \multicolumn{4}{c}{\textit{Dependent variable:}} \\ 
% \cline{2-5} 
 & \multicolumn{2}{c}{log(vistors per 1m deposits)} & \multicolumn{2}{c}{log(distance km)} \\ 
\\[-1.8ex] & (1) & (2) & (3) & (4)\\ 
\hline \\[-1.8ex] 
 Sophisticated zipcode & $-$0.136$^{***}$ & $-$0.083$^{***}$ & 0.119$^{***}$ & 0.125$^{***}$ \\ 
  & (0.046) & (0.017) & (0.016) & (0.014) \\ 
  Age Q1-Q2 & $-$0.029 & $-$0.046$^{***}$ & $-$0.116$^{***}$ & $-$0.107$^{***}$ \\ 
  & (0.025) & (0.018) & (0.013) & (0.013) \\ 
  Age Q2-Q3 & $-$0.128$^{***}$ & $-$0.158$^{***}$ & $-$0.160$^{***}$ & $-$0.151$^{***}$ \\ 
  & (0.043) & (0.019) & (0.021) & (0.014) \\ 
  Age $>$ Q3 & $-$0.330$^{***}$ & $-$0.291$^{***}$ & $-$0.036$^{*}$ & 0.009 \\ 
  & (0.056) & (0.027) & (0.021) & (0.018) \\ 
  log(Deposits) & $-$0.830$^{***}$ & $-$0.752$^{***}$ & 0.108$^{***}$ & 0.021$^{***}$ \\ 
  & (0.052) & (0.014) & (0.012) & (0.008) \\ 
  County population density & $-$0.352$^{**}$ & $-$0.450$^{***}$ & $-$1.057$^{***}$ & $-$0.294$^{***}$ \\ 
  & (0.132) & (0.084) & (0.061) & (0.088) \\ 
 \hline \\[-1.8ex] 
Sample & Large & Small & Large & Small \\ 
Bank$\times$Year FE & Y & Y & Y & Y \\ 
State$\times$Year FE & Y & Y & Y & Y \\ 
Observations & 45,116 & 52,010 & 45,116 & 52,010 \\ 
R$^{2}$ & 0.455 & 0.724 & 0.236 & 0.350 \\ 
\hline 
\hline \\[-1.8ex] 
\textit{Note:}  & \multicolumn{4}{r}{$^{*}$p$<$0.1; $^{**}$p$<$0.05; $^{***}$p$<$0.01} \\ 
\end{tabular} 
}

\vspace{1cm}
{\small Panel B}\\

    \resizebox{0.6\textwidth}{!}{
\begin{tabular}{@{\extracolsep{5pt}}lcccc} 
\\[-1.8ex]\hline 
\hline \\[-1.8ex] 
%  & \multicolumn{4}{c}{\textit{Dependent variable:}} \\ 
% \cline{2-5} 
 & \multicolumn{2}{c}{log(vistors per 1m deposits)} & \multicolumn{2}{c}{log(distance km)} \\ 
\\[-1.8ex] & (1) & (2) & (3) & (4)\\ 
\hline \\[-1.8ex] 
 College educated fraction & $-$0.065 & 0.264$^{***}$ & 0.490$^{***}$ & 0.473$^{***}$ \\ 
  & (0.106) & (0.093) & (0.043) & (0.057) \\ 
  log(Income) & $-$0.004 & $-$0.013 & $-$0.124$^{***}$ & $-$0.070$^{***}$ \\ 
  & (0.016) & (0.010) & (0.009) & (0.008) \\ 
  Stock market participation frac & $-$0.987$^{***}$ & $-$0.719$^{***}$ & 0.914$^{***}$ & 0.994$^{***}$ \\ 
  & (0.192) & (0.101) & (0.062) & (0.096) \\ 
  Age Q1-Q2 & $-$0.070$^{***}$ & $-$0.099$^{***}$ & $-$0.053$^{***}$ & $-$0.056$^{***}$ \\ 
  & (0.020) & (0.019) & (0.011) & (0.011) \\ 
  Age Q2-Q3 & $-$0.133$^{***}$ & $-$0.207$^{***}$ & $-$0.136$^{***}$ & $-$0.124$^{***}$ \\ 
  & (0.039) & (0.021) & (0.022) & (0.014) \\ 
  Age $>$ Q3 & $-$0.267$^{***}$ & $-$0.320$^{***}$ & $-$0.088$^{***}$ & $-$0.006 \\ 
  & (0.060) & (0.029) & (0.030) & (0.018) \\ 
  log(Deposits) & $-$0.793$^{***}$ & $-$0.749$^{***}$ & 0.060$^{***}$ & 0.007 \\ 
  & (0.058) & (0.014) & (0.008) & (0.007) \\ 
  County population density & $-$0.246$^{*}$ & $-$0.456$^{***}$ & $-$1.217$^{***}$ & $-$0.568$^{***}$ \\ 
  & (0.130) & (0.087) & (0.057) & (0.085) \\ 
 \hline \\[-1.8ex] 
Sample & Large & Small & Large & Small \\ 
Bank$\times$Year FE & Y & Y & Y & Y \\ 
State$\times$Year FE & Y & Y & Y & Y \\ 
Observations & 45,116 & 52,010 & 45,116 & 52,010 \\ 
R$^{2}$ & 0.464 & 0.725 & 0.293 & 0.381 \\ 
\hline 
\hline \\[-1.8ex] 
\textit{Note:}  & \multicolumn{4}{r}{$^{*}$p$<$0.1; $^{**}$p$<$0.05; $^{***}$p$<$0.01} \\ 
\end{tabular}
    }
    \caption{Usage}
    \label{table3}
\end{table}


\clearpage
\begin{figure}
    \centering
    \adddescription{
    \small
This figure illustrates the evolution of branch usage metrics over time, focusing on the impact of the COVID-19 pandemic. Panel A shows the logarithm of the number of visitors per month, while Panel B shows the logarithm of the median travel distance to branches. Panels A.1 and B.1 correspond to large banks, and Panels A.2 and B.2 correspond to small banks. The x-axis represents months, and the y-axis represents the estimated coefficients ($\beta_m$) with 90\% confidence intervals, capturing the differential effect of branch location in sophisticated zip codes on usage metrics, relative to the omitted base month (January 2019). \sigtextable
    }
 \begin{subfigure}[b]{0.45\textwidth}
        \centering
        \caption*{\footnotesize Panel A.1: Large Banks}
        \includegraphics[width=\textwidth]{resources/figures/old/dynamic_a1.jpg}
        
    \end{subfigure}
    \hfill
    \begin{subfigure}[b]{0.45\textwidth}
        \centering
                \caption*{\footnotesize Panel A.2: Small Banks}
        \includegraphics[width=\textwidth]{resources/figures/old/dynamic_a2.jpg}

    \end{subfigure}
    
    % Second row of 2x2 grid
    \vspace{1em}  % Add some vertical space between rows
    \begin{subfigure}[b]{0.45\textwidth}
        \centering
                \caption*{\footnotesize Panel B.1: Large Banks}
        \includegraphics[width=\textwidth]{resources/figures/old/dynamic_b1.jpg}

    \end{subfigure}
    \hfill
    \begin{subfigure}[b]{0.45\textwidth}
        \centering
        \caption*{\footnotesize Panel B.2: Small Banks}
        \includegraphics[width=\textwidth]{resources/figures/old/dynamic_b2.jpg}
        
    \end{subfigure}
         \caption{Dynamic Branch Usage}
    \label{fig:dynamic_did_plots}
\end{figure}   

\clearpage
\begin{table}[]
    \centering
    \adddescription{
\small
This table provides summary statistics for branch-level data in 2019, split by bank size. Panel A presents statistics for large banks, and Panel B for small banks. For each variable, the table reports the number of observations, mean, standard deviation (SD), and selected percentiles (P10, P25, P50, P75, P90). This breakdown highlights differences between large and small banks in branch characteristics and regional distributions.









    }

\vspace{0.5cm}
{\small Panel A: Large Banks}\\
    \resizebox{\textwidth}{!}{
\begin{tabular}{lrrrrrrrr}
  \hline
Variable & Obs & Mean & SD & P10 & P25 & P50 & P75 & P90 \\ 
  \hline
Deposits (mn) & 29427 & 229.52 & 3550.77 & 26.00 & 44.00 & 75.00 & 129.00 & 224.00 \\ 
  Deposit 3yr growth & 29427 & 0.04 & 0.03 & -0.00 & 0.03 & 0.04 & 0.06 & 0.08 \\ 
  Mortgage 3yr growth & 29427 & 0.03 & 0.07 & -0.06 & -0.02 & 0.03 & 0.07 & 0.11 \\ 
  CRA 3yr growth & 29427 & 0.04 & 0.06 & -0.02 & 0.00 & 0.03 & 0.06 & 0.11 \\ 
  Establishments 3yr growth & 29427 & 0.01 & 0.01 & -0.00 & 0.00 & 0.01 & 0.02 & 0.03 \\ 
  Payroll 3yr growth & 29427 & 0.04 & 0.02 & 0.02 & 0.03 & 0.04 & 0.06 & 0.07 \\ 
  Acq. branch/presence & 29427 & 0.00 & 0.07 & 0.00 & 0.00 & 0.00 & 0.00 & 0.00 \\ 
  Branch owned 3plus years & 29427 & 0.97 & 0.16 & 1.00 & 1.00 & 1.00 & 1.00 & 1.00 \\ 
  Population density (1k km) & 29427 & 0.25 & 0.13 & 0.04 & 0.13 & 0.36 & 0.36 & 0.36 \\ 
  Low to Moderate Income Area & 29427 & 0.31 & 0.15 & 0.10 & 0.20 & 0.31 & 0.42 & 0.49 \\ 
   \hline
\end{tabular}
}

\vspace{1cm}
{\small Panel B: Small Banks}\\

    \resizebox{\textwidth}{!}{
\begin{tabular}{lrrrrrrrr}
  \hline
Variable & Obs & Mean & SD & P10 & P25 & P50 & P75 & P90 \\ 
  \hline
Deposits (mn) & 43015 & 90.40 & 754.24 & 11.00 & 23.00 & 42.00 & 78.00 & 141.00 \\ 
  Deposit 3yr growth & 43015 & 0.03 & 0.03 & -0.01 & 0.01 & 0.03 & 0.05 & 0.07 \\ 
  Mortgage 3yr growth & 43015 & 0.03 & 0.07 & -0.05 & -0.01 & 0.03 & 0.07 & 0.12 \\ 
  CRA 3yr growth & 43015 & 0.05 & 0.11 & -0.04 & -0.01 & 0.03 & 0.08 & 0.16 \\ 
  Establishments 3yr growth & 43015 & 0.01 & 0.01 & -0.01 & -0.00 & 0.01 & 0.02 & 0.02 \\ 
  Payroll 3yr growth & 43015 & 0.04 & 0.03 & 0.00 & 0.02 & 0.04 & 0.05 & 0.07 \\ 
  Acq. branch/presence & 43015 & 0.01 & 0.09 & 0.00 & 0.00 & 0.00 & 0.00 & 0.00 \\ 
  Branch owned 3plus years & 43015 & 0.91 & 0.29 & 1.00 & 1.00 & 1.00 & 1.00 & 1.00 \\ 
  Population density (1k km) & 43015 & 0.15 & 0.14 & 0.01 & 0.02 & 0.09 & 0.36 & 0.36 \\ 
  Low to Moderate Income Area & 43015 & 0.26 & 0.17 & 0.00 & 0.13 & 0.26 & 0.38 & 0.49 \\ 
   \hline
\end{tabular}
    }
    \caption{Descriptive Statistics for Branch-Level Data in 2019}
    \label{table4}
\end{table}


\clearpage
\begin{table}[]
    \centering
    \adddescription{
\small
This table presents the baseline linear probability model estimates from Equation (5a), where the dependent variable indicates whether a branch was closed in a given year. Columns 1 and 2 report results for the full sample, while columns 3–6 provide split-sample estimates for large banks (above 100 billion in assets) and small banks (below 100 billion in assets). The primary independent variable, DF per dollar, captures the predicted deposit franchise value. \sigtextable

    }
    \resizebox{\textwidth}{!}{
\begin{tabular}{@{\extracolsep{5pt}}lcccccc} 
\\[-1.8ex]\hline 
\hline \\[-1.8ex] 
 & \multicolumn{6}{c}{Closed=1} \\ 
\cline{2-7} 
 & \multicolumn{2}{c}{Full sample} & \multicolumn{2}{c}{Large banks} & \multicolumn{2}{c}{Small banks} \\ 
\\[-1.8ex] & (1) & (2) & (3) & (4) & (5) & (6)\\ 
\hline \\[-1.8ex] 
 DF per dollar & $-$0.6614$^{***}$ & $-$0.9551$^{***}$ & $-$1.0436$^{***}$ & $-$1.5931$^{***}$ & $-$0.4561$^{***}$ & $-$0.5008$^{***}$ \\ 
  & (0.1053) & (0.1477) & (0.1956) & (0.2268) & (0.0786) & (0.0832) \\ 
  log(Deposits (mn)) & $-$0.0241$^{***}$ & $-$0.0243$^{***}$ & $-$0.0309$^{***}$ & $-$0.0304$^{***}$ & $-$0.0198$^{***}$ & $-$0.0200$^{***}$ \\ 
  & (0.0014) & (0.0014) & (0.0021) & (0.0020) & (0.0009) & (0.0009) \\ 
  Acq. branch/presence & 0.0617$^{***}$ & 0.0601$^{***}$ & 0.0762$^{***}$ & 0.0731$^{***}$ & 0.0491$^{***}$ & 0.0481$^{***}$ \\ 
  & (0.0085) & (0.0083) & (0.0162) & (0.0168) & (0.0089) & (0.0086) \\ 
  Branch owned 3plus years & $-$0.0077$^{***}$ & $-$0.0089$^{***}$ & $-$0.0084 & $-$0.0087 & $-$0.0089$^{***}$ & $-$0.0098$^{***}$ \\ 
  & (0.0028) & (0.0027) & (0.0101) & (0.0092) & (0.0022) & (0.0021) \\ 
  Deposit 3yr growth & $-$0.0213$^{**}$ &  & $-$0.0271$^{**}$ &  & $-$0.0124 &  \\ 
  & (0.0090) &  & (0.0121) &  & (0.0122) &  \\ 
  Mortgage 3yr growth & $-$0.0081 &  & $-$0.0113 &  & $-$0.0045 &  \\ 
  & (0.0056) &  & (0.0125) &  & (0.0049) &  \\ 
  CRA 3yr growth & $-$0.0014 &  & 0.0086 &  & $-$0.0028 &  \\ 
  & (0.0035) &  & (0.0123) &  & (0.0031) &  \\ 
  Establishments 3yr growth & $-$0.2841$^{***}$ &  & $-$0.5675$^{***}$ &  & $-$0.0964$^{***}$ &  \\ 
  & (0.0600) &  & (0.1118) &  & (0.0362) &  \\ 
  Payroll 3yr growth & $-$0.0091 &  & $-$0.0394 &  & 0.0091 &  \\ 
  & (0.0140) &  & (0.0310) &  & (0.0126) &  \\ 
  Low to Moderate Income Area & $-$0.0102$^{***}$ &  & $-$0.0203$^{***}$ &  & $-$0.0023 &  \\ 
  & (0.0026) &  & (0.0049) &  & (0.0021) &  \\ 
 \hline \\[-1.8ex] 
Bank$\times$Year FE & Y & Y & Y & Y & Y & Y \\ 
State$\times$Year FE & Y & - & Y & - & Y & - \\ 
County$\times$Year FE & - & Y & - & Y & - & Y \\ 
Observations & 655,842 & 655,842 & 264,807 & 264,807 & 391,035 & 391,035 \\ 
R$^{2}$ & 0.0925 & 0.1248 & 0.0527 & 0.1220 & 0.1406 & 0.1928 \\ 
\hline 
\hline \\[-1.8ex] 
\textit{Note:}  & \multicolumn{6}{r}{$^{*}$p$<$0.1; $^{**}$p$<$0.05; $^{***}$p$<$0.01} \\ 
\end{tabular} 
    }
    \caption{Baseline Closure Model}
    \label{table5}
\end{table}




\clearpage
\begin{table}[]
    \centering
    \adddescription{
\small
This table presents reduced-form estimates of branch closure probability using Equation (5b). Panel A reports results for the full sample, while Panel B splits the sample by bank size (large vs. small banks). The dependent variable is an indicator for branch closure (Closed = 1). A sophisticated zip code is defined as zip codes with above median income, education, and stock market participation. The coefficients of the control variables are suppressed.  \sigtextable

    }
    \vspace{0.5cm}
{\small Panel A: Full Sample}\\
    \resizebox{0.5\textwidth}{!}{
\begin{tabular}{@{\extracolsep{5pt}}lcccc} 
\\[-1.8ex]\hline 
\hline \\[-1.8ex] 
 & \multicolumn{4}{c}{Closed=1} \\ 
\cline{2-5} 
\\[-1.8ex] & (1) & (2) & (3) & (4)\\ 
\hline \\[-1.8ex] 
 College frac & 0.007 & 0.017$^{***}$ &  &  \\ 
  & (0.005) & (0.005) &  &  \\ 
  log(Income) & $-$0.003$^{***}$ & $-$0.004$^{***}$ &  &  \\ 
  & (0.001) & (0.001) &  &  \\ 
  Stock market frac & 0.037$^{***}$ & 0.029$^{***}$ &  &  \\ 
  & (0.006) & (0.006) &  &  \\ 
  Age Q1-Q2 & 0.002$^{*}$ & 0.001 & 0.001 & 0.001 \\ 
  & (0.001) & (0.001) & (0.001) & (0.001) \\ 
  Age Q2-Q3 & 0.0003 & 0.0002 & 0.001 & 0.002 \\ 
  & (0.001) & (0.001) & (0.001) & (0.001) \\ 
  Age > Q3 & $-$0.00005 & 0.001 & 0.003$^{*}$ & 0.004$^{**}$ \\ 
  & (0.001) & (0.001) & (0.002) & (0.002) \\ 
  Sophisticated zipcode &  &  & 0.003$^{***}$ & 0.003$^{***}$ \\ 
  &  &  & (0.001) & (0.001) \\ 
  County HHI & 0.0004 &  & 0.001 &  \\ 
  & (0.003) &  & (0.003) &  \\ 
  Population density & 0.002 &  & 0.010$^{**}$ &  \\ 
  & (0.004) &  & (0.005) &  \\ 
 \hline \\[-1.8ex] 
Bank$\times$Year FE & Y & Y & Y & Y \\ 
State$\times$Year FE & Y & - & Y & - \\ 
County$\times$Year FE & - & Y & - & Y \\ 
Controls & Y & Y & Y & Y \\ 
Observations & 655,842 & 655,842 & 655,842 & 655,842 \\ 
R$^{2}$ & 0.093 & 0.125 & 0.092 & 0.124 \\ 
\hline 
\hline \\[-1.8ex] 
\textit{Note:}  & \multicolumn{4}{r}{$^{*}$p$<$0.1; $^{**}$p$<$0.05; $^{***}$p$<$0.01} \\ 
\end{tabular} 
    }\\

\vspace{0.5cm}
{\small Panel B: By Size}\\
        \resizebox{0.65\textwidth}{!}{
        \begin{tabular}{@{\extracolsep{5pt}}lcccccccc} 
\\[-1.8ex]\hline 
\hline \\[-1.8ex] 
 & \multicolumn{8}{c}{Closed=1} \\ 
\cline{2-9} 
 & \multicolumn{2}{c}{Large banks} & \multicolumn{2}{c}{Small banks} & \multicolumn{2}{c}{Large banks} & \multicolumn{2}{c}{Small banks} \\ 
\\[-1.8ex] & (1) & (2) & (3) & (4) & (5) & (6) & (7) & (8)\\ 
\hline \\[-1.8ex] 
 College frac & 0.014 & 0.024$^{**}$ & 0.006 & 0.015$^{***}$ &  &  &  &  \\ 
  & (0.009) & (0.010) & (0.004) & (0.005) &  &  &  &  \\ 
  log(Income) & $-$0.007$^{***}$ & $-$0.008$^{***}$ & $-$0.001$^{*}$ & $-$0.003$^{***}$ &  &  &  &  \\ 
  & (0.001) & (0.002) & (0.001) & (0.001) &  &  &  &  \\ 
  Stock market frac & 0.057$^{***}$ & 0.052$^{***}$ & 0.014$^{**}$ & 0.008 &  &  &  &  \\ 
  & (0.009) & (0.009) & (0.005) & (0.006) &  &  &  &  \\ 
  Age Q1-Q2 & 0.004$^{***}$ & 0.004$^{***}$ & $-$0.001 & $-$0.001 & 0.003$^{***}$ & 0.004$^{**}$ & $-$0.001 & $-$0.001 \\ 
  & (0.001) & (0.001) & (0.001) & (0.001) & (0.001) & (0.001) & (0.001) & (0.001) \\ 
  Age Q2-Q3 & 0.004$^{**}$ & 0.002$^{*}$ & $-$0.002$^{***}$ & $-$0.002$^{**}$ & 0.005$^{***}$ & 0.005$^{***}$ & $-$0.002$^{**}$ & $-$0.002$^{*}$ \\ 
  & (0.002) & (0.001) & (0.001) & (0.001) & (0.002) & (0.002) & (0.001) & (0.001) \\ 
  Age > Q3 & 0.004 & 0.003 & $-$0.003$^{**}$ & $-$0.002 & 0.009$^{***}$ & 0.010$^{***}$ & $-$0.002 & $-$0.001 \\ 
  & (0.002) & (0.002) & (0.001) & (0.001) & (0.002) & (0.002) & (0.001) & (0.001) \\ 
  Sophisticated zipcode &  &  &  &  & 0.006$^{***}$ & 0.007$^{***}$ & 0.001$^{**}$ & 0.001 \\ 
  &  &  &  &  & (0.001) & (0.001) & (0.001) & (0.001) \\ 
  County HHI & 0.009 &  & $-$0.004$^{*}$ &  & 0.012$^{**}$ &  & $-$0.004$^{*}$ &  \\ 
  & (0.005) &  & (0.002) &  & (0.005) &  & (0.002) &  \\ 
  Population density & $-$0.003 &  & 0.014$^{***}$ &  & 0.007 &  & 0.018$^{***}$ &  \\ 
  & (0.008) &  & (0.005) &  & (0.009) &  & (0.005) &  \\ 
 \hline \\[-1.8ex] 
Bank$\times$Year FE & Y & Y & Y & Y & Y & Y & Y & Y \\ 
State$\times$Year FE & Y & - & Y & - & Y & - & Y & - \\ 
County$\times$Year FE & - & Y & - & Y & - & Y & - & Y \\ 
Controls & Y & Y & Y & Y & Y & Y & Y & Y \\ 
Observations & 264,807 & 264,807 & 391,035 & 391,035 & 264,807 & 264,807 & 391,035 & 391,035 \\ 
R$^{2}$ & 0.053 & 0.122 & 0.141 & 0.193 & 0.052 & 0.121 & 0.141 & 0.193 \\
\hline 
\hline \\[-1.8ex] 
\textit{Note:}  & \multicolumn{8}{r}{$^{*}$p$<$0.1; $^{**}$p$<$0.05; $^{***}$p$<$0.01} \\ 
\end{tabular} 
        }
    \caption{Reduced Form Pooled Model}
    \label{table6}
\end{table}



\clearpage
\begin{table}[]
    \centering
    \adddescription{
\small
This table reports year-by-year estimates of Equation (5a), analyzing the relationship between the deposit franchise value (DF per dollar) and branch closures. Panels A and B focus on large banks, while Panels C and D focus on small banks. Panels A and C include state × year fixed effects, and Panels B and D incorporate county × year fixed effects. The dependent variable in all panels is an indicator for branch closure (Closed = 1). The coefficients of the control variables are suppressed. \sigtextable
    }

    \vspace{0.25cm}
{\small Panel A: Large Banks}\\
    \resizebox{0.8\textwidth}{!}{
\begin{tabular}{@{\extracolsep{5pt}}lccccccccc} 
\\[-1.8ex]\hline 
\hline \\[-1.8ex] 
 & \multicolumn{9}{c}{Closed=1} \\ 
\cline{2-10} 
 & 2015 & 2016 & 2017 & 2018 & 2019 & 2020 & 2021 & 2022 & 2023 \\ 
\\[-1.8ex] & (1) & (2) & (3) & (4) & (5) & (6) & (7) & (8) & (9)\\ 
\hline \\[-1.8ex] 
 DF per dollar & $-$0.466 & $-$0.451 & $-$0.594$^{*}$ & $-$0.827$^{*}$ & $-$0.803$^{**}$ & $-$2.532$^{***}$ & $-$1.182$^{***}$ & $-$1.711$^{***}$ & $-$1.266$^{***}$ \\ 
  & (0.350) & (0.380) & (0.312) & (0.409) & (0.314) & (0.656) & (0.285) & (0.512) & (0.283) \\ 
 \hline \\[-1.8ex] 
Bank$\times$Year FE & Y & Y & Y & Y & Y & Y & Y & Y & Y \\ 
State$\times$Year FE & Y & Y & Y & Y & Y & Y & Y & Y & Y \\ 
Controls & Y & Y & Y & Y & Y & Y & Y & Y & Y \\ 
Observations & 32,475 & 32,175 & 31,686 & 30,505 & 29,427 & 29,295 & 27,206 & 26,357 & 25,681 \\ 
R$^{2}$ & 0.041 & 0.051 & 0.040 & 0.043 & 0.059 & 0.069 & 0.056 & 0.042 & 0.026 \\ 
\hline 
\hline \\[-1.8ex] 
\textit{Note:}  & \multicolumn{9}{r}{$^{*}$p$<$0.1; $^{**}$p$<$0.05; $^{***}$p$<$0.01} \\ 
\end{tabular} 
    }


\vspace{0.25cm}
{\small Panel B: Large Banks - County FE}\\
    \resizebox{0.8\textwidth}{!}{
    \begin{tabular}{@{\extracolsep{5pt}}lccccccccc} 
\\[-1.8ex]\hline 
\hline \\[-1.8ex] 
 & \multicolumn{9}{c}{Closed=1} \\ 
\cline{2-10} 
 & 2015 & 2016 & 2017 & 2018 & 2019 & 2020 & 2021 & 2022 & 2023 \\ 
\\[-1.8ex] & (1) & (2) & (3) & (4) & (5) & (6) & (7) & (8) & (9)\\ 
\hline \\[-1.8ex] 
 DF per dollar & $-$0.640 & $-$0.968$^{**}$ & $-$1.200$^{***}$ & $-$1.347$^{**}$ & $-$1.542$^{***}$ & $-$4.378$^{***}$ & $-$1.636$^{***}$ & $-$2.038$^{***}$ & $-$1.234$^{***}$ \\ 
  & (0.439) & (0.361) & (0.393) & (0.496) & (0.362) & (0.803) & (0.437) & (0.417) & (0.330) \\ 
 \hline \\[-1.8ex] 
Bank$\times$Year FE & Y & Y & Y & Y & Y & Y & Y & Y & Y \\ 
County$\times$Year FE & Y & Y & Y & Y & Y & Y & Y & Y & Y \\ 
Controls & Y & Y & Y & Y & Y & Y & Y & Y & Y \\ 
Observations & 32,475 & 32,175 & 31,686 & 30,505 & 29,427 & 29,295 & 27,206 & 26,357 & 25,681 \\ 
R$^{2}$ & 0.096 & 0.127 & 0.118 & 0.114 & 0.133 & 0.135 & 0.120 & 0.113 & 0.092 \\  
\hline 
\hline \\[-1.8ex] 
\textit{Note:}  & \multicolumn{9}{r}{$^{*}$p$<$0.1; $^{**}$p$<$0.05; $^{***}$p$<$0.01} \\ 
\end{tabular} 
}

\vspace{0.25cm}
{\small Panel C: Small Banks}\\
    \resizebox{0.8\textwidth}{!}{
\begin{tabular}{@{\extracolsep{5pt}}lccccccccc} 
\\[-1.8ex]\hline 
\hline \\[-1.8ex] 
 & \multicolumn{9}{c}{Closed=1} \\ 
\cline{2-10} 
 & 2015 & 2016 & 2017 & 2018 & 2019 & 2020 & 2021 & 2022 & 2023 \\ 
\\[-1.8ex] & (1) & (2) & (3) & (4) & (5) & (6) & (7) & (8) & (9)\\ 
\hline \\[-1.8ex] 
 DF per dollar & $-$0.356 & $-$0.202 & $-$0.728$^{***}$ & $-$0.196 & $-$0.384$^{**}$ & $-$0.865$^{***}$ & $-$0.196 & $-$0.427$^{**}$ & $-$0.505$^{***}$ \\ 
  & (0.240) & (0.180) & (0.208) & (0.243) & (0.185) & (0.219) & (0.188) & (0.204) & (0.150) \\ 
 \hline \\[-1.8ex] 
Bank$\times$Year FE & Y & Y & Y & Y & Y & Y & Y & Y & Y \\ 
State$\times$Year FE & Y & Y & Y & Y & Y & Y & Y & Y & Y \\ 
Controls & Y & Y & Y & Y & Y & Y & Y & Y & Y \\ 
Observations & 50,028 & 48,318 & 47,107 & 45,861 & 43,013 & 42,674 & 39,149 & 37,974 & 36,911 \\ 
R$^{2}$ & 0.131 & 0.195 & 0.142 & 0.122 & 0.148 & 0.133 & 0.145 & 0.107 & 0.116 \\ 
\hline 
\hline \\[-1.8ex] 
\textit{Note:}  & \multicolumn{9}{r}{$^{*}$p$<$0.1; $^{**}$p$<$0.05; $^{***}$p$<$0.01} \\ 
\end{tabular}
}

\vspace{0.25cm}
{\small Panel D: Small Banks - County FE}\\
    \resizebox{0.8\textwidth}{!}{
\begin{tabular}{@{\extracolsep{5pt}}lccccccccc} 
\\[-1.8ex]\hline 
\hline \\[-1.8ex] 
 & \multicolumn{9}{c}{Closed=1} \\ 
\cline{2-10} 
 & 2015 & 2016 & 2017 & 2018 & 2019 & 2020 & 2021 & 2022 & 2023 \\ 
\\[-1.8ex] & (1) & (2) & (3) & (4) & (5) & (6) & (7) & (8) & (9)\\ 
\hline \\[-1.8ex] 
 DF per dollar & $-$0.569$^{**}$ & $-$0.245 & $-$0.756$^{***}$ & $-$0.350 & $-$0.509$^{**}$ & $-$1.495$^{***}$ & $-$0.098 & $-$0.219 & $-$0.418$^{**}$ \\ 
  & (0.269) & (0.223) & (0.232) & (0.281) & (0.209) & (0.303) & (0.228) & (0.238) & (0.184) \\ 
 \hline \\[-1.8ex] 
Bank$\times$Year FE & Y & Y & Y & Y & Y & Y & Y & Y & Y \\ 
County$\times$Year FE & Y & Y & Y & Y & Y & Y & Y & Y & Y \\ 
Controls & Y & Y & Y & Y & Y & Y & Y & Y & Y \\ 
Observations & 50,028 & 48,318 & 47,107 & 45,861 & 43,013 & 42,674 & 39,149 & 37,974 & 36,911 \\ 
R$^{2}$ & 0.183 & 0.236 & 0.193 & 0.168 & 0.199 & 0.184 & 0.207 & 0.168 & 0.176 \\ 
\hline 
\hline \\[-1.8ex] 
\textit{Note:}  & \multicolumn{9}{r}{$^{*}$p$<$0.1; $^{**}$p$<$0.05; $^{***}$p$<$0.01} \\ 
\end{tabular} 
}

    \caption{By Year}
    \label{table7}
\end{table}




\clearpage
\begin{table}[]
    \centering
    \adddescription{
\small
This table presents estimates from Equation (5c), comparing the relative importance of deposit franchise value (DF per dollar) and branch usage metrics (log of the number of visitors and travel distance metrics) in explaining branch closures. Panel A includes results for the full sample, while Panel B splits the sample into large and small banks. The coefficients of the control variables are suppressed. \sigtextable

    }
    \vspace{0.5cm}
{\small Panel A}\\
    \resizebox{0.7\textwidth}{!}{
\begin{tabular}{@{\extracolsep{5pt}}lccc} 
\\[-1.8ex]\hline 
\hline \\[-1.8ex] 
 & \multicolumn{3}{c}{Closed=1} \\ 
\cline{2-4} 
 & \multicolumn{3}{c}{Full sample} \\ 
\\[-1.8ex] & (1) & (2) & (3)\\ 
\hline \\[-1.8ex] 
  DF per dollar & $-$0.8218$^{***}$ & $-$0.7819$^{***}$ & $-$0.4496$^{***}$ \\ 
  & (0.1727) & (0.1595) & (0.1547) \\ 
  log(No of visitors/1mn deposits) &  & 0.0011 & 0.0008 \\ 
  &  & (0.0025) & (0.0026) \\ 
  log(Distance km) &  & 0.0049$^{***}$ & 0.0070$^{***}$ \\ 
  &  & (0.0017) & (0.0020) \\ 
 \hline \\[-1.8ex] 
Bank$\times$Year FE & Y & Y & Y \\ 
State$\times$Year FE & Y & Y & - \\ 
County$\times$Year FE & - & - & Y \\ 
Controls & Y & Y & Y \\ 
Observations & 125,404 & 94,240 & 94,240 \\ 
R$^{2}$ & 0.0646 & 0.0587 & 0.0967 \\ 
\hline 
\hline \\[-1.8ex] 
\textit{Note:}  & \multicolumn{3}{r}{$^{*}$p$<$0.1; $^{**}$p$<$0.05; $^{***}$p$<$0.01} \\ 
\end{tabular} 
    }

    \vspace{0.5cm}
{\small Panel B}\\
        \resizebox{\textwidth}{!}{
        \begin{tabular}{@{\extracolsep{5pt}}lcccccc} 
\\[-1.8ex]\hline 
\hline \\[-1.8ex] 
 & \multicolumn{6}{c}{Closed=1} \\ 
\cline{2-7} 
 & \multicolumn{3}{c}{Large banks} & \multicolumn{3}{c}{Small banks} \\ 
\\[-1.8ex] & (1) & (2) & (3) & (4) & (5) & (6)\\ 
\hline \\[-1.8ex] 
 DF per dollar & $-$1.4249$^{***}$ & $-$1.3388$^{***}$ & $-$0.9206$^{***}$ & $-$0.4420$^{***}$ & $-$0.3120$^{***}$ & $-$0.0464 \\ 
  & (0.2870) & (0.2352) & (0.2017) & (0.1308) & (0.0937) & (0.1444) \\ 
  log(No of visitors/1mn deposits) &  & 0.0017 & 0.0015 &  & 0.0006 & $-$0.0001 \\ 
  &  & (0.0048) & (0.0050) &  & (0.0009) & (0.0010) \\ 
  log(Distance km) &  & 0.0080$^{***}$ & 0.0127$^{***}$ &  & 0.0027$^{**}$ & 0.0042$^{***}$ \\ 
  &  & (0.0027) & (0.0033) &  & (0.0012) & (0.0015) \\ 
 \hline \\[-1.8ex] 
Bank$\times$Year FE & Y & Y & Y & Y & Y & Y \\ 
State$\times$Year FE & Y & Y & - & Y & Y & - \\ 
County$\times$Year FE & - & - & Y & - & - & Y \\ 
Controls & Y & Y & Y & Y & Y & Y \\ 
Observations & 51,354 & 44,856 & 44,856 & 74,050 & 49,384 & 49,384 \\ 
R$^{2}$ & 0.0318 & 0.0320 & 0.0951 & 0.1122 & 0.1130 & 0.1810 \\ 
\hline 
\hline \\[-1.8ex] 
\textit{Note:}  & \multicolumn{6}{r}{$^{*}$p$<$0.1; $^{**}$p$<$0.05; $^{***}$p$<$0.01} \\ 
\end{tabular} 
        }
    \caption{With Usage Controls}
    \label{table8}
\end{table}



\end{document}

