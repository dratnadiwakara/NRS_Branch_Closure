
\section*{Introduction}

This paper examines the drivers of bank branch opening and closure decisions during the recent period of industry restructuring. We show that measures of local financial sophistication strongly predict both outcomes: depositors in these areas are more rate-sensitive, which reduces banks’ pricing power, and they rely less on physical access, which makes branch closures less risky in terms of deposit retention. The findings highlight the different incentives facing incumbent banks versus potential entrants. Incumbents are more likely to close branches where customers are both hard to profit from and less dependent on in-person services. Potential entrants, by contrast, open branches in deposit-rich areas but mostly  just in those with rate-insensitive depositors, as such customers are difficult to attract away from incumbents despite their profitability.

Bank branches grew in aggregate until 2010, with the rate of branch openings exceeding that of closings by a factor of two to three. This net branch growth occurred even as the banking system consolidated. Total branches and offices, however, peaked in 2010 and then began to decline. The rate of decline accelerated around 2015 and further intensified following the COVID-19 pandemic (Figure \ref{fig:no_of_banks_branches}). These recent patterns stand in sharp contrast to the earlier period, when openings consistently outpaced closings (Figure \ref{fig:no_of_banks_branches}, Panel B). Before the Global Financial Crisis (GFC), a forty-year period of banking-industry restructuring unfolded as large banks expanded into new markets by acquiring smaller incumbent banks and by opening new branches. This expansion was fostered by the deregulation of restrictions on branching and interstate banking (\citet{Schneider2025}). This restructuring sharply reduced the number of banks (due to M\&A), but not the number of branches. The purpose of this first wave of M\&A was fundamentally to extend large banks’ physical reach (\citet{kroszner2014regulation}). By contrast, restructuring since 2010 has reduced both the number of banks and the number of branches.

% We argue that technology is the primary driver of branch restructuring today. Internet and mobile banking allow customers to access deposits without physical proximity and make it easier to move funds in response to shocks (as in the SVB episode) or to changes in market interest rates. These advances diminish the role of dense branch networks in two ways: they reduce banks’ deposit pricing power by making customers more sensitive to interest rates, and they lower the convenience value of physical proximity by allowing customers to bank remotely. As a result, banks can more easily close branches without losing deposits when local customers rely less on in-person access. Although technology broadly pressures banks to shrink their physical footprint, our empirical strategy focuses on the fact that financially sophisticated customers adopt these technologies more quickly. In such areas, not only do higher interest rate sensitivity reduce banks’ pricing power, but lower branch usage also makes it easier to close locations without losing deposits. This recent wave of restructuring contrasts with the earlier era, when deregulation was the dominant force. Our paper thus explains the contemporary decline in U.S. branch networks and contributes to understanding the broader restructuring of the banking industry.

We argue that technology is the primary driver of the current wave of branch restructuring. Advances in internet and mobile banking, along with new payment systems, have fundamentally reduced the role of physical proximity and reduced frictions associated with moving funds across accounts or institutions in retail banking. What matters for our analysis is that adoption of these technologies varies across customer groups: financially sophisticated households embrace them earlier and more intensively than others, resulting in less pricing power for banks and making branches easier to close without losing deposits. This heterogeneity explains why branch closures concentrate in such areas. We show that, unlike in the earlier era when deregulation helped expand banks’ physical reach, today’s decline in branch networks arises from uneven adoption of digital banking technologies.

To quantify deposit pricing power, we focus on deposit-$\beta$, which captures how strongly a branch’s depositors respond to shifts in market rates. As shown in \citet{drechsler2017deposits}, this measure provides a sufficient statistic for local market power, reflecting not only market concentration but also depositors’ financial sophistication, attentiveness, and willingness to switch banks. We extend their concept by first estimating bank-level sensitivity from the demographics of the areas in which the bank operates, and then mapping those demographics to individual branches to generate branch-level predictions of $\beta$. This approach allows us to recover heterogeneity in depositor rate sensitivity across locations within the same bank and thereby assess how differences in depositor behavior shape branch restructuring decisions. Put differently, our regression models exploit within-bank variation in depositor sensitivity while absorbing overall bank- and time-level factors with fixed effects. 

For incumbent banks, we observe where they close branches.  For potential entrants, we construct a set of candidate zip codes for each bank. This set includes all zip codes within CBSAs where the bank already operates at least one branch, as well as those in CBSAs where it subsequently opens a new branch. As in the closure analysis, we examine opening decisions within bank-year. For each candidate zip code, we assign the same Deposit-$\beta$ measure described above, which in this context reflects the expected price sensitivity of depositors if the bank were to enter by opening a new branch. Thus, while incumbents face realized depositor behavior at existing branches, entrants base their decisions on expected depositor sensitivity in potential markets.

We use our indirect approach to capture depositor behavior at the branch level, since branch-level measures of deposit sensitivity are not directly available. Recent studies have used deposit quotes from RateWatch, but these data cover only a subset of branches (“rate-setting” branches) and do not capture the underlying deposit mix. As a result, they provide little insight into how depositors actually reallocate funds across products—for example, shifting balances from checking to time deposits when the spread is large. Moreover, as emphasized by \citet{begenau2023uniform}, many large banks follow uniform pricing, posting the same rate across broad geographic areas. Our branch-based Deposit-$\beta$ addresses these limitations by capturing depositor sensitivity at the branch level—that is, the profitability a bank can extract from its local depositor base—regardless of whether it sets rates uniformly or differentially. If a bank varies rates across branches, high-$\beta$ branches will exhibit lower spreads (and thus lower profits) than low-$\beta$ branches. If, instead, the bank uses uniform pricing, high-$\beta$ branches will experience larger deposit outflows when rates rise, either because local competitors post more attractive rates or because rate-sensitive customers shift funds into substitutes such as money market funds. In such cases, banks may attempt to retain deposits in high-$\beta$ markets through ancillary services, advertising, or costly promotions.\footnote{This kind of behavior was used widely during the late 1970s to alleviate disintermediation from binding interest rate caps on deposits. Some banks went so far as offering expensive gifts such as televisions to retain deposits.} In either case, Deposit-$\beta$ provides a consistent ranking: low-$\beta$ branches generate more value for the bank than high-$\beta$ ones.


Our location-based Deposit-$\beta$ strongly predicts both branch closures and openings, but operates differently across the two margins. Incumbent banks are less likely to close branches in areas with price-insensitive customers (low $\beta$), as such customers are especially profitable and easy to retain. In contrast, potential entrants are less likely to open branches in these same areas, since low rate sensitivity makes depositors difficult to attract away from incumbents. This asymmetry helps explain why the industry has been—and will likely continue to be—reshaped as usage of mobile payments and digital banking expands.
 

The relationship between Deposit-$\beta$ and branch restructuring—both closures and openings—is stronger for large banks than for small ones. Large banks respond more systematically when deciding which branches to close or where to open new ones, whereas small banks exhibit weaker and less consistent patterns. These differences likely reflect underlying variation in customer composition and strategic focus: large banks tend to attract more financially sophisticated customers—those with higher income, education, and digital adoption—who are more interest-rate sensitive. This sorting is consistent with evidence from \citet{narayanan2024depositor}, who use cell phone data to show that large-bank branches are disproportionately visited by customers from more affluent and educated areas. \citet{kundu2024diverging}, also studying the largest banks, find larger declines in branches among those paying high deposit rates. As a result, large banks are more active in adjusting their branch networks in response to changes in the profitability of their deposit base. \citet{d2023deposit} find that large bank customers value the broader set of financial services they provide and are willing to accept lower deposit rates to stay with the bank. We find, however, that the large banks increase their rates more aggressively relative to small banks when market rates rise, a finding that is consistent with \citet{drechsler2021banking} who document a positive relationship between bank size and deposit betas.\footnote{Notably, the sample period in \citet{d2023deposit} ends before the COVID-19 pandemic, when these effects become most pronounced.} 

The COVID-19 pandemic further amplified these patterns. It served as a “teachable moment,” when many households discovered that digital banking could substitute for branch proximity, and remote work arrangements reinforced online rather than in-person interactions. As a result, the effect of Deposit-$\beta$ on closures became even stronger, particularly among large banks operating in financially sophisticated markets.

%Variation in the value of the DF also matters more for large banks’ branching decisions than it does for small ones. The starkest difference appears in the years following the Pandemic. The Pandemic year itself came with the largest overall decline in branches, with almost 8\% of large-bank branches being closed (Figure \ref{fig:pct_branches_closed}). For large banks, during the 2020-2023 regime a one-standard-deviation decrease in the DF raised the probability of a branch closure by about 1 percentage point (or about one-quarter of the mean closure rate during these years), with a much smaller impact on smaller banks. This is likely due to the difference that reflects the sorting of customers within localities across banks based on size, whereby the more sophisticated customers gravitate toward large banks. Consistent with this idea, \citet{narayanan2024depositor} use cell phone data to link customers' characteristics to the branches that they visit. This paper shows that large banks attract customers living in areas with higher levels of income, education, and other measures of financial sophistication. Similarly, \citet{d2023deposit} argue that large banks cater to different clientele who value more complex services that go beyond deposits, compared to smaller banks.

% As a complementary exercise, we replace Deposit-$\beta$ with its demographic determinants in the regressions. Consistent with the interpretation above, closures are more common in areas with highly educated residents and those more exposed to the stock market—individuals more likely to adopt digital banking technologies and more interest-rate sensitive \citep{haendler2022keeping,jiang2022bank,koont2023digital}. We also find that depositors in these areas visit branches less frequently and travel greater distances when they do, reducing the strategic value of maintaining a physical presence. Finally, the marginal impact of deposit rate sensitivity on closures increases sharply after the COVID-19 pandemic, a ‘teachable moment’ when many people learned that technology could substitute for physical proximity and remote work arrangements reinforced online rather than in-person interactions.

Having established that depositor sensitivity to interest rates is a key predictor of restructuring, we next turn to the role of physical access. We measure the amenity value of branches using cell phone mobility data, which captures both how often customers visit branches and how far they travel when they do. Financial sophistication strongly predicts these patterns: households in sophisticated areas visit less often and travel farther, consistent with substitution toward digital banking. These usage measures help predict branch closures but are less informative for openings. Including them in our regressions attenuates the effect of deposit rate sensitivity somewhat because usage and rate sensitivity are both shaped by demographics. This reinforces our core argument: in sophisticated areas, technology both raises rate sensitivity and lowers the value of proximity, reducing the value of deposits to banks.

Since both deposit rate sensitivity and branch usage are strongly predicted by local demographics, we estimate a reduced-form specification that replaces these measures with their demographic determinants. In particular, we focus on education and stock market participation, which capture the degree of financial sophistication in a branch’s local market. These characteristics predict both higher deposit-$\beta$ and lower branch usage, consistent with evidence that sophisticated households adopt digital banking technologies more readily \citep{haendler2022keeping,jiang2022bank,koont2023digital}. Closures therefore concentrate in areas with these demographics, where banks face less pricing power and weaker demand for physical access.

Compared to deposits, we find at best only weak evidence that lending variables can explain branch restructuring. This is surprising given that much of the prior banking literature has demonstrated the importance of physical distance between bankers and borrowers (e.g., \citet{petersen2002does,berger2005does}). But technology has significantly reduced the importance of distance. Until recently, most business sales were conducted largely using cash, which necessitated close physical proximity to a bank branch, if nothing else as a means to safeguard cash receipts. Today, businesses accept an increasing fraction of their sales from electronic payments (as opposed to cash), reducing the need for local branches for security-related purposes. Beyond that, the information environment has also changed significantly, again because payment flows are now dominated by electronic means.\footnote{Penetration of phone-based payments technologies has been faster in many parts of the world than in the US. As a result, for example, the number of branches per capita has fallen twice as fast in the EU as in the US. In the Netherlands, to take an extreme case, the number of bank branches has fallen by 85\% (https://fred.stlouisfed.org/series/DDAI02NLA643NWDB).
}  As such, physical proximity no longer matters much for information production.\footnote{\citet{buchak2018fintech} focus on the increasing market share of fintech lenders in the mortgage space and \citet{gopal2022rise}, who report a high and growing share of lending to small businesses by non-banks. For a review of the growing role of Fintech lenders generally, see \citet{berg2022fintech}} For these reasons, we argue that the demand for lending does not help explain branch restructuring because bank location matters little for effective credit provision by banks (or other lenders, such as Fintechs).\footnote{Note that we are not claiming that bank relationships no longer matter; nor are we arguing that bank-borrower lending relationships are no longer sticky, as has been documented across many studies. We are instead arguing that the physical distance between banks (or bankers) and borrowers matters much less than in the past.
} 



We contribute to a nascent literature studying drivers of branch closures. \citet{keil2024demise} study the de-branching regime as we do, but that paper emphasizes variation across banks in technology adoption. Similarly, both \citet{haendler2022keeping} and \citet{jiang2022bank} show that banks offering customers online or phone-based access are more likely to close branches. \citet{koont2023digital} argues that bank investment in digital technologies leads to branchless competition. Our approach takes banks’ investments in technology as given (by absorbing bank-time effects) and instead focuses on how branch-level variation in the customer base explains closures. As such, our work is complementary to theirs. The core difference is that our effects depend on variation in customer adoption of technologies – that is, customer demand for non-branch access to their funds - which lowers the value of bank branches. We shut down variation in the supply of technology by controlling for bank-year and county-year fixed effects. 

Evidence from other jurisdictions supports the view that tech-savvy customers lower the value of a physical branch location.  \citet{yuan2023your} show that younger customers (Gen Z) who switch to similar banking products offered by Fintech firms leads to a rise in both the number and share of branch closures in China. \citet{zimin2022profitability} show that higher levels of financial digital literacy are related to de-branching in Russia.  Consistent with our arguments, \cite{benmelech2023bank} and \cite{koont2024destabilizing} argue that digital banking reduces depositor stickiness, both with regard to changes in market interest rates and also around the SVB crisis.

Our paper complements \citet{kumarjames2025lowinterest}.  While our paper focuses on understanding which, among a given bank’s existing branch network, a bank chooses to close, their focus is on the role of the deposit franchise in explaining closure rates across banks.  Like \citet{sarto2023secular}, they emphasize that banks lose value in the low interest rate environment after the GFC because deposit spreads over market rates compress.  Our empirical design complements theirs by absorbing these macro-level effects and instead focusing on how variation in the customer base influences branch-level closure decisions.

In contrast to our results, where we find little evidence that local lending affects branch closures, \citet{cespedes2024branching}, who study an earlier period (2011-2017), find that shadow bank entry affects branch closures by lowering local residual loan demand, especially at banks with a high cost of violating the Community Reinvestment Act (CRA). Their evidence, along with ours, suggests that the impact of lending may have changed in recent years, as information technology has become increasingly important. 

To the best of our knowledge, this is the first paper to study gross effects -- branch closures and openings separately, rather than focusing only on the \textit{net} change in branch counts within a locality. This distinction matters because incumbents and potential entrants face opposing incentives. Incumbents find areas with low depositor interest-rate sensitivity attractive, as they can extract rents from rate-insensitive customers by pricing deposits below market. Potential entrants, however, avoid these same areas because sticky customers are difficult to draw away from incumbents. Our results therefore highlight that Deposit-$\beta$ is a more informative measure of local market power than the traditional HHI. Low $\beta$ strengthens market power through two reinforcing channels: (i) incumbents exploit rate-insensitive depositors, as emphasized by \citet{drechsler2017deposits}, and (ii) potential entrants face higher barriers because depositors in low-$\beta$ markets are difficult to attract away from incumbents.

Our motivation stems from the widespread de-branching in the U.S. since the GFC, a process, we believe, that is far from complete. In fact, branch decline has proceeded nearly twice as fast in the EU, with some Scandinavian countries experiencing reductions of up to 90\%. Information and communications technology has reduced the importance of physical branches, and our results show that banks recognize heterogeneity in customer technology adoption and adapt strategically by closing the least profitable branches first. Earlier literature emphasized that branch ownership and location introduced frictions in banking services.\footnote{For example, areas with highly concentrated ownership of branches experienced less competition in both deposit and lending markets. Branch closures reduced local small business lending \citep{nguyen2019credit}. Capital flows were also shaped by ownership networks across local areas \citep{gilje2016exporting,cortes2017tracing}. The most powerful evidence of the importance of branching emerged from studies of deregulation of restrictions on branch ownership within and across state lines \citep{jayaratne1996finance,rice2010does}. Following such deregulation, credit and deposit market competition improved, capital mobility increased, and the supply of financial services expanded.} These frictions are now being eroded by information technology, which renders geographic distance and proximity increasingly irrelevant. Taken together, our evidence suggests that branch networks will continue to shrink and play a diminishing role in shaping deposit pricing and credit availability.

% \begin{comment}
\section{A Framework  to Understand Branch Restructuring}\label{sec:fw}
Incentives to close or retain branches arise primarily from two \textit{deposit-side} channels, both shaped by local demographics. The first is pricing power—how sensitive customers are to market rates and thus how much rent the bank can extract from its depositor base. The second is convenience value—branch usage and proximity—which determines how costly closure is for customers and therefore how easily banks can shed physical locations without losing deposits. Lending considerations may also matter through local credit demand and relationship intensity; we control for these and find they do not materially impact branch restructuring.


The first channel—customers’ sensitivity to market rates, and thus the rents a bank can extract from its depositor base—can be formalized following \citet{dss2023}. They show at the \textit{bank level} that deposit value depends on three components: the scale of deposits, the costs of operating them, and depositor behavior. Applying their framework to the branch level, the value of deposits at branch $i$ can be written as

\begin{equation}
V_i^{dep} = D_i \times (1-\beta_i - \frac{c_i}{r^p}) \times \Big[1-\frac{1}{(1+r^p)^{10}}\Big].
\end{equation}

where $D_i$ is the stock of deposits at branch $i$, $\beta_i$ is the deposit rate sensitivity, $c_i$ is the per-dollar operating cost, and $r^p$ is the long-term discount rate. As the expression makes clear, deposit value is increasing in scale, and decreasing in operating costs and depositor sensitivity.  \citet{dss2023} also assume deposits run off equally in each year over the next 10 years.

The first two components are relatively straightforward to handle empirically. Deposit scale is directly observable in Summary of Deposits (SOD) data, and local operating costs—driven by rents and wages—can be absorbed using county-by-year fixed effects. The third component, depositor behavior, is more subtle and requires a measure of how sensitive customers are to changes in market rates.

To measure depositor behavior, we focus on branch-level deposit interest-rate sensitivity, deposit-$\beta$, which serves as a sufficient statistic for local sophistication, rate sensitivity, and market power. Innovations in payment systems (e.g., PayPal, Venmo, Zelle) and the widespread use of internet and smartphone banking have heightened depositor sensitivity to market rates, lowering banks’ pricing power and diminishing the value to customers of close proximity to a branch. We take these technological shifts in the financial infrastructure as given, but exploit the fact that adoption of such technologies varies substantially across households. Younger and better-educated households, for example, adopt and use digital channels earlier and more intensively than older or lower-income households (\citet{FDIC2023}). As a result, branches with low deposit-$\beta$ tend to serve less sophisticated or inattentive customers who remain relatively insensitive to pricing, allowing banks to extract rents. Branches with high deposit-$\beta$, by contrast, serve customers who monitor market rates closely and reallocate balances when spreads widen, reducing the profitability of those deposits. In this way, branch-level deposit-$\beta$ captures both the value of a branch’s depositor base and the intensity of competitive pressure it faces.

Although some branch-level data exist—for example, RateWatch data for a subset of “rate-setting” branches—they are incomplete and typically insufficient to identify branch-level $\beta$ across the full universe of branches. Large banks often use uniform pricing, and information on deposit mix or reallocation of balances across products is not available at the branch level. We therefore impute branch-level deposit-$\beta$ as a latent attribute of the depositor base. In practice, we first estimate a regression of bank-level deposit sensitivity on demographic variables correlated with financial sophistication and rate sensitivity (age, education, income, stock-market participation).\footnote{\cite{egan2025dynamic} argue that depositor sleepiness - meaning the reluctance to move funds across accounts - explains much of the variation in the deposit franchise value.  Consistent with our results on rate sensitivity, they find lower account turnover among older depositors.} We then map the coefficients from this bank-level regression to the demographics of each branch’s local market to generate branch-level predictions. In this sense, our branch-level deposit-$\beta$ is not a literal measurement of past behavior, but an estimate of expected depositor sensitivity derived from observable characteristics of the local population.

A key advantage of this \textit{imputed} branch-level deposit-$\beta$ is that it provides a consistent ordering of branch profitability regardless of whether banks set deposit rates uniformly or differentially across locations. Under branch-specific pricing, banks optimally lower margins in markets where demand is elastic, so per-dollar profits are decreasing in deposit-$\beta$. Under uniform pricing, quantities adjust instead: high-deposit-$\beta$ branches experience larger outflows to substitutes or competitors when rates rise.\footnote{Consistent with this argument, Begenau and Stafford (2025) show that the gap between a bank's average deposit rate from the average paid by competing banks in a local market is strongly correlated with branch-level deposit growth in that same market.} In either case, higher deposit-$\beta$ reduces the value of a branch, through either lower margins or smaller quantities. This robustness makes branch-level deposit-$\beta$ a particularly useful summary measure of depositor behavior in the modern banking environment, where technology both heightens rate sensitivity and erodes the value of geographic proximity.

The second channel influencing branch closure decisions is convenience, which reflects the value customers place on proximity to a physical branch. When customers rely less on in-person access, banks can close locations with minimal risk of losing deposits. We capture this convenience channel through measures of branch usage and proximity, which proxy for how much depositors value continued access to a given branch.

In sum, local demographics drive both depositor sensitivity to rates and their reliance on physical branches, which together shape branch restructuring. The relationships can be summarized as:

\begin{itemize}
\item Local Demographics $\rightarrow$ Interest-Rate Sensitivity
\item Local Demographics $\rightarrow$ Branch Usage
\item Interest-Rate Sensitivity and Branch Usage $\rightarrow$ Branch Openings and Closures
\end{itemize}

In areas with more financially sophisticated households, depositors adopt digital technologies earlier, leading to higher interest-rate sensitivity and less reliance on branches. Consequently, branches in these markets—marked by high deposit-$\beta$ and low usage—are the most likely to close, while locations combining sensitivity with continued usage may still present opportunities for entry. 
% \end{comment}

\begin{comment}

\section{A Framework to Understand Branch Restructuring}
We follow \cite{dss2023}, who construct a simple measure of the profitability of bank deposits - the deposit franchise value (DF) - equal to the present value of gains associated with pricing the current stock of deposits at below-market rates of interest (so the deposit interest rate $r_i^d < r^f$, the market interest rate). In this framework, banks set their deposit rate equal to a fixed fraction of the market rate (so, $r_i^d = \beta_i \times r^f$); $\beta_i$ represents the deposit "mark-up" over marginal cost (the market rate of interest) or, equivalently, the interest-rate sensitivity.  We apply this framework to the branch level, as different branches held by the same bank have different clientele depending on their location, thereby facing different levels of interest-rate sensitivity. \citet{dss2023} also assume deposits run off equally in each year over the next 10 years, so the profits generated at each branch depends on three factors, as follows:


\begin{equation}
DF_i = D_i \times (1-\beta_i - \frac{c_i}{r^p}) \times [1-\frac{1}{(1+r^p)^{10}}].
\end{equation}


Where $D_i$ = total deposits at branch i, $\beta_i = \Delta r_i^d/ \Delta r^f$ is the price sensitivity at branch i (and represents the level of market power), $r_i^d$ is the interest expense on deposits per dollar of deposits, $r^f$ is the Fed Funds rate, $r^p$ is the the long-term interest rate. In (1), $c_i$ captures the annual non-interest (operating) cost of holding each dollar of deposits. In our models of branch closures (and openings), we include separate variables to capture each of these factors: the log of the amount of deposits at the branch (which is observable directly), the $\beta_i$ in each branch (not directly observable), and the per-dollar operating cost ($c_i$). Below we describe how we build the branch-level $\beta_i$.  For operating costs, which depend mainly on local rents and wage rates, we absorb them with county-time fixed effects. 

The DF framework assumes that banks price deposits differentially across their branches based on the local rate sensitivity.  But different levels of interest-sensitivity will capture the rank-ordering of each branch's value to its owner even under uniform pricing, meaning cases in which the bank sets a single deposit rate across all branches in a given region or even across its entire network.  This follows because deposit quantities ($D_i$ in (1)) would adjust, being higher in areas with low-$\beta$ customers (because the uniform-pricing bank would set deposit rates on better terms than the local market would otherwise bear) and vice versa in areas with high-$\beta$ customers.


We argue that innovations in both payment systems (e.g., PayPal, Venmo, Zelle, etc.), along with easy access to deposits via the internet and smartphones, have increased depositor sensitivity to market rates (thereby lowering bank pricing power) and also reduced the value to them of close proximity to a bank branch. We take these technological changes in the basic financial infrastructure as given, but exploit the fact that customer adoption of these technologies exhibits substantial heterogeneity. Younger and better-educated households, for example, interact with their banks via technology at higher rates and earlier than older and lower-income households (\citet{FDIC2023}). Hence, differences in the local demographics faced by different branches will drive variation in the value of those branches. We can represent the relationships we have in mind schematically as follows:
\begin{itemize}
    \item Local Demographics $\rightarrow$ Interest-Rate Sensitivity
    \item Local Demographics $\rightarrow$ Branch Usage
    \item  Interest-Rate Sensitivity and Branch Usage $\rightarrow$  Branch Openings and Closures
\end{itemize}

These relationships capture the central mechanism in our analysis. First, local demographics—such as age, education, income, and stock market participation—affect how depositors interact with banks. In areas with more financially sophisticated households, depositors are more likely to adopt digital technologies, resulting in higher interest-rate sensitivity and a lower deposit franchise value (as in (1)). Second, this same demographic variation drives patterns of branch usage: people in sophisticated zip codes visit branches less often and travel farther when they do, consistent with their substitution toward online and mobile banking channels. Third, branches that serve such communities—those with both low DF and low usage—are prime candidates for closure, as they offer less pricing power and limited in-person engagement. Finally, areas with high interest-rate sensitivity (low DF) but relatively higher potential usage (e.g., less sophisticated markets not yet fully penetrated) may still be attractive for entry, as new branches in those areas can target customers who remain dependent on physical access but are not yet served by the bank. 

This framework underlies our empirical design and helps explain the observed heterogeneity in restructuring decisions across banks, branches, and local markets.
Our empirical section below estimates these relationships in three steps: 1) we first estimate the relationship between local demographics and rate sensitivity at the bank level, and then use that model to compute rate sensitivity at the branch level; 2) we then estimate how demographics affect branch usage based on cell phone data; 3) we estimate how rate sensitivity and branch usage affect branch restructuring decisions. As argued by \citet{egan2022cross}, most of the value created by banks stems from payments-related services, which allow banks to raise and retain deposits at interest rates well below market interest rates (\citet{lu2024making}). \citet{drechsler2017deposits} show how this pricing power over deposits leads to a novel channel of monetary policy transmission driven by the cyclicality in bank deposit pricing. As such, we focus most of our attention on changes in the value to banks of paying below-market interest rates on deposits. In our last set of tests, we introduce branch usage metrics to the models, which are only available during the most recent portion of our sample.
\end{comment}
\section{Data Sources}
We combine data from several sources to construct our analytic samples. This section describes the underlying datasets and how we use them in the analysis.

\subsection{Summary of Deposits}
We use the FDIC’s annual \textit{Summary of Deposits} (SOD), which allows us to measure the amount of deposits and location of each bank’s branch network in June of each year, and also to observe branch openings and closings. We estimate our restructuring models during the years from 2001 to 2023. A branch is `closed’ ('opened') in year \textit{t} if it appears (does not appear) in June of year \textit{t} in the SOD dataset but does not (does) appear in the June of year \textit{t+1}. This definition can be established with certainty not only because the SOD contains a branch-based ID variable, but also because it contains detailed data on each branch’s physical location (e.g., latitude and longitude, as well as state, city and street address). Figure \ref{fig:pct_branches_closed} reports the fraction of branches closed (Panel A) and opened (Panel B) in each of the years we study, split by bank size (with large defined as banks with more than \$100 billion in assets).  Branch opening rates exceed that of closings in every year prior to the GFC, and vice versa in every year after.\footnote{We have verified that branches identified as openings are indeed new, as opposed to a branch which may have been closed and subsequently purchased by a bank entrant.}  Branch closures spike during the year of and following the COVID-19 Pandemic, with large banks closing nearly 8\% of their branches in 2020 - where again we define 2020 as the period between June 2020 and June 2021.


\subsection{Bank Call Reports}

The Federal Financial Institutions Examination Council (FFIEC) requires US banks to file information on their financial health and performance at the end of each quarter and these are made publicly available. These ``Call Reports'' provide a breakdown of balance sheets and income statements. %For our purposes, we obtain bank yearly information on assets, deposits, and interest expense on deposits (since we can only measure branch closures on an annual basis).

\subsection{Demographic data}

To capture local demographic factors, we use the American Community Survey (ACS) 5-Year Data, which provides economic and socioeconomic information across various geographical levels in the United States. Because ACS data begin in 2009, we use 2009 values—based on 2005–2009 averages—to proxy for earlier years in our analysis. We use the census tract level information on income, education, and age, and we also capture a measure of stock market participation using zip code level data from the IRS Statistics of Income (SOI) on Individual Income Tax Returns, specifically the fractions of tax returns reporting dividend income and capital gains.

\subsection{Branch Usage data}

We use the Advan (formerly SafeGraph) Monthly Patterns dataset, which provides aggregated raw counts of visits to points of interest (POIs) in the US, gathered from a panel of mobile devices. This anonymized and aggregated dataset provides details on monthly visitor frequency, duration, the origin census block group, and the distance the median branch visitor traveled. The dataset initiates from January 2019 and ends in 2023; however, we do not have usage data during the height of the pandemic. We use these data to identify how many customers use each bank branch and how far the median branch visitor traveled. Because of its limited coverage of branches closed prior to 2022, and because we lag the usage measures, we report branch closure models controlling for usage only for 2022 and 2023.

\subsection{Other Data Sources}

In addition to the primary datasets outlined above, we incorporate several other data sources to construct control variables for our regressions. The Home Mortgage Disclosure Act (HMDA) data is used to calculate bank-county mortgage originations as well as mortgage growth at county level; the Community Reinvestment Act (CRA) data is used to calculate bank-county small business loan originations and growth at the county level. County Business Patterns (CBP) data is used to calculate establishment growth and payroll growth at the county level. Additionally, the Federal Housing Finance Agency (FHFA) Underserved Areas data is used to create a zip code-level dummy variable indicating whether a zip code is classified as a low-to-moderate income (LMI) area, defined as having a median income below 80\% of the area median income.  


\section{Constructing Branch-Level Predictors of Restructuring}
To analyze branch restructuring, we require measures that capture how valuable a branch is to the bank. This value depends on two distinct channels shaped by local demographics: pricing power and convenience value.

\subsection{Deposit-$\beta$: Measuring Pricing Power}\label{sec:beta}
\subsubsection{Bank-Level $\beta$ and Its Determinants}

As shown in Section \ref{sec:fw}, the value of the deposit franchise depends on the quantity of deposits at each branch; the deposit-$\beta$, which captures market power and depositor behavior; and operating costs.  The first and third components are straightforward, as the SOD contains each branch's total deposits, and costs can by stripped out with local time-varying fixed effects.  This section describes how we build the $\beta_i$.

We focus on the last three monetary tightening cycles, using bank-level realizations of $\beta$: first for the 2004–2006 (early) cycle, second for the 2016–2019 (mid) cycle, and third for the 2022–2023 (late) cycle, applying a consistent estimation framework in each period. We first estimate cross-sectional regressions to reveal how local factors affect the deposit $\beta$. This allows both the mean level of $\beta$ to evolve over time and also allows the cross-sectional effects of demographics and concentration to vary over time. Realized $\beta$ equals the change in bank annualized interest expenses per dollar of deposits over each cycle (from Bank \textit{Call Reports}), normalized by the change in Federal Funds rate over that cycle (= 4.25\% in the early cycle, 2.5\% in the mid cycle, and 4\% during last cycle). We use just the increasing portion of each rate cycle. The end of both the mid and the late cycles coincide with crises (GFC and the COVID Pandemic), making the subsequent rate declines and bank reaction to those declines difficult to interpret, and not reflective of their normal response to market-rate changes.\footnote{Banks received massive deposit inflows during this time due to large transfer payments to households, workers and firms under the CARES Act. These exogenous shocks to deposits may disturb the normal pricing reaction of banks to a decline in interest rates which would not be a good representation of their pricing power.} The sample period in the third cycle ends at the first quarter of 2023 to ensure that changes following the Silicon Valley Bank (SVB) collapse do not impact our estimations.

To build cross-bank local drivers of $\beta$, we average the demographic and market characteristics of residents living near each bank’s branch (based on zip code), weighted by the amount of deposits each bank holds in each of its branches.\footnote{We rely on pooled models (i.e., large and small banks together) to identify the effects of local variables on $\beta$. Estimating these models for large banks alone would be problematic because they own branch networks distributed widely across the country, thus restricting the variation in their exposure to (averaged) local factors.} These regressions are structured, as follows:

\begin{equation}
    \beta_{j,t} = \sum \gamma_t^k D_{j,t}^k + \eta_t HHI_{j,t} +\text{Other controls}+\varepsilon_{j,t}
\end{equation}

where \textit{j} represents bank, \textit{t} represents one of the three rate cycles, \textit{k} represents four demographic variables: age (using three quartile-binned indicators), log of mean family income, the fraction of tax filers reporting stock-based income, and the fraction with a college degree. We average each of these demographics across each zip code in which bank \textit{j} owns its branches, weighted by deposits in each branch. In addition, we capture the deposit-weighted average level of concentration across each bank’s markets (\textit{HHI\textsubscript{j,t}}), where markets are defined at the county level. The other control variables include bank-level population density, calculated as the county-level population density weighted by deposits in each county, a measure of bank size (log of total assets), transactions deposits as a percent of total assets, and uninsured deposits as a percent of total assets. The dependent variable in Equation (2) equals the change in the interest expenses on deposits per dollar of deposits in each cycle, scaled by the corresponding change in the Fed Funds rate.

Table \ref{tab:bank_desc_stats} reports summary statistics for the regression samples in the three cycles, separated for large ($>$\$100 billion) and small banks. Explanatory variables are measured at the beginning of each rate cycle. For small banks, the mean realized $\beta_j$ ranges from 0.17 to 0.23; for large banks, the $\beta_j$  ranges from 0.24 to 0.32. Large banks operate in areas with younger, more educated, wealthier populations that have higher rates of stock market participation than small banks. These demographic differences do not vary much over time across large and small banks. Small banks are more likely to have branches in rural areas.

Figure \ref{fig:df_density} reports histograms of the bank-level $\beta_j$ values in Panel A and the branch-level $\beta_i$ values in Panel B, split by the three monetary tightening cycles and by bank size. Large banks have higher $\beta$s on average in all three cycles (recall Table \ref{tab:bank_desc_stats}), although there is substantial overlap in the distributions.

Table \ref{tab:deposit_beta_reg} reports estimates of Equation (2) for the three rate cycles. Age, income and education are all correlated with bank pricing power as expected (columns 1-3). Banks with (potential) customers near their branches who are younger and more highly educated have \textit{lower} pricing power (higher $\beta$). The age effect is driven by the oldest quartile. Increasing the share of a bank’s clientele with a college degree by one sigma (from the large-bank sample) raises $\beta$ by 0.08 (=0.15 $\times$ 0.56) in the late cycle, for example. Banks have higher pricing power in areas with higher income. Concentration (HHI) enters all of the models negatively (suggesting greater bank pricing power in more concentrated areas). In contrast, local population density – which itself is strongly correlated with HHI - has a positive impact on $\beta$, with increasing importance over time. In other words, people living in urban areas are more price sensitive, and this difference in sensitivity has grown sharply over time.  Large banks also exhibit much higher $\beta$s than small, particularly in the last cycle.\footnote{Note that we absorb size effects in our closure models with fixed effects.}

Columns 4-6 of Table \ref{tab:deposit_beta_reg} report a more parsimonious model which collapses two demographic characteristics into one: the fraction of residents in ‘sophisticated’ zip codes. We build the zip-code classification by flagging localities with above-median education and above-median stock market participation. We use this indicator to differentiate areas dominated by financially sophisticated people versus those without. We include age and income in these models as separate factors. As the results show, the deposit $\beta$ is consistently higher in sophisticated areas, and the impact of financial sophistication increases in the late cycle relative to the first two. For context, a bank raising all of its deposits in areas with financially sophisticated depositors would have a deposit-$\beta$ that is 7\%-10\% higher, compared to a bank raising all of its deposits in areas with less-sophisticated depositors.  Such a bank would pay 7-10 basis points more per 100 basis points increase of interest rates.

\subsubsection{Predicted $\beta$ at the Branch Level}
We construct branch-level measures of $\beta_i$ applying the coefficients from Equation (2) to each bank’s branches using the demographic measures from that branch’s zip code (opposed to the average across all branches, as in Equation (2)). We use the coefficients from the early interest-rate cycle for the years 2001-2014, from the mid cycle for 2015-2019, and the coefficients from the late cycle for the years 2020-2023.\footnote{Our aim is the allow the marginal effects of customer demographics to shift over time. We recognize that the coefficients we use are not strictly out-of-sample. Changing this mapping has little effect on our results because the coefficients are fairly stable, and because the demographic variables are very persistent.} Since banks own branches in zip codes with different demographic factors, this procedure generates within-bank variation in the interest-rate sensitivity which we exploit in our closure and openings models.

Panel B of Figure \ref{fig:df_density} shows the distribution of predicted $\beta_i$ at the branch level. The densities reveal substantial within-bank heterogeneity, reflecting variation in local demographic characteristics across branch locations. This variation provides the identifying power for our branch-level analysis of closure and decisions, as it allows us to compare branches within the same bank that differ in the profitability of their deposit base.

The persistence of both bank-level and branch-level differences over time is illustrated in Figure \ref{fig:df_scatter}, which compares the $\beta$s across interest rate cycles. Panels A and B scatter the bank-level values for the early vs. mid cycles and mid vs. late cycles, respectively. Panels C and D show the same comparisons using the branch-level predicted values. As expected, the cross-sections are highly correlated over time, with a tighter relationship at the branch level. This reflects the stability of both the explanatory variables (demographics change slowly) and the coefficient estimates, as documented in Table \ref{tab:deposit_beta_reg}. Put simply, branches with above-average $\beta$ early in the sample tend to remain above-average in later years. The figure also shows that their $\beta$ values are generally higher in the post-pandemic period. Between the mid and late cycles, the fitted line rotates upward around the 45-degree line, consistent with a sharp increase in the effects of bank size and population density on bank-level $\beta$s. Larger banks and those in urban areas now face higher $\beta$s, and these effects are strengthening over time. We use these predicted values in our core branch closure models below.

\subsection{Branch Usage: Measuring Convenience Value}\label{sec:usage}
% Our core results show that branch-level $\beta$ summarizes depositor behavior and predicts branch restructuring. A natural question, however, is whether this imputed measure of rate sensitivity also captures differences in how customers use branches, particularly in response to new technologies. To validate the mechanism, we complement our $\beta$-based analysis with direct evidence on branch reliance from cell phone mobility data. These data allow us to observe how often people visit branches and how far they travel, providing a window into the amenity value customers place on proximity.

% We emphasize that this analysis does not introduce a separate explanatory channel. Instead, it demonstrates that the same demographics that drive higher $\beta$—younger, more educated, more financially sophisticated households—also predict reduced branch usage, especially around the COVID-19 pandemic when many customers learned to substitute technology for in-person banking. In this way, the usage patterns provide corroborating evidence that branch-level $\beta$ is a sufficient summary statistic, capturing not only rate sensitivity but also differences in the value of proximity that shape banks’ closure decisions.
As discussed earlier, customer demographics shape branch restructuring through two main channels: by influencing banks’ pricing power and by affecting the value customers place on physical proximity to a branch, particularly through differences in technology adoption. The COVID-19 pandemic provided a shock to this valuation, as many individuals became more comfortable using technology to substitute for in-person interactions. To capture these changes, we begin by examining shifts in branch foot traffic between 2019 and 2021 using Advan cell phone data, where \textit{Drop in Visits} is defined as (Traffic in 2019 – Traffic in 2021) / Traffic in 2019. We interpret larger declines in foot traffic as indicative of greater reliance on digital banking relative to in-branch services. We also use the Advan data to measure the median distance traveled by visitors to reach a branch in 2019. These metrics, based on location and time-stamped mobile device data, serve as inputs into the following model:

\begin{equation}
    \text{Usage}_{j,i} = \sum \gamma^k D_{i}^k +\text{Other controls}+\varepsilon_{j,i}
\end{equation}

where \textit{j} indexes banks, \textit{i} indexes branches, \textit{k} indexes the demographic variables observed in each branch's zip code.\footnote{As noted, we do not have cell phone data during the Pandemic year of 2020. Even if we did, these data would be highly unrepresentative of normal behavior due to the effect of lockdowns and general fear of COVID contagion.} The control variables include a bank fixed effect, state fixed effect, the county-level population density, and the log total deposits held in branch \textit{i}.

Table \ref{table3} reports the estimates of equation (3) with all four demographic factors (Panel B), and also the more parsimonious version in which we collapse education and stock-market participation into a single sophisticated indicator (Panel A), as in Table \ref{tab:deposit_beta_reg}. Regressions are split based on bank size.

These results show a strong effect of local demographics on branch usage. For both large and small banks, usage declines sharply around the pandemic: the mean decline in foot-traffic between 2021 and 2019 equals 31\% for large-bank branches and 15\% for small.  The regressions show that this decline is higher in areas with more sophisticated clientele.\footnote{\cite{sakong2025drives} find higher demand for branches among high-income populations, based on foot-traffic prior to the Pandemic.  We also find the usage falls less around the Pandemic in higher income areas, but our results are hard to compare with theirs because we also condition on education, stock market participation and age, all of which are correlated with income.} The number of visits, for example, drop 7 percentage points more at large-bank branches located in these areas (and 10 percentage points more for small banks). Moreover, visitors from sophisticated areas are on average traveling from further away when they do visit a branch. In other words, customers in financially sophisticated locations value physical branches less than customers in other areas. Age also correlates strongly with usage, with foot-traffic falling most for areas with many young people (the omitted group in the regression). As the model shows, the age effects on the drop in visits is monotonic, while the effect on distance is non-monotonic across the distribution.

Next, we turn to studying how the two bank branch usage measures evolve dynamically around the Pandemic. To examine these changes, we estimate a regression of the following form:

\begin{align}
    \text{Usage Metric}_{i,m} &= \sum_m \beta_m \times \text{Sophisticated zip} \times (month=m)  \nonumber \\
    &+\text{Other controls}+\text{Fixed effects}+\varepsilon_{i,m}
\end{align}
where \textit{i} is the branch and \textit{m }is the month. The regression includes bank, month, and zip code fixed effects. The coefficient  captures the differential effect of financial sophistication on branch usage across months, relative to the omitted base month (January 2019).

Figure \ref{fig:dynamic_did_plots} reports the  estimates with the corresponding 90\% confidence intervals, presented separately for large and small banks. Panel A uses monthly visits as the dependent variable and Panel B uses the natural logarithm of the median travel distance to the branch. The clear pattern in Figure \ref{fig:dynamic_did_plots} is that the number of visitors to branches drops more in financially sophisticated areas (i.e., the betas plot below zero), and the effect of financial sophistication grows around the pandemic.  These effects are most pronounced for the small banks (consistent with the regressions in Table 3). Like visits, the average travel distance to branches is longer in financially sophisticated areas; while this effect increases somewhat after the Pandemic, the change is smaller than for the number of visits.  As such, we focus on the cross-branch variation in travel distance, rather than its change around the Pandemic, in our models of branch restructuring below.

These findings align with broader societal shifts induced by the pandemic. Lockdown rules forced many people to rearrange their work schedules, leading to lasting behavioral changes, including a significantly higher prevalence of working from home. As \citet{barrero2023long} document, only about 5\% of Americans worked from home before the pandemic, but this figure surged to 60\% during the lockdowns and has since stabilized at around 30\%. Consistent the results in Table \ref{table3}, the increase in remote work was far greater for individuals with a college degree or higher, reflecting their greater ability to adapt to remote work arrangements. The large difference in usage patterns based on demographics follows because higher income and more educated people were more likely to be able to work at home, compared to other people. 

As noted, these cross-sectional and time series patterns suggest that more financially sophisticated people value proximity to a nearby bank branch less than other people.  When the Pandemic hit, they reduced branch visits more, and traveled further when visiting a branch. Such effects, we argue, occur because these customers are accessing banking services increasingly with technology – internet and mobile banking – and more so than less sophisticated customers. The results of Tables \ref{tab:deposit_beta_reg} and \ref{table3} together imply that a branch’s natural depositor clientele – the people living near the branch - drives both the pricing power of those branches (Table \ref{tab:deposit_beta_reg}) as well as the usage of those branches (Table \ref{table3}). We test whether these two usage factors help explain branch opening and closing decisions in later regression.

 



\section{Determinants of Branch Closings and Openings: Empirical Design}
Our analysis so far has shown that local demographics drive variation in branch usage and deposit sensitivity, with more financially sophisticated areas exhibiting higher Deposit-$\beta$s and reduced reliance on physical branches. We now turn to examining how these factors influence banks’ decisions about which branches to close and where to open new ones.

%


% We estimate three types of models to analyze the drivers of branch restructuring. The first focuses on deposit sensitivity, measured by branch-level predicted $\beta$. Since the data structure differs between openings and closings, we describe each in turn:

\subsection{Closures}

We use the following linear probability model where the dependent variable (\textit{Closure}) indicates if a particular branch \textit{j} owned by bank \textit{i} was closed in year \textit{t}:

\begin{equation}
    \text{Closure}_{i,j,t} = \gamma Beta_{i,j,t} + \theta log(Deposits)_{i,j,t} + \text{Other controls} + \text{Fixed effects} + \varepsilon_{i,j,t} \tag{5a}
\end{equation}

In (5a), we capture the effect of the deposit franchise with the first two terms, the quantity of deposits at the branch (\textit{log(Deposits)}) and their interest-rate sensitivity ($\beta$) for branch \textit{j} owned by bank \textit{i} at time \textit{t}, as described above. Our baseline estimates pool the branch-year data across the full 2001-2023 period. We report models with $state\times year$ and $bank \times year$  fixed effects, and we report models with  $county \times year$ fixed effects as well.  These granular fixed effects absorb variation in branch-level operating costs. Moreover, by including $bank \times year $ fixed effects (as well as  $county \times year$ effects in some specifications), we fully absorb both the general trends in banking and technology, as well as heterogeneity in the supply of technology across banks. For instance, \citet{haendler2022keeping} shows that large banks adopted and updated mobile apps earlier and more frequently than smaller banks. This approach absorbs supply-side differences in the quality and quantity of online and mobile banking services, as these are common across all customers of a given bank, regardless of branch location. In addition, the $county \times year$ absorbs variation in local access to technology, such as differences in investment in the quality of the cell phone network. Thus, identification comes solely from variation in the impact of local demographics (i.e., demand-side factors) on branch closures.

At this stage, we exclude measures of branch usage, as these are only available for the final two years of our sample. We also estimate the model separately for large and small banks, reflecting differences in customer demographics (\citet{d2023deposit}) and the potential variation in marginal effects due to differences in the quantity and quality of services offered.

To capture the potential effects of local lending and loan demand, we include the following: the (logged) level of mortgage originations and small business originations at the bank-county-time level, as well as the county-level three-year past growth rate of both measures. In principle, we could also build an analogous metric to the Deposit-$\beta$ based on loan-market pricing power.  However, available measures of loan pricing, such as average interest income on C\&I loans, would be driven mainly by large borrowers; such loans would not reflect local pricing power. Moreover, most of the variation in observed lending rates reflects differences in risk rather than mark-ups from market power.  Hence, we focus on quantity-based measures of local lending conditions. 

In addition to these, we include market control variables such as the three-year past growth rate of county deposits, the county-level growth in the number of establishments, and payroll growth; these help capture local economic growth effects. Two M\&A-related indicators are added: the first equals one if the bank has owned at least one branch in the zip code for the past three years, and the second equals one if the current branch was acquired by the bank in the past three years and bank has owned at least one other branch in the same zip code prior to the acquisition; these help capture any restructuring effects associated with M\&A activity. Finally, we include an indicator for branches in low- and moderate-income (LMI) areas, as defined by the Community Reinvestment Act, where banks face regulatory pressure to lend locally.

\subsection{Openings}

To understand branch openings, we construct a $bank \times zipcode \times year$ dataset which captures candidate zip codes where each bank might choose to open a new branch.  For each bank-year, we include all zip codes in the CBSAs where the bank owned at least one branch in the prior year, and we add all zip codes in CBSAs in which the bank opens a new branch in the current year.\footnote{The latter zip codes - those in CBSAs where a bank opens a branch for the first time - are potentially endogenous because they are conditional on the bank entering the area.  In Internet Appendix, however, we verify that our core results are similar if we exclude these observations.}  Note that the set of candidates zip-codes differs across banks and time.  We also drop all zip codes in which no bank ever owns a branch during our sample. The dependent variable is set to one if the bank opens a new branch in the candidate zip code and zero otherwise.  

With this sample, we estimate linear probability models parallel to 5(a), although we replace the lagged log level of deposits in an incumbent branch with the log of (1+deposits) based on all bank branches located in the given zip code during the prior year (i.e., branches owned by competing banks).  We build the lending variables analogously using lagged log values for county mortgage and small business lending volumes.  Since potential entrants have no deposits (lending) from the prior period, we interpret this variable as a measure of the potential (or maximum) level of deposits a new bank could raise.  As in 5(a), we estimate the base model with $bank \times year$ fixed effects, as well as $state \times year$ or $county \times year$.

\begin{comment}
    

\subsection{Reduced Forms}

In our second set of models, we extend the analysis by using a reduced-form version of (5a) and its analog for openings, replacing the predicted Deposit-$\beta$ with the underlying demographic and market concentration variables that were used to construct it. The regression specification is as follows:

\begin{equation}
    \text{Closure}_{i,j,t} = \sum_k \gamma_t^k D_{i,j,t}^k + \eta HHI_{i,j,t} +  \theta Ln Deposits_{i,j,t} + \text{Other controls} + \text{Fixed effects} + \varepsilon_{i,j,t} \tag{5b}
\end{equation}

Here, $D_{i,j,t}^k$ represents the k-th demographic variable at branch \textit{j} owned by bank \textit{i} and time \textit{t}, and $HHI_{i,j,t}$ captures the concentration in branch \textit{i}'s market.  These effects collectively capture the impact of the Deposit-$\beta$. Fixed effects again include $bank \times year$ and  $state \times year$ or $county \times year$.  We report similar regressions for the choice to open new branches. 

This approach allows us to test which local variables are most tightly linked to branch restructuring decisions. In these models, we report an additional specification combining education and stock-market participation into a single financial sophistication measure. 

\subsection{Models with Usage}


\end{comment}
Across all models, we build standard errors by clustering at the bank level.

\section{Determinants of Branch Closings and Openings: Results}

Figure \ref{fig:pct_closed_df_bin} reports binned scatter plots by deciles of predicted $\beta$. Panel A shows the percentage of branches closed within each decile of branch-level predicted $\beta$. Panel B shows the percentage of ZIP codes with new branch openings by decile of ZIP-code–level $\beta$.

Across cycles—especially the second and third—we observe a strong positive gradient: higher-$\beta$ branches close more often, and higher-$\beta$ ZIP codes see more openings. Closure rates reach roughly 3–5 percent per year in the top $\beta$ deciles, compared with under 2 percent in the lowest deciles, while openings also concentrate in the high-$\beta$ deciles.%\footnote{A similar graphical analysis is not appropriate for branch openings, as banks face a wide range of potential locations for new branches and not just the zipcode where they actually open a branch. In contrast, branch closures are limited to zip codes where the bank already operates.}

%We first present summary statistics of the branch-year panel sample used for the closure regressions, focusing on the years 2012 and 2019. 
Tables \ref{tab:branch_desc_stats} and \ref{tab:branch_opening_desc_stats} provide summary statistics of key characteristics for the branch closure sample and branch opening sample, respectively. For both the dependent variables as well as the $\beta$ and the level of deposits, we report the sample standard deviation (SD) as well as the "SD (within)," which removes variation explained by the bank-time and county-time fixed effects; we use that latter metric to assess economic significance of the regression results below.

The tables split the summary statistics by bank size to highlight differences between small and large banks. While many mean characteristics are similar across the two groups, there are notable distinctions. Large banks, for example, hold significantly more deposits per branch, with the typical branch holding more than twice as many deposits as those of small banks. Geographically, small banks are more prevalent in rural areas, as indicated by their branches being in regions with much lower population density. 

\subsection{Baseline Results}

Table \ref{tab:baseline_closure_regression} reports our estimates of Equation (5a). We report the pooled sample (columns 1 \& 2), and then the split-sample results by bank size (over versus under \$100 billion in assets) in columns 3-6. For each set of specifications, we report models with $state \times year$ fixed effects and separately with $county \times year$ fixed effects. For all banks, in columns (1) and (2), lower $\beta$ leads to lower probability of a branch being closed. The magnitude is substantial across all banks, but also larger for the large banks. A one-standard-deviation increase in $\beta$ (in 2019 $\approx$ 0.02 for large banks and 0.015 for small banks after removing variation explained by fixed effects), for example, leads to a increase in large-banks' branch closure probability of about 0.5 percentage points (column 4). This effect is economically important, equal to more than 10\% of the unconditional mean closure rate (=4\% per year for the large banks) during our sample. Beyond $\beta$, which captures local market power, higher levels of deposits in the branch from the preceding year also has strong power to predict branch closures for both large and small banks. Hence the franchise value or rents from deposits seems to drive closure decisions.

Unlike the deposit franchise value, lending has little power to explain branch closures.  In the pooled model (column (1)), neither bank-level nor county-level growth of small business lending has statistical significance.  Mortgage growth does correlate negatively with closures, but this is driven by small banks.  We also see a marginally significant effect of small business lending growth on small-banks' branch closures (column 6). Comparing the collective power to explain closures illuminates the difference: in the pooled model the two deposit variables (Ln Deposits and $\beta$) have an F-statistic of 480.4; by contrast, the four lending variables have an F-statistic of just 2.96. The contrasting power of deposits v. lending suggests that the core purpose of bank branches (especially for large banks) is to support the deposit franchise, where banks remain dominant. On the other hand, banks have become increasingly \textit{less} important as suppliers of local credit - mortgages and small business loans. Moreover, deposits constitute about 85\% of all bank financing, while local lending comprises a small percentage of total bank investments (again, especially for larger banks).

The models also suggest that banks are much more likely to close branches acquired recently if they already had branch presence in the zip code, and less likely to close `legacy’ branches, meaning those which have not been acquired over the past three years. We find no evidence that banks close branches in localities defined as LMIs under the Community Banking Act; if anything, large banks are \textit{less} apt to close branches in these areas.\footnote{Banks are required to give regulators and local customers notice before closing a branch under Section 42 of the Federal Deposit Insurance Act. Hence, large banks may be concerned that closing branches in LMI areas could lower their CRA rating, which in turn could impinge on future acquisitions.}

Table \ref{tab:baseline_open_regression} reports the baseline estimates for branch openings. The DF again plays a central role in predicting entry, but with opposing effects depending on whether we focus on quantity or on price-sensitivity. Higher interest-rate sensitivity (which lowers DF) among local depositors is associated with \textit{increased} entry. A one-standard-deviation increase in $\beta$ (in 2019 equal to approximately 0.02 for both large and small banks, after removing variation explained by fixed effects) raises the probability of opening a branch by 0.12\% (=0.02 x 0.06, from column (4)), or about one-quarter of the unconditional opening rate of 0.42\% for large banks.\footnote{For large banks, the unconditional branch opening rate declined from approximately 0.8\% prior to the GFC to about 0.2\% afterward. The corresponding rates for small banks fell from 0.3\% to 0.09\%.} Local deposit levels, in contrast, positively predicts entry, as does establishment growth. Hence, banks enter new markets which are 'rich' in deposits, but only when price sensitivity is high.\footnote{Consistent with our results on entry, \citet{begenau2023uniform} show that banks are more likely to expand into large, wealthy, and competitive markets with high pass through rates and deposits grow much faster at new branches compared to older ones.} 

Table \ref{tab:baseline_open_regression} again suggests that lending has much less power to explain branch openings than deposits.  The four local lending variables have some statistical power, but with inconsistent sign patterns. For example, both county-level mortgage growth and bank-county levels of small business lending are negatively associated with entry for the large-bank sample (column 3).  For small banks, two of the lending variables enter positively (as one would expect), but the other two negatively.  The results do show a strong and consistent positive effect of overall local economic conditions (i.e., establishment growth), but this effect does not pin down a deposit v. lending channel, as both would tend to be positively related to a locally booming economy.\footnote{We have also estimated closure models which control for an indicator equal to one for zip codes with new branches opened within the past three years.  These markets exhibit higher closure rates, but adding this variable has little impact on our core results.  Similarly, zip codes with recent closures have a higher probability of entry (openings), but again adding this has little effect on our core results.  See the Internet Appendix for these results.}

% Table \ref{tab:baseline_open_regression} reports the analogous estimates for openings.  The level of deposits at the zip-code level as well as local economic growth both enter strongly and with opposite sign compared to closures: more deposits and higher establishment growth predict greater entry.  For one standard deviation increase in within fixed-effects standard deviation ($\approx$ 0.004 for large banks and 0.0045 for small banks), the probability of opening a branch drops by 0.12\%, which is approximately 30\% of unconditional probability of opening a branch.  But the effect of the DF - which varies based on local-resident price sensitivity - explains both entry and exit similarly.  Low values of DF predicts that incumbent banks become more likely to exit, but also increases the probability of entry by new banks.  Said differently, areas where customers have high interest sensitivity (and thus low DF) are characterized by high levels of both entry and exit.  These areas are more dynamic; areas with low rate-sensitivity depositors, in contrast, experience low levels of both entry and exit.  Beyond deposits, lending growth has at best a weak relationship to entry with inconsistent sign patterns, similar to its weak effect on closures.

%The economic significance of DF is also sizable for openings, as it is for closures.  After removing the fixed effects, the DF in the openings data has residual standard deviation of 0.004. So, a one-sigma decrease in DF raises the probability of opening a branch in a given zip code by 0.0012 for large banks (=0.004 x -0.3, from column (4)).  The unconditional probability of opening in a given zip code equals about 0.01 before the GFC and about 0.005 after (conditional on a bank opening at least one branch), so the economic impact is on the order of 10 to 20 percent of the unconditional probability.

\subsection{Estimations Over Time}

Tables \ref{tab:closures_by_regime} and \ref{tab:openings_by_regime} report estimates of Equation (5a) during four regimes: the period prior to the GFC (2001-2007); the years affected by the GFC (2008-2011); the post-GFC / pre-Pandemic years (2012-2019); and the post-Pandemic years (2020-2023). Panel A reports the model for large banks, and Panel B small ones (with $county \times year$ or $state \times year$ effects). These results suggest that branches with high Deposit-$\beta$s (i.e., with interest-sensitive customers) are consistently most likely to be closed, yet branch entry is also higher in areas with high Deposit-$\beta$s. For closures, the effects are largest after the Pandemic, as the rate of de-branching accelerated. For large banks, a standard-deviation decline in the Deposit-$\beta$ (=0.0177) comes with an increase in closure probability of 0.7\% (column (8)), equal to about 14\% of the unconditional closure rate. As in the baseline model, openings are also consistently higher in areas with high $\beta$s, both across time and also for large and small banks.\footnote{In an earlier version of this paper we report consistent effects over time using year-by-year regressions rather than pooled ones across the four regimes.}  

To summarize, the effects of both the level and pricing of deposits across all models, across all types of banks, and across time tell the same story.  First, incumbent banks \textit{close} branches where the per-dollar economic rents are low and where the total amount of deposits in their branch are also low.  Second, banks \textit{open} new branches in areas with high levels of deposits held by incumbents (because they are entering to raise deposits), but only where local depositors are price sensitive (because they can't effectively draw deposits away from incumbents when customers are rate insensitive).

\subsection{Adding Branch Usage}
Next, we incorporate the two cell phone-based branch usage measures as right-hand side variables. These specifications allow us to assess the relative importance of pricing power (deposit $\beta$) versus customer convenience or the amenity value of proximity.  

For the closure analysis, we can directly observe the two usage measures, as we do in Table \ref{table3}.  Branch-level usage patterns, however, are potentially endogenous and may respond to depositor expectations that a given branch will close.\footnote{In fact, by regulation banks are required to inform depositors of an impending closure by mail with at least 90 days notice.  See https://www.fdic.gov/consumer-resource-center/2024-07/your-bank-branch-relocating-or-closing.}  For example, if depositors are informed that their branch will close, they may increase their in-person visitations to the branch. Hence, we adopt a 'leave out' strategy to build the two usage measures, as follows: for each branch, we compute the average \textit{Drop in Visits} and the average \textit{Log(Distance Km)} for all other branches located in the same zip code.  As such, we drop all branches which are located in zip codes without competing branches.

For openings, there is no latent endogeneity problem because banks can only form expectations of usage based on patterns observed for existing branches of other banks.  So, we build the usage measures based on the zip-code level averages of \textit{Drop in Visits} and \textit{Log(Distance Km)} for all branches located in each bank's candidate zip codes.

As noted, although the Advan data start in 2019, we estimate these models only in 2022 and 2023. The industry (NAICS) codes for the closed branches were changed in 2021. As a result, branches closed in 2020 and 2021 have different NAICS in the Advan data, which doesn't provide a historical time series of these codes by location. When creating the dataset, we initially filtered locations with NAICS code 522 to indicate a banking office, so by necessity we filtered out these locations due to the change in the NAICS to a different code.\footnote{It is computationally prohibitive to standardize all the US addresses and then match only by the address without first filtering by the NAICS code.}
Tables \ref{tab:closure_with_usage_controls} and \ref{tab:openings_with_usage_controls} report our estimates for closures and openings after incorporating usage.  These models allow us to compare the relative importance of branch usage (from declines in foot traffic from 2019 to 2021) vs. banks’ deposit pricing power ($\beta$). The regressions include only the last two years of our sample (2022 and 2023) due to data constraints in the Advan cell phone data, as noted above. Both the results for openings and closings continue to show that the DF remains the dominant driver of branch restructuring. Adding the usage metrics attenuates the effect of $\beta$, from 0.24 to 0.12 for large banks (Table \ref{tab:closure_with_usage_controls} , Panel B, columns (2) vs. (4)); in contrast, the effect of the level of deposits does not change with usage added to the model.  Consistent with expectations, areas which experience large drops in branch usage (\textit{Drop in visits}) around the pandemic experience more closures.  This effect represents the 'teachable moment' of the Pandemic, in which many people learned how to substitute on-line technology for in-person interactions.  As such, it represents a large shock to the value of close proximity to a bank branch.  In addition, we find consistent evidence that banks are more likely to close branches located in areas where customers travel from greater distances.

The effects of usage on openings are less clear, however.  Like closures, $\beta$ continues to correlate positively with branch openings.  However, we find some evidence - mainly from small banks - that openings also are higher in areas with large declines in foot traffic.  This may reflect the fact the these areas are ones where technology adoption was greatest, which raises price sensitivity and also lowers the amenity value of close geographic proximity.  We find essentially no explanatory power of travel distance in any of the openings specifications.


\section{Reduced Form Results}
In our next set of models, since both deposit rate sensitivity and branch usage are strongly predicted by local demographics, we estimate a reduced-form specification that replaces these measures with their demographic determinants. We implement a version of (5a) and its analog for openings, replacing the predicted Deposit-$\beta$ with the underlying demographic and market concentration variables that were used to construct it. The regression specification is as follows:

\begin{equation}
    \text{Closure}_{i,j,t} = \sum_k \gamma_t^k D_{i,j,t}^k + \eta HHI_{i,j,t} +  \theta Ln Deposits_{i,j,t} + \text{Other controls} + \text{Fixed effects} + \varepsilon_{i,j,t} \tag{5b}
\end{equation}

Here, $D_{i,j,t}^k$ represents the k-th demographic variable at branch \textit{j} owned by bank \textit{i} and time \textit{t}, and $HHI_{i,j,t}$ captures the concentration in branch \textit{i}'s market.  These effects collectively capture the impact of the Deposit-$\beta$. Fixed effects again include $bank \times year$ and  $state \times year$ or $county \times year$.  We report similar regressions for the choice to open new branches. 

This approach allows us to test which local variables are most tightly linked to branch restructuring decisions. In these models, we report an additional specification combining education and stock-market participation into a single financial sophistication measure. 

Tables \ref{tab:reduced_form_closure} and \ref{tab:reduced_form_opening} reports parallel models of branch closures and openings estimated as reduced forms (as in Equation (5b)). Panel A reports the full sample, and Panel B split by size. We find that branch openings and closings, for both large and small banks, are more likely in areas dominated by financially sophisticated people. Each of these effects is consistent with our interpretation that the value of the deposits plays a key role in bank branching decisions via its link to market power (from higher or lower levels of interest-rate sensitivity) and usage. Comparing these results with those from Tables \ref{tab:deposit_beta_reg} and 3, we see that high levels of financial sophistication lead to higher Deposit-$\beta$s and lower usage (i.e., larger declines visits and customers traveling from further away); these in turn are associated with both higher rates branch openings and closings.  The entry effects are similar, comparing areas with younger vs. older populations (Table \ref{tab:reduced_form_opening}).  For closures, however, the effects of age are less consistent across banks of different sizes.





\section{Conclusion}

Banks opened new branches at a higher rate than they closed them until the GFC, when this pattern reversed sharply. We show that variation in economic profits generated from bank deposits helps explain branch restructuring patterns, as banks are most likely to close branches in areas with  low franchise value due to high interest sensitivity of local residents. Technologies which make physical proximity less important and which lower the cost of moving funds to substitute investments, we argue, drove the regime shift in branching starting around 2010. In contrast to other research, our empirical design exploits different rates of technology adoption by bank customers, which varies depending on characteristics associated with financial sophistication. The results suggest that the 2020 Pandemic had a large effect on technology adoption, leading to a sharp decline in foot-traffic at branches and an overall rate of branch closure roughly double what had come before.

Our results point to the importance of analyzing industry structural change using gross measures of openings and closings.  Incumbent bank incentives differ sharply from those of potential de novo entrants.  As we show, entry is higher consistently across bank types and over time in areas where local residents have high price sensitivity – exactly the areas where incumbents are most likely to exit.  Conversely, price insensitivity of depositors creates an endogenous entry barrier for greenfield investment, and helps explain why most bank extensions into new markets happen via M\&A facilitated by deregulation, which allows an entering bank to buy the existing customer base.

Understanding the drivers of branch closures matters because branch-based frictions have traditionally mediated flows of capital across markets and have affected local-market competition in both deposit and credit markets. Such frictions reduce financial market efficiency and integration. Lowering these frictions through technology furthers a process which began in the 1980s with deregulation of restrictions on branching and interstate banking. As such, continued bank restructuring will likely improve the functioning of local financial markets further.



