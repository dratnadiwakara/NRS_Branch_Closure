The Decline of U.S. Bank Branches

Bank branches have long been a cornerstone of the U.S. financial system, serving as the primary point of contact between households and banks. Over the past decade, however, the industry has undergone a marked transformation: branch counts are shrinking rapidly after nearly half a century of expansion. The figures below illustrate the core patterns and highlight the drivers of this shift.

Long-Run Trends

Figure 1 plots the total number of banks (blue line) and branches, or “offices” (yellow line), from 1980 through 2023. The number of banks has fallen steadily for decades, driven largely by mergers and acquisitions (M&A). The number of branches, in contrast, rose sharply until 2010, even as consolidation reduced the number of banking firms. At its peak, the U.S. had more than 90,000 branches. Since 2010, however, the trend has reversed: branch counts have fallen by roughly 20,000, with the pace of decline accelerating in recent years.

This turning point raises an important question: why did branch expansion persist through waves of consolidation for decades, only to give way to retrenchment after 2010?

Openings and Closures

Figure 2 decomposes branch dynamics into openings (blue line) and closures (orange line). Before 2010, openings consistently outpaced closures, resulting in net branch growth. After the Global Financial Crisis, the pattern reversed: closures exceeded openings every year from 2010 onward. The COVID-19 pandemic further accelerated the process, with closures spiking to more than 4,000 in 2020 alone.

The aggregate decline therefore reflects not only a slowdown in new branch creation, but also an active contraction of existing networks.

Depositor Behavior and Branch Profitability

Why do some branches close while others remain? A central factor is depositor sensitivity to interest rates, measured by deposit beta. Deposit beta captures how quickly depositors respond to changes in market interest rates by reallocating balances. Branches serving more rate-sensitive customers generate lower profits, as banks have limited ability to fund themselves cheaply in these markets.

Figure 3 plots branch closure rates across deposit beta deciles during three monetary tightening cycles. The pattern is clear: branches in the highest-beta areas are consistently more likely to close. In the most recent cycle, closure rates reached nearly 5% per year for the most interest-sensitive branches, compared with less than 2% in the least sensitive deciles.

These results underscore that depositor behavior, not just bank strategy, drives branch restructuring. Technology adoption magnifies these effects: households that are younger, more educated, and more financially sophisticated adopt mobile banking more quickly, making them both more interest-rate sensitive and less reliant on in-person branch visits.

Heterogeneous Incentives

The deposit beta mechanism also helps explain differences between incumbents and entrants.

Incumbent banks retain branches in low-beta markets, where depositors are “sticky” and provide a stable, profitable funding base. They exit high-beta areas, where depositors are costly to retain.

Potential entrants, by contrast, avoid low-beta markets. Sticky depositors are difficult to attract away from existing banks, creating a barrier to entry. Entrants target higher-beta areas, where depositors are more willing to move.

This asymmetry highlights why M&A has become the dominant mode of expansion: acquiring an existing network transfers its embedded customer base, whereas de novo entry into sticky markets is rarely profitable.

Large vs. Small Banks

Large banks exhibit stronger responses to depositor sensitivity than small banks. Their customer base is more affluent and digitally engaged, which makes deposits more mobile and branch networks more costly to maintain. Small banks, by contrast, serve many communities where customers continue to value in-person access, dampening the effect of deposit beta on closure decisions.

The Role of the Pandemic

The COVID-19 pandemic provided a structural shock that accelerated these patterns. Cell phone data on branch traffic show steep declines in visits between 2019 and 2021, particularly in areas with financially sophisticated populations. For many depositors, the pandemic lowered the perceived value of proximity by normalizing mobile and online banking. Branch closures spiked in response, especially at large banks.

Implications

The evidence suggests that deposit profitability—shaped by rate sensitivity and technology adoption—is the primary driver of modern branch restructuring. Lending plays a minimal role, reflecting the reduced importance of geographic proximity in credit markets.

The decline of branches has two main implications. First, local frictions in deposit and credit markets are eroding, likely increasing efficiency and integration. Second, communities face uneven impacts: areas with more sophisticated households are shedding branches faster, while those with less mobile depositors retain them.

The U.S. has thus entered a new phase of banking restructuring. The long era of branch expansion, enabled by deregulation in the 1980s and 1990s, has given way to contraction driven by technology and depositor behavior. If international experience is a guide—Scandinavian countries have seen reductions of up to 90% in branch counts—further declines in the U.S. are likely.